
\begin{center}
\includegraphics[width=0.9\textwidth]{OpenImage.png}
\end{center}

\section{First Section}

The three following biblical sketches will give, though doubtless only a faint impression, of why Professor Gederdrup’s lectures on the Old Testament so powerfully captivated his hearers. The noble, poetic language, the thoroughly biblical tone, the holy reverence for the Lord’s testimony as it came to His people through His preacher, were such prominent characteristics of this branch of his teaching that it seized the listeners in a peculiar manner and held them fast. When, with his clear gaze sharpened by the Spirit of God, he opened the Scriptures for his students and looked into the hidden depths of God’s counsel, many a heart was made to burn. — Ed.






\bigskip





\subsection{David and Jonathan}

[Source: Svartal-Skrift for the Norwegian Lutheran Church in America. Edited by Prof. Gederdrup and Oftedal. Third Year. 1877. Pages 44–48. — Ed.]

This section appears on pages 1–5 of the original volume. — Present Ed.

\bigskip

\begin{quote}
And because iniquity shall abound, the love of many shall wax cold.
\end{quote}

The time seems to have come in which this word is being fulfilled. At a deep and fundamental level there stirs the unrighteousness of self-love, and it appears as though petty vanity and rending envy have become dominant, not only in the politics of this world, but also among those who call themselves by His name, who went into death for His friends.\footnote{Sverdrup’s prose is woven like fabric held under tension and is best understood when read aloud. — Present Ed.} Even in the Church of God it seems that the gentle south wind does not blow, that wind which causes the flowers to give forth their fragrance in God’s garden, but rather a cold, cutting winter snow, which causes the fairest plants to wither and the noblest shoots to die.

Therefore we will bring forth the loveliest image of the love of friends that we can find, and by it we will warm our own hearts. For though thou shouldest take the coldest thing thou knowest—though thou shouldest take death’s cold mist\footnote{The original text uses ``Dødens kolde Taarning'' While Taarning resembles the word for``tower'' in this 19th-century context it functions as an archaic variant of taage/taaning (mist/fog). "Mist" better captures the intended metaphor of a chilling, numbing atmosphere that stiffens the heart. — Present Ed}, wherein human hearts grow stiff and the blood congeals—yet love is strong as death. And where the fire of love burns, there it is of no avail that the wind blows and the waters stream over it; for every cold gust shall only cause the flame to blaze the more freshly upon the hearth, and the many waters cannot quench love. In tribulation it grows, in darkness it shines, in cold it is the warmer; the greater the conflict, the greater the courage; the more hindrances, the more glorious the outcome.


A young man, ruddy of countenance, comely of form and fair to behold, was David, the son of Jesse, from Bethlehem. Scarcely more than a youth among bleating sheep upon the pastures of Judah; over its hills and through its valleys he led his flock; his playing and his song sounded over the green meadows and by the murmuring brooks. His harp he had learned to tune by the rushing of the rivers and in the stillness of the forests, and the praise of the Lord he had learned to sing while he walked alone upon the field where Abraham, Isaac, and Jacob had pitched their tents with their flocks, in faith and expectation of the Lord’s promise, that the land should be given unto them and to their children for ever.

The glorious inheritance of promise which had been given to the tribe of Judah had become the shepherd boy’s inheritance, and by it he had been assured that Goliath of the Philistines was not to be feared by the people of God, but that the Lord would give His people victory over all their enemies, if they boldly trusted in Him.

Then came a day when the prophet Samuel went down, by the command of the Lord, to anoint one of the sons of Jesse to be king over His people; for Saul was rejected from before the face of the Lord. And Jesse brought forth his sons, one by one, before the prophet of the Lord. And the prophet said, when he saw Eliab: “Surely the Lord’s anointed is before Him.” But the Lord said: “Look not on his countenance, nor on the height of his stature; for I have rejected him: for man looketh on the outward appearance, but the Lord looketh on the heart.”

And Jesse brought forth his seven sons, but none of them was chosen of God. Then David was lacking; he who kept the little flock had yet to come. And the Lord said unto Samuel: “Arise, anoint him; for this is he.” And Samuel anointed him, and the Spirit of the Lord came upon David from that day.

But over Saul there was an evil spirit; and David was brought to Saul, and his pleasant playing was a refreshment and a restoring of life for the sick and torn heart of the man, and the evil spirit departed from him at David’s playing. But Saul was rejected of the Lord, and David was chosen in his stead.

At this time the Philistines went up against Israel, and Saul summoned the host against them. And Saul had a son whose name was Jonathan, and he was a hero in war, and victory followed him against the Philistines, for he also had set his heart upon the LORD, and in the name of Israel’s God he smote the enemies of Israel. Among Israel’s heroes Jonathan was the flower, an ornament unto Israel, a joy and delight to the people of God. He was the king’s son, and the people rejoiced in hope of the day when he should sit upon his father’s throne.

Then came the great champion of the Philistines, the mighty Goliath, and set himself forth to single combat. And he reviled the armies of Israel, and he mocked their God, and there was none who dared to meet him. Fear and dread lay upon the whole host of Israel; but David was not among them, for he went about upon the fields of Bethlehem and kept his father’s flock. Yet he also would behold the Philistine champion, and he went to the camp; and the Spirit of the LORD came upon him. With a sling in his hand and smooth stones from the brook in his shepherd’s pouch, he went against the champion in the name of God. And against the champion with sword and spear he set the LORD as his shield and defender, and the Philistine fell before the despised shepherd-boy of Israel.

Then there was jubilation in Israel, then the daughters of the people sang with gladness the praise of David. But a dark and hellish thought crept into the heart of Saul; envy seized him, and bitter hatred toward David, because the LORD had chosen him and given him the strength of faith, which was the true kingly adornment among the people of the LORD. But the heart of Jonathan opened itself toward the unknown shepherd-boy; his heroic spirit rejoiced in the heroic deed, his faith was strengthened by David’s boldness of faith; he recognized the LORD’s Chosen and loved him. And as brooks that meet in the valley, gently and silently glide together and cast themselves into one another’s embrace, unable to do otherwise, so heart was bound to heart and soul to soul, when Jonathan and David found one another on the day of victory.

Saul’s hatred and persecution brought, from that day onward, tribulation and distress upon David. It was the Lord’s school of love, wherein the chosen king was to be prepared to become a true ruler of people and land. Thus, driven from the king’s court, threatened with death, hunted from place to place like a wild beast, he no longer had any home, any resting place, save in the Lord’s faithfulness and in Jonathan’s friendship.

Jonathan sheltered him from danger; Jonathan warned him against his father’s cruelty; Jonathan went to him in the forest and strengthened his hand in God. Jonathan defended him before his father, so that Saul, even in his fury, hurled his spear at his own son.

Was there anything for Jonathan to gain by such faithfulness to his persecuted friend? On the contrary, it seemed as though there was everything to lose. There were many bonds that surged against this love and would have quenched it, yes, even turned it into bondage to Saul. Temptation lay on every side for Jonathan to give way; yet in all things he was found faithful.

Saul, his father, was rejected by the Lord, and David was chosen in his stead. David was to receive Jonathan’s inheritance; the son was to lose all because of the father’s sin. Envy and wounded vanity might well have gathered bitterness in Jonathan’s heart; yet his friendly mind remained at peace with this thought: David shall be the first in the kingdom, and I shall stand at his side.

Jonathan was the pride of Israel; now a shepherd boy was to go before him. He who had gone foremost in Israel’s wars was to give way to one who had gone behind the bleating sheep. It seemed a bitter thought, but Jonathan found rest in the confidence that the Lord chooses whom He will to prepare His people.

Daily Saul’s kingdom declined; daily Jonathan had to see faithfulness to Saul give way to the growing sympathy for David. David’s nobility cast a heavy shadow over Saul’s faithless course. Jonathan felt how it cut through the heart to see his father overwhelmed with shame; yet in all this he prevailed.

And David, who was hunted and persecuted, who was not deemed worthy to have even a cave in the land where Saul had his throne—his soul too was surely tempted to hate both father and son. He must have feared what Jonathan would do when Saul was dead. He might well have thought it better that the whole house of Saul were destroyed, if he were to have peace upon Israel’s throne. There was temptation when he saw Jonathan alone and himself despised; there was temptation when he met him alone in the field, alone in the forest. Might he not think: “This is the heir; if he is gone, then the struggle is ended at Saul’s death”? Yet in all this David prevailed.

Then came the final battle. The Philistines attacked Israel; Saul and Jonathan went out against them. Alone the friend went with his blinded father. He fought for the rejected one; he himself had nothing to gain thereby. For David’s land fought Saul’s son; for Saul’s victory fought David’s friend. But David was not at his side. The tumult of slaughter increased, and the struggle grew fierce; Jonathan fell in the unequal fight. But David’s lament has borne Jonathan’s name unto this day: 

\begin{quote}
I am distressed for thee, my brother Jonathan: very pleasant hast thou been unto me; thy love to me was wonderful, passing the love of women.
\end{quote}

Why did Jonathan fight, why did David suffer? 

For Israel’s hope, which was greater to them than their own honor. 

Thus the Lord bound hearts together in unbreakable friendship and rock-fast faith for His people and its hope.