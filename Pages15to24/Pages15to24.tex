
\subsubsection{Hannah’s Song}

Samuel was to be consecrated unto the Lord, a Nazirite from the cradle unto the grave. Such was the meaning of the Nazirite calling: that, in a living image, they bore witness before Israel to its vocation and its standing. As Israel was the Lord’s covenant possession, so were the Nazirites to be a testimony thereof before the people. — Samuel was also to abide in the Lord’s sanctuary all the days of his life. Hannah was not to keep him at home; with the Lord was his home to be. He belonged to that God who had heard his mother’s prayer. In the sanctuary he was to grow up, that his soul, from childhood onward, might live with God, and that he might thus be strengthened and prepared for his work in Israel.

Hannah therefore kept him with her until he was weaned. Yet the vow unto God lay upon her and had to be fulfilled. Many a time must her eye have rested, with tear-mingled joy, upon the boy upon her lap, whom she was to give back unto the Lord. And many thoughts must have flowed through the soul of that godly woman as she looked with longing toward better times for God’s people, and pondered whether the Lord would suffer the little one to grow up into a chosen instrument for the working of His Spirit in the nation. Her thoughts doubtless turned back to Sarah, the barren woman who became the mother of Israel; to Rebekah, who had cried unto the Lord as she herself had done, and had become the mother of Jacob. And within Hannah there arose a hope which the Spirit of the Lord fashioned into glad assurance in her heart, that her son would become a means in the Lord’s hand to usher a new age upon God’s people. That which was scattered was to be gathered, that which was disturbed was to be made firm, and the Lord’s anointed was to grant the people the peace and freedom for which they sighed.

Thus time passed on, and the mother’s heart was enlarged more and more by the thought that the sacrifice she was to bring would be richly repaid through the great work which the Lord would appoint her son to accomplish for His people. The mother who has first learned to behold the people’s distress and the Lord’s consecrated love, she offers with joy her dearest child, if his life and labor, far from home, might bear blessed fruit for the whole people. True, the hut would seem empty when little Samuel no longer leapt smiling about within it; yet the mother thought of how, through her pain, a light might be kindled whose blessed radiance could fall into so many hearts and dwellings in Israel. And thus she remained faithful to her God and to the vow she had made.


The day came when Samuel must go with her to Shiloh. The light and strength of the land followed Hannah, and she went with joy along that road which she had so often before imagined as painful and heavy. To Shiloh, to the sanctuary, to Eli went the little company; and the least in that company was to become the greatest in the people of God. With deep and inward emotion Hannah stepped forward before Eli, and now Eli learned what Hannah had prayed and vowed on the day when she had spoken so bitterly and poured out her soul before the Lord. There stood Samuel, granted by the Lord; now he was to be presented back to Him.

But Hannah prayed again—not a prayer of pain as on that former day of sorrow, but a prayer of praise and joy, such as only one woman after her has sung. A song of praise the Spirit of the Lord laid upon Hannah’s lips, which was to resound again with unspeakable joy from the mouth of the mother of Jesus:

\begin{quote}

Glad is my heart in the Lord;
Exalted is my horn in the Lord;
My mouth is opened wide against mine enemies,
For I rejoice in thy salvation.

There is none holy as the Lord,
For there is none beside thee;
Neither is there any rock like our God.

Speak no more exceeding proud words;
Let not insolence come out of your mouth;
For the Lord is a God of knowledge,
And by him actions are weighed.

The bows of the mighty are broken,
And they that stumbled are girded with strength.

They that were full have hired themselves out for bread,
And they that were hungry hunger no more;
Yea, the barren hath borne seven,
And she that had many children languisheth.

The Lord killeth, and maketh alive;
He bringeth down to the grave, and bringeth up.

The Lord maketh poor, and maketh rich;
He bringeth low, and lifteth up also.


He raiseth up the poor out of the dust,
He lifteth up the needy from the dunghill,
To set them among princes,
And to make them inherit the throne of glory.
For the pillars of the earth are the LORD’s,
And He hath set the world upon them.

The steps of His faithful ones He preserveth,
But the wicked shall be silent in darkness;
For not by might shall a man prevail.

The LORD—His adversaries shall be broken to pieces;
Over them shall He thunder in heaven.
The LORD shall judge the ends of the earth;
He shall give strength unto His king,
And exalt the horn of His anointed.

\end{quote}

It is no longer the plaintive cry of a wounded woman’s heart; it is a mother’s jubilation, a believing woman’s thanksgiving for answered prayer, a daughter of Abraham’s joy over the hope of Israel. God’s holy longing for the time of refreshing has gathered itself within Hannah’s soul, and in true prophetic spirit she has, in the birth of Samuel, beheld a token of a new season for Israel.

The advent of a new age—a reformation, a revolution if one so will—Hannah proclaims in her song. As the first green shoot in spring bears tidings of thousands that shall follow; as the first leaf gives witness to the mighty powers which soon shall clothe the whole forest in a new raiment; so Hannah sees in Samuel’s birth and childhood a sign that the Spirit of the LORD is mightily stirring within the people of the LORD, and will work a renewal of the entirety of its national life. There is spring in the land, and spring-life in the people. Samuel is the first message of what is to come. We have seen the snow melt in the spring; and whereas before it lay cold and hard upon the grass-shoots, it now sinks down and gives moisture and sap to the roots, so that it must help forth the grass which it once held down.

Thus is the work of the Lord’s Spirit in Israel. That which hitherto has stood highest and lain heavy and oppressive upon Israel’s life shall be cast down and removed; but that which hitherto has been crushed and held in bondage, this will the Lord lift up and set free, so that with its new, young, fresh life it shall become a blessing for the people. Hannah sees in the Spirit how the strong and the rich, the satisfied and the proud, the insolent and boastful—who until now have been the leaders and oppressors of God’s people—shall receive their judgment and be thrust aside to make room for the lowly and poor people, who, renewed and born again by the Lord’s Spirit, shall go before in the struggle and lead it to victorious end. And the mighty stirring which the Lord will bring into the dead bones shall not cease until it transforms Israel into a kingdom. God’s people, who for a long time have dreamed of a king who might gather the people into unity and guard its independence, shall now soon receive him; and anointed with the Lord’s Spirit, he shall understand Israel’s high calling and lead the people forward upon its appointed path.

This is ever the Lord’s way. When human striving after height and power ends in inward hollowness and lifelessness; when the Spirit has departed from those who stand at the top and only the outward show remains, while the people gasp in deathly anguish for a little refreshment for their tormented hearts—then it is the Lord’s hour to send, in merciful love, the fresh breath of the Spirit over the young and hope-weary hearts. Then He sends awakening among the people; and while the spiritless great men turn away with cold contempt from the call of the Spirit, the Lord kindles light round about in the hearts of the people. It is deemed fanaticism and folly; it is hated and persecuted, mocked and derided; yet it carries on its quiet growth to the full. The Lord’s hour is near; and it comes with crushing judgment upon the crafty faces and the jubilant feasts and the mighty oppressors; yet it comes also with quiet peace and eternal honor for those who have learned to wait upon the Lord.

Many times has the Lord, through great and small alike, revealed Himself in this manner; but never with clearer hand than when He sent His Son into the world. That the lowly man from Nazareth with His poor Galileans should stir the world from one end to the other, undermine the mighty Roman Empire, and build His Church upon its ruins—this is the greatest and most glorious revelation of that law of God: the Lord bringeth low and lifteth up together. A mightier empire has never been raised than the Roman; and greater misery and wretchedness than that of Galilee’s poor, afflicted, leprous people has scarcely any seen. Yet from despised Galilee went forth the Word which overthrew the empire and built up the Church.


But Samuel was to be the instrument of a like work. Israel’s people were to experience a spiritual awakening, which should transform both its inward and its outward condition. Base as Israel already was, it was to be brought yet lower. That which had become its pride and its vanity it should lose; yet in its uttermost need the Lord would prepare strength for it through Samuel’s prayer. Then Israel, trampled down and crushed beneath its enemies, should rise up in the strength of its God and strike its foes back, blow after blow. As a lion should the Hero of Judah step forth against the enemies of God’s people, and in confusion should they retreat before him.

Such was Hannah’s bright hope and faith when she brought the heavy sacrifice of leaving Samuel behind at the sanctuary, while she herself departed quietly on her way. Were there more mothers like Hannah, there would also be more sons like Samuel.

For a mother’s prayer availeth much; and Samuel, who was dedicated for his whole life with tears and with jubilation, should not put to shame the hope that was bound up with him. Of life he as yet understood only life’s sorrow and life’s joy; but there are dreams in the hearts of children which no one understands, and there are moments that make indelible impressions upon the young. Such a moment it must surely have been for the little boy, when his mother’s holy enthusiasm shone upon him, and when he saw her so glad and so deeply moved. He may well have remained behind with childlike sorrow and fear; and yet he may also have forgotten his childish dread in the Lord’s sanctuary, where he was left alone.

\subsubsection{The Judgment upon the House of Eli}

By Hannah’s song of praise Samuel was consecrated to be a reformer in Israel. Yet a reformation has its preparation through long ages of abuse and spiritual deadness. When the commandment and the guilt in the existing order reach a certain boundary, and when corruption advances so far that it cries to heaven for vengeance, then the day of the Lord comes with crushing judgment upon the old. But reformation is not merely judgment upon the reigning corruption; it is also salvation and deliverance for that which has lain groaning and longing for the Lord’s light and the life of the Spirit. Therefore a reformation has not only its negative preparation in ancient obstinacy and stiffening injustice; it has also its positive preparation in a people through the cry of oppressed and persecuted hearts for salvation, through longing for deliverance, through hope of brighter and better times. Judgment and salvation, destruction and upbuilding, go hand in hand in the Lord’s householding with His people; and it was not to be otherwise on this occasion.

Samuel had already in his home seen much of the good powers at work in the life of the people of Israel. There must have been peace and joy in his mother’s quiet homestead; and the blessing she had experienced must surely have rested upon her whole life with thankfulness, and have laid piety’s own imprint upon her entire being. A song of praise such as Hannah’s does not well up from a false and impure heart, nor does it overflow from light-minded lips. The hidden man of the heart, in the incorruptible ornament of a meek and quiet spirit, must assuredly have been this woman’s adornment. And Samuel, without knowing it, bore within himself the woman’s inheritance from the mother who hoped in him to behold the Lord’s instrument for the awakening of the people.

It is all the more grievous, with this thought of Samuel’s home, to pass over to the consideration of the ungodliness with which Samuel was to become acquainted in the Lord’s sanctuary. For from there the light of holiness and the life of love ought indeed to have streamed forth over the little people, who now by their spiritual strength were to raise their standing in relation to the mighty heathen nations that surrounded them on every side. From the sanctuary the Lord’s truth and the Lord’s Spirit should have gone out to all the scattered members of the congregation and bound them together into a living body. The bloody sacrifices were to be signs of atonement and reconciliation for the sinful people, while the priest’s blessing was to follow them back into their daily labor and lay the Lord’s power into their work. There, thought Hannah, the little Samuel must surely grow up amid pure and holy surroundings, where the child-heart’s guilelessness and innocence would not be disturbed by the many temptations that a coarse popular life brought with it.

But it was not so in the Lord’s sanctuary. Defiled and profaned was the house of God by those who were the Lord’s priests and who by their calling ought to have been the light of the people. Old Eli himself had been a man of somewhat blameless conduct in his dealings. But stripped of the Spirit’s power, and blind to the Lord’s holiness and the people’s need, he had not been able to stem the tide against the ever-growing corruption among the priests. His own sons became representatives of this spiritual decay in its most grievous embodiment.

It is sorrowful to say it; yet it is no less true, that the sons of Eli are not the only sons of priests, nor the only priests of their kind. It is dreadful for any house and for any man to be without the Spirit of God and without sincere fear of God; but it is, if possible, doubly dreadful for a priestly house and for a priest. For where the holy calling to be the Lord’s servant is daily denied and daily violated by a profaning hand, there is gradually laid so thick and close a shroud of corruption over heart and soul that at last it becomes an impossibility for the two‑edged sword of the Word to pierce through and judge the thoughts and counsels of the heart. Where the holy calling is used merely as a means of gain, the spiritual power finally degenerates into an ungodly cowardice which stifles all spiritual influence. It is fairly well known also in the spiritual experience of our people, that the Lord’s judgment has not spared ungodly and spiritless priestly houses; for Eli’s fearful judgment, to behold his own weakness and slackness punished by the corruption and ungodliness of his sons, sounds a grave and piercing shout from the masthead\footnote{For this section, the term Varsko is rendered as a "shout from the masthead" to reflect its maritime origins and the seafaring culture of 19th-century Denmark. The use of "shackled" preserves the "Chain of Sin" motif, maintaining the rhythmic, "sung" quality of the original vertical imagery—contrasting the high calling of the Spirit with the inescapable descent into the abyss. — Present Ed} to all who are the Lord’s servants, that they be so in spirit and in truth, lest the judgment swiftly begin from the house of the Lord.

The sin of Eli’s sons, their indulgence and sensual lust, was all the more detestable in that the Lord’s sacrifices were thereby despised and the Lord’s sanctuary profaned. Their father’s reproof they treated with scorn, and the shame and pain which they inflicted upon the old man were to them a matter of indifference. All love and all right had to yield to their coarse and brazen craving for sensual pleasure. And Scripture adds a heavy, dreadful word: “They hearkened not unto the voice of their father, because the Lord would slay them.” It is one of the mysteries of divine justice, that where the Lord’s hour has come for judgment, there He lets sin pass over into hardening. Sin begets sin; one sin, as it were, forces forth the other; one link in the chain shackles to the next, and draws the sinner—first slowly, then more swiftly, at last without restraint—down into the abyss of perdition.

But Samuel’s childlike mind remained uncorrupted by the foulness that surrounded him. And he grew in stature and in favor both with the Lord and with men.

Over the house of Eli the judgment drew near. The Lord did not leave him without warning. A prophet announced to Eli that the wrath of the Lord would come upon him and upon his lineage. If the priesthood had dishonored the Lord, then He would cause shame and disgrace to come upon the priesthood. That which He had set highest in the people of Israel, and made the true bearers of the whole of Israel’s spiritual life, He would cast down into the deepest contempt and misery. The house of Eli would not die out; yet a faithful priest would take his place, and the wretched descendants of Eli’s house would beg and implore him for a priestly office, that they might obtain a piece of bread to eat. As the sons of Eli had transformed the Lord’s holy priestly office into a means of self-indulgent luxury and sensual pleasure, so in return the descendants of Eli would be made to experience the full bitterness of having to beg for a priestly office for the sake of hired bread. Eli himself was to behold the sign of this wretched future: his two sons would die in one day.

As Samuel grew, the rising contrast between what the priests were and what they ought to have been must have presented itself in sharp and glaring light to his childlike mind. There is nothing for which a child has a keener eye than hypocrisy and falsehood. None is more authoritative than a faithful and God-fearing child; and as the clearest mirror bears witness to every glance and desire in a human soul, so a child’s soul, the purer and nobler it is, becomes all the more offensive to every form of dishonesty in its surroundings. With ever-deepening dread Samuel beheld the profanation of the Lord’s sanctuary. Thus he was being prepared for the Lord’s first revelation to him; for that revelation was to be a new warning of judgment upon the house of Eli.

Three times the Lord called to Samuel before Samuel, at Eli’s bidding, answered: “Speak, Lord, for Thy servant heareth!” Yet this was not only spoken from the depths of the young heart; it became as a constant and unceasing cry from Samuel’s inmost soul throughout his entire subsequent life. Faithful, and even grave as he was, it became the earnest striving of his life to be open and receptive to the Lord’s voice, to lay his ear to His mouth, to listen to His speech by day and by night, that he might become for his people a voice of the Lord, crying in the wilderness. As John the Baptist went before the Messiah, so Samuel went before the reign of David and Solomon with a clarion-cry of awakening. Samuel’s first vision was a message of death over the house of Eli. Thus it was that the Lord would first reveal Himself in this time. With judging righteousness He would strike down that which was set in high station among His people; for His honor was violated by the ungodliness of the priests, and the people were led into dreadful labyrinths\footnote{The term``labyrinth'' for Afveie (literally ``sideway'' or ``wrong paths'') reflects the moral disorientation caused by corrupt leadership. — Present Ed.} by those who should have been its leaders. Yet no one could well have foreseen in what strange manner the Lord would bring His counsel to fulfillment.

Samuel rose from day to day in the people’s acknowledgment. His prophetic voice accomplished what no judge or hero had been able to accomplish since the death of Joshua. He gathered the whole people of Israel, from Dan to Beersheba, around his prophetic voice. This is his first great and significant work in Israel. Divided and torn as Israel was, Samuel’s voice was nevertheless heard throughout the entire people. It was something altogether new in those days that all Israel should be gathered by a spiritual means and around a prophet of the Lord. Yet much was still lacking before Israel was wholly permeated by the Spirit of the Lord.

Yet it was an immediate reversal. It was not long since Samson had stood alone against the Philistines, yes, since even his own countrymen would have delivered him bound into the hands of his mortal enemies. Now the whole of Israel gathered at Samuel’s word for battle against the oppressors.

The encounter took place upon the ancient battlefield, where the Lord would later glorify Himself by hearing Samuel’s prayer. But this time Samuel had gathered the people in order to let that let the hammer blow of judgment fall,\footnote{The Danish 'Dom bryde ind' suggests a judgment breaking in or bursting through. I have rendered this as 'hammer-blow' to maintain the rhythmic force of the sermon. — Present Ed.} forming the beginning of Israel’s renewal. Israel was smitten with great loss, and now it was to be made manifest how far the priesthood of Eli and his sons had led the people away from a living knowledge of God.

For not in repentance and prayer, but in heathen superstition did Israel seek the help of its God. Not to the living God did they flee, but to the Ark of the Covenant, which they supposed would help by its outward presence. It had been given to Israel as a sign and as the nearness of the covenant. It bore within it the holy tables of the Law of old, hidden beneath the mercy seat, where the blood made atonement for Israel’s sin. It was therefore a living witness that the glory of the Lord could dwell in the midst of a sinful people only through the mystery of atonement, and that broken and contrite hearts alone were sacrifices well-pleasing to the Lord. But this outward sign of the covenant did not bring a penitent and superstitious people into the Lord’s good pleasure. Nowhere has Israel’s spiritless reliance upon the outward signs of election revealed itself more clearly and more repulsively than in this holy matter, and it is impossible to conceive of a more dreadful judgment upon their heathen superstition than that which befell them.

The priesthood of Eli and his sons had, itself spiritless, hypocritical, and hollow, turned the people to regard the Lord’s covenant sign with spiritless and profane hearts. They regarded the Lord as a God who, being dependent upon His people and His sanctuary, must help wherever His people with outward exactness observed the ordinances and ceremonies. And Israel had gone so far astray from the worship of God in spirit and in truth that it supposed that when the Ark of the Covenant was present, then the Lord was present. Much the same are those among the so‑called Christians who imagine that outward ordinances and certain ceremonies and certain doctrines and similar outward forms will help them, while true spiritual fear of God and living devotion to Him are a matter of indifference to their hearts.

Israel fetched the Ark of the Covenant from Shiloh. With jubilation and joy it was received; with fear and terror the Philistines heard of its coming. Israel’s God had so mightily revealed Himself in Israel’s exodus from Egypt that the memory thereof still lived on among the heathen nations; and the Philistines feared that mysterious power which had once been so dreadful to mighty Egypt.

But the Lord would hold judgment. Heathen superstition, idolatrous reliance upon the outward sign of the covenant, the spiritually lifeless priesthood, and the profaned house of Eli were to receive their deserved recompense. The way was to be prepared for that spiritual knowledge of God which, through Samuel’s awakening word, would fall as living seed into the hearts of the people and bring forth the noble fruits that lie before us in David’s Psalms. From the coarsest outwardness to the deepest inwardness the way led only through a fearful judgment of the Lord.

The Ark of the Covenant was taken by the Philistines. It was a shattering blow to Israel’s hollow worship of God. Their confidence and their strength were gone. The sons of Eli fell. Eli himself perished, unstrung\footnote{The Danish 'gjennemrystet' means 'shaken through.' The term 'Unstrung' is used to reflect the total collapse of Eli’s physical and spiritual frame at the news. — Present Ed.} by the horror of the calamity.

The mighty day of judgment ended with a heart-rending event which gathered into itself, as in a mirror image, all the features of Israel’s misery. The wife of Phinehas was seized by terrible birth pangs at the tidings of disaster that came from the battlefield. She breathed out her life amid the agony and called her son’s name Ichabod; for, said she, the glory has departed from Israel, for the Ark of God is taken.

The woman dying in childbirth, who seems to see all as lost, yet gives a son life in her hour of death. Thus sinks old Israel, the old time, into death, when the Ark of the Covenant and the priesthood are in one day trampled to dust; but a new Israel and a new time stand at the door; for the Lord lives, and Samuel is His Spirit-bearing instrument. As yet the new time is only as a newborn child, yes, as a child stirring in the womb\footnote{The Danish 'uroligt Foster' literally means a 'restless fetus.' In this homiletic context, 'stirring in the womb' captures the hidden, internal nature of God’s new work before it is visible to the world. — Present Ed.} in the eyes of men; but in its time it shall be seen that salvation and redemption sprout from the soil of judgment.
