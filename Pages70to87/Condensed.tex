\subsection{The Church and the Ecclesiastical Communions}
\textit{Condensed}

\bigskip

This section presents a paraphrase of Georg Sverdrup’s essay *“The Church and the Ecclesiastical Communions.”* The purpose of this version is not to reproduce the author’s text in full, but to make the **structure of his argument** accessible to readers who may find the original nineteenth-century prose, length, and rhetorical density difficult to follow at first encounter.

In preparing this paraphrase, **no new theological positions have been introduced**, and none of Sverdrup’s central claims have been altered. What has been omitted are (1) his extended polemical expansions against specific historical forms of visible church unity (especially the Roman papacy and later Protestant parallels), (2) many of his long periodic sentences, rhetorical repetitions, and escalations in tone, (3) his detailed logical refutations expressed in layered, cumulative form, and (4) his subsequent section of “historical illumination,” which applies these principles to concrete historical developments. The paraphrase therefore preserves the *trajectory* of the argument while removing much of its *texture*.

It must be emphasized that what is lost in such condensation is not merely length but **eloquence**. Sverdrup’s original discourse was not written as a neutral academic treatise but as the literary trace of a powerful spoken theology. His argument is carried as much by rhythm, cadence, irony, and rhetorical pressure as by logical sequence. The original text unfolds through carefully balanced contrasts, repeated returns to key themes, and a mounting intensity that cannot be fully reproduced in outline or paraphrase. Readers should therefore regard this condensed form as an orientation and guide, not as a substitute for the original — and, where possible, should return to Sverdrup’s own words to experience the full force, beauty, and seriousness of his thought as it was first delivered. — Present Ed




\begin{quote}
But now are they many members, yet but one body.\\
1 Cor. 12:20
\end{quote}

\bigskip

\textbf{Christ the Head, the Church the Body}

The congregation is not Christ’s unless Christ is its Head, and it is not His congregation unless it is His Body. This truth stands firm even where it is little acknowledged. Wherever Christ’s Word and Sacraments are, there the Body of Christ is present. No one taught by the Spirit to call Jesus Lord is ignorant that he belongs to this Body, for he knows that Christ’s life flows through him by the one Spirit, making many members one Body.

\bigskip

\textbf{Different Members, Different Work}

Just as surely as every Christian knows he is a member of Christ’s Body, he also knows that not all members have the same task. No one imagines that only pastors, teachers, or elders belong to Christ, or that others have no share in Him. Such talk is rightly judged madness.

\bigskip

\textbf{The Dangerous Claim of One True Communion}

Yet another kind of speech has long troubled the Church. It says: because Christ’s Body is one, only one ecclesiastical communion can truly be Christ’s Church. From this comes the charge: “Because you are not as we are, you do not belong to Christ.” This claim confuses consciences, stirs strife, and must be judged by the Word of God and by history.

\bigskip

\textbf{Unity Misunderstood}

It may seem obvious that among many communions one must be the true Church. But this forgets that a body is not one member, but many. God, who gives diverse gifts within a congregation, may also distribute diverse callings among many communions, each with its own labor in the world.

\bigskip

\textbf{The One True Starting Point}

All Christian thinking about the Church must begin here: the Church is Christ’s Body, and Christ is its Head. Wherever this truth is forgotten, confusion follows. We must return to it if we are to see clearly.

\bigskip

\textbf{False Visible Unities}

All agree that the Church is one, but disagreement arises when men try to define this unity by visible things. Some claim it rests in a single man, others in an office, an order, a discipline, or a system. Each seeks to claim for itself what belongs only to the whole Body.

\bigskip

\textbf{The Error of Rome}

The Roman Church claims that unity with Christ is secured through unity with the Pope. But this makes the Church’s unity outward and visible, and replaces the living bond with Christ by dependence on a man. No throne or crown can guarantee that one belongs to Christ, much less make him the bond of unity for others.

\bigskip

\textbf{No Human Bond Can Unite with Christ}

The same error appears when unity is sought in bishops, pastors, assemblies, or discipline. All such bonds may unite men outwardly, but they cannot unite the visible Church with the invisible Christ. Wherever unity is made visible, Christ’s honor is diminished and human pride is strengthened.

\bigskip

\textbf{Pure Doctrine Is Not the Unity}

Others claim that pure doctrine alone is the mark of Christ’s Church. This sounds pleasing to those who think they possess it, but it is a dangerous error. Even the purest doctrine cannot prove that a church body lives in union with Christ. Doctrine, polity, discipline, and authority all fail at the same point: none can give life.

\bigskip

\textbf{Life Is the True Unity}

The Church’s unity is unity of life. The Body is one because the life of Christ flows through all its members. Wherever this life is found—among Jews or Greeks, in this communion or that—there is the Body of Christ. Its unity is invisible, living, and real.

\bigskip

\textbf{One Birth, One Faith, One Hope}

This life begins with one birth: Baptism by water and the Spirit. It has one essence: faith, working through love. And it moves toward one goal: eternal glory with Christ. Thus the Church’s unity is unity in Baptism, unity in faith, and unity in hope—unity in the life of Christ.

\bigskip

\textbf{Why No Single Communion Can Claim the Body}

No single outward church body can claim to be alone the Body of Christ, for Christ’s life is not confined to one fellowship. Nor can any point to itself and say that all its members without exception live by Christ’s Spirit.

\bigskip

\textbf{The One Invisible Church}

Though there are many churches, the Church is one. Men may divide outward fellowships, but they cannot divide Christ or His Body. The unity of the Church is invisible and known by faith, yet it is stronger than all division.

\bigskip

\textbf{Why Many Communions Exist}

The division of the Church serves the growth of the Body when each communion carries out its God-given task. Just as a body needs many members, so the Church grows as each communion develops its particular calling under the same life.

\bigskip

\textbf{When Division Becomes Sin}

Division harms the Church only when one communion denies another’s right to exist. Then peace is broken, the Spirit is grieved, and God’s work is hindered. Pride and the hunger for power, not diversity of labor, bring the Church to shame.

\bigskip

\textbf{The Right of a Communion to Exist}

A communion has the right to exist if it lives from Christ and serves a real task in His Church. Where God has given a work to do, no man may forbid it. But where a communion claims exclusive right to Christ and condemns all others, it bears the mark of Antichrist.

\bigskip

\textbf{Not Indifference, but Struggle}

The truth that many communions may exist does not mean all are equally true. Some grasp the knowledge of Christ more deeply than others. Therefore struggle is necessary—not a war of destruction, but a striving for truth.

\bigskip

\textbf{Struggle Under Love}

Because sin remains, every communion is tempted to overvalue its own task and fall into one-sidedness. The struggle between communions checks error and brings truth to light when it is guided by love for the truth rather than love for power.

\bigskip

\textbf{Love’s Narrow Way}

The Church must walk between two dangers: surrendering truth for peace, or crushing others for pride. The narrow way of love holds fast to every grain of truth, yet lets go of every recognized error, whatever the cost.

\bigskip

Thus the Church remains one Body with one life, even while it appears in many communions, and the struggle for truth, rightly waged, serves not its destruction but its growth.
