
\begin{center}
\includegraphics[width=0.9\textwidth]{OpenImage2.png}
\end{center}

\bigskip


\subsection{The Church and the Ecclesiastical Communions}

Source: Quarterly Journal. First Year, 1875. Pages 20–42; 70–82; 145–153; Second Year, 1876. Pages 6–21; 51–65; 110–128.
See the note, page 58. — Ed.

This section appears on pages 70–87 of the original volume. — Present Ed.

\bigskip


\begin{quote}
But now are they many members, yet but one body.
1 Cor. 12:20
\end{quote}

It is a truth of God revealed in Christ, that the congregation is not Christ’s congregation unless Christ is its Head, and the congregation is not Christ’s congregation unless it is Christ’s Body. And however little this truth is acknowledged, and however scant a trace we see of the Church, with the full earnestness of life, seeking to present the Body of Christ in the world, it nevertheless belongs to those things of which all have knowledge, that the Church is Christ’s Body, and that each single congregation, with Christ’s Word and Christ’s Baptism, with His Supper, is a manifestation of Christ’s Body. And no one whom the Spirit has taught to call Christ Lord is ignorant that he is a member of Christ’s Body; for he has perceived and experienced how Christ’s life is his, how it flows through him, ennobling him by the one Spirit, by whom we are all baptized to be one Body.

But just as little as any Christian is ignorant that we are thus all members of Christ’s Body, just so little is it to him a strange or new thing that not all members have the same work. No one is surely so devoid of understanding of the form of Christ’s Body and of the Spirit’s diverse gifts of grace in the congregation that he should think all the members of the congregation must necessarily have the same work in the congregation. No one is surely so simple-minded that he says: because you are not a pastor in the congregation, therefore you do not belong to Christ’s Body; because you are not a teacher, therefore you are no Christian; because you are not an elder of the congregation, therefore you have no part in Christ. Indeed, we are almost all agreed that such speech is madness.

But there is another speech, which indeed does not come to us with the simplicity and lowliness of God’s Word, but with much cleverness and human authority, and which for that very reason ought to be suspicious to us; there is another speech which is just as perverse, but not as plain-spoken. It too is a speech about Christ’s Body, about Christ’s Church, which has fallen into distress, and which therefore has borne the bitterest fruits in the history of the Church and bears them to this very day. And now it sounds round about us, and it confuses consciences and stirs up strife and discord where there ought to be forbearance and love. Therefore we must consider it well and closely; therefore we must view it in the light of the Word and of history and judge it accordingly.

It is this old speech we have in mind: that just as Christ’s Body is one, so the Church also is one; of all the many ecclesiastical communions there is only one which is in truth Christ’s Body, His one true Bride. Therefore the one church body asserts against the other: Because you are not as I am, therefore you are not Christ’s Body; because you do not belong to me, therefore you do not belong to Christ. Therefore it is heard here in America so often among us: Because you are another communion than we, therefore you cannot be Christ’s Church.

And if this question did not stand so near to us, and with such necessity press demand an answer, we should not treat it here; for time is as precious as it is serious, and no Christian in these days can have time for useless and contentious questions, such as it is the duty of Christians at all times to flee and avoid.

There is something which seems so immediately clear—this, namely—that among the many ecclesiastical communions which call themselves by the name of Christ, there must yet be one which alone is the true one; one communion, surely, must be Christ’s real Church and His real Body. And yet that which seems so superficially clear is very far from being true. For it has been forgotten that the Body is not one member, but many. Or could one really say: one member must surely be the real, the true Body? Or would one not at once perceive that this is altogether inverted talk? Or should God—who within the single congregation is mighty to give to each his own gift, and thereby to set each one in his proper place—not have, for the whole Christian Church, many and rich gifts to distribute, which of necessity require that different communions take up each its own task, each its own labor in the world?

Therefore we must not allow ourselves to be dazzled by the seductive clarity of such talk, but with a steady eye fix our gaze upon it and examine—not its appearance, but its power.

There is a fundamental truth which must stand fast as the point of departure for all contemplation of the Church that would be Christian; it is that which we set forth at the beginning of this treatise: that the Church is Christ’s Body, and Christ is the Head of the Body. Every discourse concerning the Church which does not point back to this must necessarily become false, because it rests upon false presuppositions. And it is in the forgetting and the setting aside of this truth that we seek the root of all the confusion and all the obscurity which in our days prevail in this question. Therefore we believe that it is timely now to attempt to hold forth what lies within that fundamental truth, within that comprehensive principle, for consideration by those who, despite the self-love of party strife and its blindness, have yet preserved a small remnant of that humble simplicity which is necessary to recognize the truth in its purity.

It has been the confession of the Church through all ages and in all its divisions, and in every place where it has come under the governance of the Lord, that the Church is one. And it is indeed self‑evident that, if it is the body of Christ, then there can be no doubt that it is a unity. Nor is it here that the dispute hinges; for all parties confess with one mouth that Christ is not divided, and His body just as little. But when we go a step further, we immediately encounter a swarm of contradictions to the simple truth. For if we ask wherein this unity consists, we receive a multiplicity of answers which, by their mutual disagreement, show only that here each of the ecclesiastical communities appears with its own party interest and seeks to usurp what belongs to the whole Church—namely, to be the Body, to be the Bride.

We find a multitude of visible things set forth as that in which the unity properly consists; and each of these visible things is recognized by a particular church body and brought forward in order that it may be made evident that precisely this particular community possesses the true unity, is in truth the body of Christ. And closely bound up with this is the conclusion that all other communities have no right to exist alongside the Body, but must hasten to relinquish their independence and become parts of the one community which, because it is the Body, has the sole right to exist.

Here we first encounter the Roman Catholic Church, whose claim to be the true body of Christ and to possess the right unity seems to many so crude and palpably false that it appears scarcely worth the trouble to examine it. And yet in reality it is the only one that has some semblance of truth with which it may adorn itself. And if we succeed in showing its falsity, then the labor of dealing with other claims concerning the unity of Christ’s Church will be easy.

The Roman Catholic Church maintains with unshakable definiteness that the unity of the Church is precisely its unity with the Head, Christ, and in this claim it has the clearest possible right. But this unity with the Church’s invisible Head is mediated through Christ’s visible vicar upon earth, whom He Himself has appointed and upon whom He has built His Church. And therefore the unity has become a visible unity. It is the infallible Pope who in reality is the Church’s unity, and the entire visible community which acknowledges him and receives all truth and all the gifts of God’s grace from him is precisely the body of Christ. And every other community is outside the Body because it stands outside connection with the earthly head of the Body, infallibility in Rome.

This entire collection of claims would be utterly impossible if there had not been committed a breach with the presupposition that was indeed correct. The Church’s unity as its unity with the Pope is substituted for the real bodily unity with the invisible Christ. The error here lies in this, that one makes into an outward thing what in itself must be invisible and inward. Even if the Pope were ever so much Christ’s vicar upon earth, he could nevertheless never be the unity of the Body; for then it would at least have to be absolutely certain that he himself belonged to the Body. But now we all know that neither papal throne nor papal crown can secure to anyone that he is a member of Christ’s Body, still less that he should be a nexus of unity.

Thus it avails nothing that the Pope be ever so infallible; for no infallibility can secure him that he has any share in Christ. But if it is not even with absolute certainty infallibly sure concerning the Pope himself that he has a share in Christ, then it is still less certain that those who have a share in the Pope have a share in Christ. And thus it becomes altogether impossible that the community which stands in dependence upon the Pope can be Christ’s true Body and His rightful Bride; for no one can assert that this community, through the Pope, has any life‑union with Christ.

We see, therefore, that the claim that a single man is the Church’s unity is a manifest error, a breach with the correct premise upon which all knowledge of the Church’s unity stands or falls. But no less false is it to make an estate, or rather the unity of an estate—for example, the episcopate—the Church’s unity; for even if one were to take together all the bishops of the world, nothing would thereby be demonstrated, that their office were Christ’s office, or that there were any sort of connection whatsoever between them and the Head, Christ. It is once again the same error as before; for once more one seeks to make the invisible, to some degree, visible.

No human being is thus the unity of the Church, nor is any human estate—be it anointed bishops or ordained pastors. No such person binds together the Body of Christ, and no one is excluded from the Body of Christ because he stands outside communion with the Pope or outside connection with the highly-placed bishops and priests. Yet we encounter other assertions concerning the unity of the Church, which would vindicate the right of particular church-communions to be the Body of Christ. It is said that it is the grossest of all errors to claim that the Pope is the unity and head of the Church; it is the greatest of all lies to say that the Body of Christ is there, and only there, where he alone rules. But, it is added, the Body of Christ is therefore not invisible, such that one should not know where it is and where it is not. The assembly—the right and sound church order with the vigorous exercise of church discipline—that, in truth, is the mark of the Body of Christ. Bring all Christians to acknowledge this, and the unity of the Body of Christ will be present and will consist in the unity of the assemblies. Let each congregation teach as it will, and administer the sacraments as it will; the unity will still be there, and we shall have the Body of Christ in its proper form. But we again hold fast to our proposition, well grounded in the Word of God, by which we judge all things; and we say again: the assembly is not the unity of the Body of Christ; for there is no assembly in the world—however much church discipline may be bound up with it—that secures for the congregation which possesses it, acknowledges it, and employs it, any place in the Body of Christ as a member thereof.

With these attempts to find the unity of Christ’s Body and thus to form a single church fellowship, to which all others must attach themselves if there is to be one flock and one Shepherd, men have therefore come utterly to grief. And since men never grow weary of building their own fabrications of the heart alongside the simple truth of God’s Word, and never finish devising means whereby they may lay hold of Christ’s honor for themselves, it should not surprise us if ever new attempts arise within the Church to present a single church fellowship as the only true one, which alone bears upon itself the true marks of Christ’s Church, and which alone possesses the true unity of the Church, to which all others must come if they are to have part in Christ.

Pure doctrine seems in these latter times to be made the true unity of the Church; and indeed it has a very pleasing sound in many ears when it is declared to them that a church fellowship which does not possess the pure doctrine cannot possibly have any connection with the Head, Christ. For when pure doctrine is in the world, given by God, then it is, so it seems, plain enough that all who do not acknowledge it deny God’s truth and give themselves over to the lie. But how should such men of the lie have part in the eternal Truth, Christ?

This sounds like sweet and pleasant music in the ears of those who suppose themselves to have the pure doctrine; and it rises like smoke into their nostrils when it is said: We who have the pure doctrine, we are Christ’s Bride, we are His Body. To us, who have the light, must everyone come who does not wish to perish in the darkness. Yet however well-sounding this speech may be, it is hollow as sounding brass and a tinkling cymbal.

And there is in that speech a power of deceivers, which is thus prepared for the congregations. For where this talk of pure doctrine is heard, and where it is said that pure doctrine is the true mark of Christ’s Bride, there men are robbed by pride, and there they are lulled to sleep by false security, and they sleep upon the pillow that is thus laid beneath their head. And if it were indeed the truth that Christ’s Church is there, and there only, where pure doctrine is found, then this would have to be preached, and the danger would have to be borne, and souls would have to perish because they took the truth in vain. But now it is a great error and a dreadful lie that the unity of Christ’s Body and the sole true mark of Christ’s Church is pure doctrine.

For even if a church fellowship possessed a doctrine as pure as that of the Pharisees upon Moses’ seat, as pure as all the subtle systems of the dogmaticians, this would yet give no assurance whatsoever that there was any connection at all between such a church fellowship and the Church’s only Head, Christ. Pure doctrine is not better able than pure polity, or pure church discipline, or the infallible Pope, how to be the unity of Christ’s Body and to present a church fellowship as the only true one. Everywhere the same thing is lacking: there is no assurance that what is put forth as the bond of union between the Church and the ascended Christ can truly bring such a union to pass.

The visible church fellowship and the invisible Christ always remain disconnected as soon as the bond is to be something visible. For a visible bond of union may indeed unite the visible church fellowship into an outward unity; but it can never unite the visible with the invisible. And every such visible means is indeed well suited to give a church fellowship lofty thoughts of itself and to lead it to delight in its own greatness; but it is wholly unsuited to bring Christ the honor that is His due.

Thus we are directed back to the Word of God when we would seek the true unity of the Church, the unity of the Body of Christ. And we have this confidence, that we shall not seek in vain; for just as the Word of God has shown itself mighty to overthrow imaginations and every height that exalteth itself against the knowledge of God, so is it also mighty to guide us into all truth. The Word of God is not appointed only to uproot and to tear down, to destroy and to overthrow, but also to build and to plant. It has been necessary to reject the misleading paths by which human contrivance has sought to lead the Church of God, in order that there might be room for a serious examination of the uncorrupted truth of God’s Word.

The Body of Christ is one, and it has but one Head, Christ. Yet then the unity of the Church is a unity of a body, and therefore can only be unity in life. This is that which is the binding bond of the Body of Christ and its unifying principle: that the life of Christ flows through it and permeates all the individual members. Where the life of Christ is found—be it among Jews or Greeks, within this church body or that—there the Body of Christ is found. In this way every body is a precisely coherent unity, and in precisely the same way the Body of Christ also forms a perfect unity; and in this way, and in this way alone, the Church is one. Everywhere that it is found, there is the same life; and when it is asked what distinguishes this Body from all other bodies, then it is not that there is some other bond of unity than that of life, but this: that the life which here pulses in every single member is the life of Christ.

And if we would go further in the investigation of the unity of the Church, this can only take place through a deeper penetration into its peculiar life and into the necessary elements of that life. And we can immediately recognize that if it is one and the same life which thus unites the Church of Christ with itself, then it is certain that this life, like every other, takes its beginning with a birth; for only birth brings forth life, and if the life is to be the same, then the birth must also be the same. And this is something concerning which indeed all Christians are agreed: that there is but one birth whereby human beings, born of flesh, can become partakers of the life of Christ, namely the birth of water and Spirit, Baptism, with its one Spirit unto one Body.

Thus it becomes for us, who have recognized that the unity of the Body of Christ is unity in life, an indisputable truth that the unity of the Church is first and foremost a unity in Baptism with one Spirit. Here stands the first mark of the Church of Christ, and nothing else can be placed before this; for the unity of life cannot be prior to life itself, and life cannot be prior to birth.

But we can proceed further in the determination of the unity of the Church; for the life which is thus born has its own peculiar essence, which is found again everywhere. Just as the same blood flows through every vein and is found in every member of the human body, so also the life of Christ in sinful human beings everywhere has the same fundamental nature: it is faith, which is the peculiar essence of this life—the liberating knowledge of the truth and the life-giving power of love. This is that which constitutes the second element in the unity of the Church’s life: unity in one faith.

Thus we have gained a new determination of wherein the unity of the Church consists. We see that it is the one birth of the life of Christ and its one essence which become the true unity of life, that which forms fellowship in the Church. But just as the life of Christ has one source, just as the entrance into the Kingdom of Heaven is one—the birth of water and Spirit in Baptism—and just as the life of Christ has a common essence and the children of the Kingdom have a common citizenship, faith; so also the life of Christ everywhere awaits one consummation where it is found, and the Kingdom of Heaven advances toward the goal of glory, in which all its children, whether from East or West, from near or far, shall become partakers.

The Church has one future, and the Body of Christ awaits one full and eternal union with its Head; and therefore all who have the life of Christ in this world are united by one and the same hope. Thus we come to understand why the Word of God speaks of the unity of the Church not as a common subjection under a pope, not as a common association around a constitution, not as a common agreement concerning pure doctrine, but as unity in Baptism with one Spirit, unity in faith in one Savior, in hope of one glory—and we gather it all together when we say: unity in the life of Christ.

Where the life of Christ is found, there is His Body, and nowhere else. The Head is not Head for any other body than that which is permeated by its own life. Thus it will no longer surprise us that we have denied—and must deny—that any single, definite, outward church body alone can lay claim to being the true Body of Christ and His rightful Bride; for no one can prove, or even assert, that only within a single church body are there such as partake of the life of Christ. And still less can anyone point to a single church body in which all members without exception are filled with the Spirit of Christ and partake of His life.

It has indeed been claimed by the Roman Catholic Church to bear witness concerning itself that it is the one saving Church, and even more, that he who is a member of this Church is assured of having part in Christ; but this claim has already been judged, and it would scarcely be necessary once again to repeat the testimony of history. And every Lutheran Christian knows what he ought to think of anyone who dares to give himself a similar testimony.

We have recognized that the only true mark of the Body of Christ is the life of Christ; and if anyone is so endowed that he can discern where the life of Christ is found, he will also be able to point out the Body of Christ. But everyone will surely be compelled to acknowledge that the life of Christ can be found everywhere—in any church body, or apart from all ordered church bodies—where the Gospel of Christ in Word and Sacrament performs its work.

Accordingly, despite all division, there is one Church, one Body, because there is the unity of the Spirit among all those who are partakers of the one Life. But because the unity of the Church is that of Life and of the Spirit, therefore it is not bound to any single society of those who may be called by the name of Christ. For the Spirit bloweth where it listeth, and Life springeth forth wherever the good Seed is received in believing hearts. And precisely for this reason the unity of the Church is not a visible thing like an outward ecclesiastical society, which anyone may lay hold of and feel; but it is an invisible glory, which only the eye of faith beholdeth. Therefore we also confess that we believe that the Church is one; but faith is a sure confidence of that which is not seen and not understood.

They who believe know the Christ is not divided, and that His Body is not divided; and by virtue of the Life that is in them, they are able to keep the unity of the Spirit in the bond of peace, despite all strife between society and society. Whithersoever they perceive the Life of Christ in Spirit and in truth, there they recognize a brother or a sister; and in their unanimous confession of the great works of God toward them—how He saved their soul from darkness and their life from death—all else is forgotten, and all discord is overcome in a blessed assurance of a unity mightier than all division: unity with the Lord, who is Head over all things.

But while in this manner the unity of the Church, which is Christ’s own life, is invisible, and while with all the Fathers we confess our faith that the Church is one, our daily experience nevertheless shows that there are many churches; and thus it might seem as though our faith were only a comforting imagination, a fabric of dreams which all reality demands that we abandon. But this is a far too hasty conclusion. The differences among the many ecclesial communions, the motley throng of churches that lies before our eyes, can in no way make void the unity of the Church or of the Body of Christ. Just as little as it is given into the power of men to divide Christ, just so little does it stand in the power of men to abolish the unity of His Body. Men may form many churches, but the Body remains one.

And in truth it is so far from being the case that the many churchly communions, if they understand their position in earnestness, dissolve the unity of the Body and assert the falsity of the Body, that they much rather, in the Lord’s governance of His redeemed people, serve only to make the Body stronger. Where it stands unshakably firm that the Body is one only, and in spirit and in truth one, permeated by the one and the same life, there cleaves to it no fear and no doubt if this Body unfolds itself in many members, and each member is governed by the same life and unfolds itself unto greater and greater strength. The strength which the individual communion unfolds is not to the Body’s harm; rather, the growth of the Body will then only become vigorous when each communion is permitted freely to unfold its indwelling power.

The division of the Church into diverse ecclesial communions is grounded in a diversity of task, a diversity of calling, which is given by the Lord. And the Lord has set the many diverse tasks in place, in order that the Body may grow unto ever greater maturity. It is indeed so that the Body of Christ in its totality has a common task, namely this: always to press deeper and deeper into the knowledge of Christ, and thereby to be made fit more and more to become like Him; but because the knowledge of Christ is infinitely rich and an inexhaustible fullness, which only eternity can give fully and completely, therefore it is given to us on earth to know in part, and no one can boast himself as one who has known all.

Therefore it can never be otherwise than that the one grasps one part of the knowledge of Christ, the other another part; and when each, with all strength and all earnestness, immerses himself to that which has been given him, the growth of the Body by no means suffers thereby, but rather prospers all the more. For the sum of this is that the knowledge of Christ increases and is enlarged, and the life of Christ is bound together in power. Thus it is no harm to the Body of Christ that the riches of the knowledge of Christ through the labor of the many communions come ever more clearly to light, but it is a gain when this great labor is divided, because thereby more labor is accomplished.

But that which harms the Body of Christ and breaks down the Church of God on earth is this: that the one communion will not acknowledge the legitimacy of the other communion’s labor. Then God’s work is disturbed, insofar as it is given into the power of men to disturb it; for when the one says to the other, Thou hast no right to exist, then strife and discord begin, then the unity of the Spirit is disrupted, because the bond of peace is broken.

But if it may be regarded as acknowledged, that we all know in part, and that for the one a branch of the knowledge of Christ must appear as the primary concern, while the other sees nothing else than that what he has known is the chief matter, then it is surely just as certain that the one receives his share of the work, and the other his, because it is a divine order that the one is most fitted for this work, the other for that. The work is divided because it is infinitely great; but each receives his share of the work because he is the one who is most apt for this task. Just as upon the craftsman's workbench each worker has his particular assignment, because each is proficient in a peculiar direction, so also will it be within the Church, that every community has its specially appointed task, because it possesses a particular disposition in an individual direction. And we therefore arrive at the result, that the division of the Church into various church-communities has a twofold justification, which for us may be presented as on the one side a divine, on the other side a human, whereby in another sense we may indeed acknowledge that all here is divine. The one side is this, that the Church is the body of Christ and can grow only through the development of each member’s own peculiar power. It is necessary for a body that it be not exclusively eye, but likewise ear; it is necessary not only to have a sure hand, but also a firm foot, and thus there lies in the very nature of the body a necessity that it have many members. And this is the side which we call the divine; for it is an order set by God, that the body is only then in its right growth when each member develops freely. But on the other side the Church is in the world; it consists of a multitude of men with different dispositions and different gifts, and each of these shall confess and bring it back with increase; and therefore there is a human justification for the division of the Church into several communities, each with its own task, in order that the gifts of the one may come to their richest application no less than those of the other.

One will naturally object here that this line of thought either does not go far enough, or that it goes far too far. On the one hand, it will be said that the various tasks which correspond to the peculiar dispositions can very well find room within the same church community, and indeed must find room there, since every church community consists of a multitude of individuals, each with his different dispositions; and thus the different tasks do not require different communities. On the other hand, it will be said that if it is true that every task and every new disposition requires a new church community, then one cannot stop until one has arrived at the conclusion that every human being stands alone by himself. Only then would the demand of the proposition cease, that the division of the Church is grounded in difference of task and disposition.

But neither of these assertions is a necessary consequence of our principle; or rather, both of these inferences will of necessity be refuted by the Church’s own development. For as regards the first, that a single church community can contain within itself all different tasks and satisfy all dispositions, this is true to a certain degree, and precisely so much nearer to the truth as this single church community approaches the true essence of Christ’s Church. Yet in reality, every church community, however all-embracing it may appear in its beginning, will unavoidably, in the course of its further development, be drawn more and more into one-sidedness. And the longer this endures, the more intolerable the narrowing of the task will become, and the less room there will be for labor on behalf of the other tasks. And development will, according to all that history has yet disclosed, end in a breach within the church community, and one has precisely that division which one thought might be avoided.

Our principle, which requires that the different tasks may ground different communities, is at once grounded in the necessity that all sides of the Christian life be developed to the greatest possible clarity, and in the impossibility that this can take place within a single community—an impossibility which history has demonstrated not once, but many times—an impossibility which indeed would not exist, were it not that sin also plays a role in the Church’s development toward one-sidedness.

And as to the other side of the assertion, that our principle carries too far, until it dissolves the Church into individuals, inasmuch as each human being must be a church community for himself, since he has his own peculiar task and his own peculiar dispositions, it is quite true that every Christian individually is a member of Christ’s body, and according to his peculiar dispositions has his peculiar task. But if one wishes to draw the conclusion: therefore each is a church community for himself, one has leapt over the fact that the essence of the Church is, of necessity, community, and that it is therefore a false inference that because the Church divides itself into communities according to task and disposition, it must therefore divide itself into individuals. One then forgets that an ecclesial task, in order to be carried out, precisely requires a community; and if a task shows itself not to be community-forming, then it thereby shows itself not to be any ecclesial task at all.

Our principle leads with necessity to the division of the Church into ecclesial communities, but it cannot lead further; for the work that is to be done is communal work and can in no wise be performed by individuals alone.

We have seen wherein the unity of the Church consists, and we have seen what it is that calls forth the many ecclesiastical communions; yet our task is not thereby concluded. For here there arises yet another question of the greatest importance, which in truth is the most burning one of our own time. If the very growth of the Church and its development require a division into several church communions, then it is of the utmost importance to know whether every division of the Church is thus for the Church’s good, and whether every church communion, without further consideration, has the right to exist.

We shall begin with the latter question, and upon the foundation which we have now sought to lay, attempt an answer to that question upon which a church communion’s right to exist depends. From ancient times it has been an assertion of the Catholic Church against the Protestant communions, that they had no right to exist, because they were founded upon a breach with the Catholic Church; and it is again a very common assertion in our own days, from one communion against another, that it has no right to exist. There is renewed and earnest talk of breaches of one communion from another, and there is an outcry over legalistic rights, because development brings it with it that new church communions arise.

For this reason we believe that a calm discussion of this question, independent of party inclinations, is in its proper place and comes at its proper time. And if we have not altogether erred in our preceding exposition, it will be seen that we must arrive at the conclusion that a church communion’s right to exist independently rests upon its own vital power. “If this work be of God,” says Gamaliel, “ye cannot overthrow it.”

But we have seen that a church communion only then bears this name with justice, and only then can possess vital power, when it is a member of the Body of Christ. Participation in Christ, unity with Him, is the first condition and the fundamental condition for the existence of every church communion. Yet if a communion is to have the right to exist alongside others, if it is to have the right to be an independent member of the Body, then it must have its own peculiar task, its distinctive ministry. It is precisely this that will come into question; and every communion which is unable to demonstrate its particular task within ecclesiastical development lacks the fundamental condition for being independent.

But where a church communion can show that it has a work to do which hitherto was neglected, where it can show that it has an aspect of the knowledge of Christ to set forth and a branch of the life of Christ to unfold which finds no place within the already existing communions, there it has its right to exist; and there the already existing communions have the duty to acknowledge that right, if they would not be found among those who strive against God.

Only in this way can the Protestant communions defend their right to exist alongside the Catholic Church, and only in this way can, in our own time, every church communion assert its right as independent. And only under this presupposition can it be made a duty for one communion to acknowledge another alongside itself. And only when the many communions respect one another’s work and mutually acknowledge one another as those who are set by God to labor in one Spirit forward toward one goal, which stands above them all and is greater than them all—only then will the Church in its totality, the Body of Christ, go forward with power and accomplish its work with life.

But where a single church communion seats itself upon the judgment seat, and where it claims for itself an exclusive right to be Christ’s Church and an exclusive right to exist; where, in mad arrogance, it seizes to itself the honor of alone being the Body of Christ, and with blasphemy against God’s saving power calls itself alone saving and delivers the remaining communions over to eternal perdition, and hurls the ban against those whom God has marked with the mark of Christ and made partakers of His own life—there such a church communion has stamped itself with the sign of Antichrist, he who is God’s enemy and yet sits upon the throne in God’s congregation.

It is not the division of the Church into many communions that undermines God’s work among us, but factionalism and the striving after one’s own honor and one’s own power; these are what bring the Church to humiliation, God’s honor to diminishment, and stagnate God’s work.

But now new questions arise, which seem altogether to demand the completion of all that we have previously recognized. If it be so that no single church body alone has the right to call itself the Body of Christ; if there are many church bodies that may lay claim to being Christian—are we then to understand this in such a way that all communions are equally good, that it becomes wholly a matter of indifference to which communion I belong, and that all the strife which is carried on between church and church is sheer absurdity?

Is it sheer folly when the Lutheran Church contends against the Catholic, and the Reformed Church contends against the Lutheran? Are we henceforth to be entirely at ease, to let each church body labor for itself, and to make no effort to uphold our confession and to set forth what advantages we possess over other communions?

We answer with a definite No to all these questions. There is nothing in our earlier development that affords them any support; and we shall show that indifference has no right within God’s Church, and that the struggle between the contending communions is just as necessary for the Church of God’s flourishing and for its future course along the path of the knowledge of truth, as it is necessary that one communion, so far as the truth allows it, acknowledge the other’s right to exist.

For if it is indeed the case that the division of the Church rests upon a diversity of tasks; if each communion lays hold of a side of the knowledge of Christ, which is nothing other than a distinctive side of that fullness of the knowledge of Christ which is laid up in Holy Scripture; and if each communion has the work of setting forth a branch of the life of Christ—then it is already certain in advance that among the various communions there will be a high degree of difference, according as it is given them to draw near to what is central in the knowledge of Christ.

History will also in the clearest manner establish this for us; and later we shall let it speak. There will be a clear and definite difference between the communions, according as, in the exceeding riches of Holy Scripture, they are able to appropriate that which is the all-governing and all-penetrating fundamental truth of the Word of God, or whether they lay hold only of a side of this fundamental truth, which stands in nearer or more distant connection with it. One communion may penetrate more deeply into the truth than another; and it is precisely this which makes it a contempt for all seriousness and a mockery of the truth to suppose that all communions are equally good, and that it is a matter of indifference whether I belong to the one or to the other.

It is true that every communion with the Word of Christ and the Sacraments, and with its distinctive task to contend for, has the right to exist; but it is a manifest error if one would from this conclude: therefore the one is just as good as the other. And there are cases in which it becomes an open denial of the truth to leave a communion that has a deeper knowledge of Christ’s saving truth, in order to give oneself over to another that in this respect is less advanced.

Yet so long as a Christian, in the sincerity of his heart and fully persuaded in his own mind, stands within a church communion—be it a much or a little erring communion—so long has no one the right to deny him participation in Christ’s salvation and in eternal blessedness, however severe the judgment may sound upon the communion to which he belongs. By this we have not leveled the difference that exists between the communions; we have only made it possible for that difference to be acknowledged in its proper worth. For the difference is not this, that one church communion alone is Christ’s Church and the other is wholly outside the Body of Christ; but the difference is this, that the one may have laid hold more deeply of the knowledge of Christ than the other.

And it is this understanding of the relation that alone makes the struggle possible, and makes it not only justified, but altogether necessary. It is not a war of extermination that is to be waged between community and community within the Church, but a struggle to uphold, each on its own side, a portion of the Truth. 

And here we take up a new moment, which earlier we could to a certain degree leave out of consideration: sin within the Church. When an ecclesial community has grasped a single side of the truth of Christ, it soon becomes an impossibility for it to acknowledge that it has not grasped the whole. The opposition to the other communities and the zeal for its own cause drive it to an overvaluation of its own task; and, driven ever further and further from a good beginning out into the one-sidedness that already lay there from the point of departure, it is narrowed more and more, and it cannot be otherwise than that the one-sidedness reveals itself ever more clearly, until the distortion that lay in the beginning becomes more prominent than the truth that lay therein. And if no particular development leads the individual community onto a new path, then there is a high degree of danger that the course which began well will end in grievous delusions and in the manifestation of an antichristian direction. 

But it is under this development, to which any community whatsoever is exposed, that the struggle comes into its fullest right. It is in this course that the one community asserts its apprehension of the truth over against the other. The one-sidedness with which each holds fast to its own—and must hold fast to its own—brings each side of the truth forth in its sharpest light, and the keenest weapons are wielded by practiced hands; and the Truth suffers not, for its light shines only so much the more clearly, the more boldly its various moments are set forth; and Love suffers not, for the struggle is, in the all-guiding hand of the Lord, the means by which the work of Antichrist is checked and hindered wherever love for the Truth is still greater than party-spirit and the desire for one’s own honor, and it is the means by which the work of Antichrist is made manifest even to the simplest, there where love for the Truth has already become weaker than the craving for power. 

Thus it is no ungodliness when a struggle is waged upon the ground of the Truth; it is a necessary means for the Church’s growth in the knowledge of the Truth, and we should neither wish nor even pray that the struggle, so long as it is waged in this manner, might cease, while the Church is still upon the earth and knowledge is in part. If the struggle ceases, it is a sign that the seriousness of truth-knowledge is diminishing; but if the struggle degenerates into bitterness and suffocation, into a war of extermination of community against community, then it is a sign that seriousness has ebbed away into party inclination and carnal desire. But midway between these two extremes lies love’s narrow way, which so few find: on the one hand, to esteem the Truth higher than all and not to surrender even a hair’s breadth of the truth that has been recognized, even if, in appearance, the whole world might thereby be gained; and on the other hand, not to hold fast even a hair’s breadth of a recognized error, though all one’s own honor should be lost by letting the error go.

Just as the division of the Church into many communities does not abolish its unity, which is the unity of life with the Lord, so the recognition of the right of the individual communities to exist does not abolish the superiority of the one community over the other, and the common ground of truth does not abolish the mutual struggle.\footnote{What follows the author himself designated as “historical illumination.” — Ed.}
