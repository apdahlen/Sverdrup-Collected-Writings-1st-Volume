
\begin{center}
\includegraphics[width=0.9\textwidth]{OpenImage3.png}
\end{center}

\bigskip


\subsection{The Prophet Jeremiah}

Source: Quarterly Writings. Third Year. 1877. Pages 113–137. — Ed. 1.

This section appears on pages 36–57 of the original volume. — Present Ed.

\bigskip

The prophets of the Old Covenant are the great and glorious privilege of the people of Israel, grounded in this, that Israel was the people of God. For it is the privilege of God’s people that the Word of God is entrusted to them; but the distinctive mark of the prophets, that which makes them to be called prophets, is precisely this, that the Word of the Lord came unto them, and that they proclaimed it to the people. A concept may indeed be formed of what a prophet is, yet none is simpler than this: “A Prophet will I raise up unto them from among their brethren, like unto thee; and I will put my words in his mouth, and he shall speak unto them all that I shall command him” (Deut. 18:18).

God’s wondrous counsel of salvation with His people and with the world—that is God’s thought; and because the thought of the Lord is higher than the thought of men, therefore it must be revealed in the Word of the Lord. There may be many deep thoughts of men and many lofty human plans, but that which the natural man does not comprehend is this, that the way of life passes through death, both for the individual and for the people. It is the Lord alone who has conceived it, it is the Lord alone who has understood it, and it is the Lord alone who has wrought the salvation of His people by smiting sin and scattering the sheep, by crushing unto death the Prince of Life.

—It is this way of salvation which the prophets are to proclaim to the people. The people believe it not, and their hardening becomes one of the means in the hand of the Lord whereby His plan is fulfilled. —The people believe it, and the faith of the people becomes another means for the advancement of the Lord’s will; for

\begin{quote}
What counsel He chooses to perform,\\
It stands as firm as mountains stand,\\
And this He surely shall perform,\\
Though earth and heaven shuuld burst.\\
\end{quote}

God’s counsel of salvation is accomplished in Israel, whose entire veiling points toward one single goal, Jesus Christ, in whom the counsel of salvation is brought to completion in the fullness of time. Yet as the people of Israel drew nearer to the fullness of time, it was prepared through the Lord’s manifold acts of deliverance, so that it might rightly receive the Savior—or reject Him and give Him the death of a criminal—just as love had resolved, and as love alone could bring to pass. He is the true Israel; He is the Seed, Abraham’s Seed—not many, but one: Christ.

And as the people advanced ever nearer to the fullness of time, so the word of the Lord followed them by the mouth of the prophets, explaining to the people how far they had come, and what the next step would be in the preparation, so that those who did not believe might fill up the measure of their hardening, and those who believed might abide in patience and attain the goal of their faith. Thus the prophetic word is the light and the truth upon the people’s path, pointing toward the goal, while at the same time powerfully and effectually leading the people forward toward that goal.

Every people has its calling in God’s economy. A people may despise its calling, and it perishes; yet even through its perdition it must serve the counsel of the Lord. A people may honor its calling and walk worthy of it; then it is saved, and in being saved it serves the counsel of the Lord willingly. But since both the one who perishes and the one who is saved must serve the Lord, it is no merit for you that you fulfill the counsel of the Lord with all your strength. You have nothing of which to boast; rather you must thank the Lord for everything, for you are only an unprofitable servant, because you did what you were bound to do.

The people of Israel also had its calling. It was to be a holy people, a chosen people, a people for possession, in whose midst the Lord would dwell, that He might be their God, and they His people. But in this Israel stood under the same two dangers that confront everyone whom God has called. The one is to be unfaithful in one’s calling, not to esteem it high and holy as coming from the Holy One; the other is to boast of one’s calling, or to misunderstand it, and from it seek to make oneself a merit. Against these two dangers the prophets must testify and contend.

On high the prophet lifts the people’s calling and its goal. In sharp, mighty words he speaks to the aimless, wavering, halting people of the Lord’s great calling which He gave them, of the great and glorious future which He prepared for them. And with flashing words, full of divine wrath, he strikes down the people’s proud delusion that it already is the Lord’s people merely by right of birth. With burning zeal for the truth he repeats again and again the warning that the people must not forfeit its glorious future through the miserable delusion that it has already reached the goal and may now begin to enjoy its peace. And because it is always so that many are called but few are chosen, and yet the zealous priests would draw all to themselves, therefore the prophets are full of the hidden ways of God and of afflictions, and of sharp threats and judgments; for as high as the Lord’s mercy is, so high is His righteousness, and if sinners will not immerse themselves in the one, they shall surely be overthrown and crushed by the other.

There are, then, three things in which the work of the prophets is gathered together.
It is the revelation of God’s counsel of salvation, the final, eternal deliverance from the bondage of sin. Next, it is the preparation of the people to receive the coming Savior. Finally, it is the early and unceasing task of reminding the people of their calling, and again and again to warn: “Take heed, return!”—to all apostate and stubborn children.

“And we have the prophetic word made more sure, whereunto ye do well that ye take heed, as unto a light that shineth in a dark place, until the day dawn, and the Day-star arise in your hearts” (2 Peter 1:19). Thus the Apostle Peter exhorts us. And the Christian congregation ought indeed far more diligently to follow this exhortation, and to use the prophetic word for its edification. For it is the promise of Abraham which has now come, and daily comes, to all peoples; it is the hope of Israel which is now being fulfilled from day to day. There is still a coming of Christ in the hearts, for which they are prepared by the Word of the Lord; it is still necessary to cry out: “Turn ye, my fallow ground, and go down no longer into the thorns”; and there is still a holy calling which goes forth to all lost sons and daughters in the whole of humanity: “Turn ye unto me, and I will turn unto you!”

But the prophetic word is greatly neglected among us, and thereby we have lost much power and boldness; thereby we have forgotten the great and holy goal which God’s Word has set before us: “Ye shall be my people, and I will be your God.” We have become apathetic and sluggish to spread God’s Word, because we have forgotten the calling—that the Lord wills that through us the Gospel of the Kingdom should be proclaimed in all lands and in all tongues. We have forgotten that “in the last days the mountain of the Lord’s house shall be established in the top of the mountains, and shall be exalted above the hills; and all nations shall flow unto it.”

If it might be possible to spur one or another to more diligent use of this blessed Word, and thereby to gain comfort and boldness in these evil days, when the congregation seems destined to perish, because the chief priests and Herod and Pilate have again joined together against it; if it might be possible to strengthen the congregation in its despair unto a final victory over all its enemies; if it might be possible to cry “Repent!” into a rotting Christendom—then we will attempt to give a small glimpse into the life of the prophet Jeremiah and his book. And we believe that if anyone should allow himself to be moved to read and ponder this prophet of sorrow and affliction, he would experience that where the Word of the Lord crushes and smites, there it also raises up and heals.

\subsubsection{Israel’s People in the Time of Jeremiah}

The fair vine which the Lord had planted in the land that flowed with milk and honey was at this time already grievously laid waste. In the victorious and peaceful days of David and Solomon it had struck its roots deep and spread itself wide; it had sent forth its branches unto the sea and its shoots unto the river. But now its walls were broken down; all that passed by plucked at it. Its crown and trunk were felled to the dust, and one part was hewn off and cast far away beyond the Euphrates. The kingdom of Israel was no more; and the kingdom of Judah, which still remained, was like the vine that beareth no grapes: there is no joy in it and there is no profit in it; it yieldeth no wine, and its wood is good for nothing. The kingdom of Judah stood straining and waiting for the axe which had cut down the kingdom of Israel to be lifted a second time and to fall dripping upon the golden tendril.

The kingdom of Israel had never had any better future in store for it from the day when it said, “What portion have we in David? we have no inheritance in the son of Jesse. To your tents, O Israel! now see to thine own house, David.” By this it had entered upon the path of decay; it had reckoned itself among the heathen nations and had thus voluntarily chosen the portion of the heathen, which is death by the power of the kingdoms of this world. The kingdom of Judah, on the other hand, had better ordinances and better hope; for the house of David had the promise of abiding for ever, and the temple of the Lord had the promise that it should become a house of prayer for all peoples. And the word of the Lord dwelt in Judah, and was God’s power and cleaving-wedge for the little afflicted people who from the valleys of Judah went up to Jerusalem, the lofty-built city upon the mountains, round about which the Lord was wall and rampart, the wondrous city in which the nations should be born anew.

But in Judah everything depended upon this one thing—upon faith. “If ye will not believe, surely ye shall not be established.” Human malice could not make void the faithfulness of God; but if Judah would not believe, would not drink of the fountain that giveth strength unto everlasting life, then must the crushing judgment pass over the people of hardening. And thus it came to pass. The kingdom of Judah set its neck and a hardened heart against the word of the Lord; therefore the word of the Lord became a hammer that breaketh the rock in pieces. Hardening and destruction follow one another as death and corruption.

There is a place in the word of God which sets before us this inevitable connection between hardening and corruption; it is the sixth chapter of the prophet Isaiah. In the year of King Uzziah’s death the prophet receiveth this command: “Go, and tell this people, Hear ye indeed, but understand not; and see ye indeed, but perceive not. Make the heart of this people fat, and make their ears heavy, and shut their eyes; lest they see with their eyes, and hear with their ears, and understand with their heart, and convert, and be healed.” Then said the prophet, “Lord, how long?” And he answered, “Until the cities be wasted without inhabitant, and the houses without man, and the land be utterly desolate; and the Lord have removed men far away, and there be a great forsaking in the midst of the land.” From this hour onward, the growing hardening and the judgment that draweth ever nearer and nearer are the history of Israel. The growing hardening is this, that the word of the Lord soundeth early and earnestly and is not heeded; the advancing judgment is this, that the birds of prey of the world-kingdoms gather more and more closely over the stiffened carcass of the kingdom of Judah. The Lord suffered both to come to pass. He sent His prophets early and earnestly unto a rebellious people; and as ungodliness increased, He raised up the mighty kingdoms of the world, the rod of His wrath over the nations, and above all over the chosen people.

The prophet Isaiah lived after the death of Uzziah, under three kings—Jotham, Ahaz, and Hezekiah—and in this period two wondrous events occurred which, by righteous steps, led the people of Israel toward their appointed end.

The first is this: that Ahaz would not believe the word of the Lord nor the help of the Lord against the kingdom of Israel and Syria, who had attacked him (Isa. 7); but instead he entered into alliance with the Assyrian world-empire, which came and destroyed the kingdom of Israel and overflowed Judah, until it reached even unto the neck. The second is this: that Hezekiah believed the help and the word of the Lord against the Assyrians and was delivered from death; yet soon thereafter he drew so near to Babylon in friendship that the Lord proclaimed that Babylon should become the rod of chastisement over Judah.

After Hezekiah followed his son Manasseh, who reigned over Judah for fifty-five years and, during this long reign, brought Judah’s sin to its full measure, so that the people, thoroughly permeated by idolatry favored from above, became utterly corrupted and could no longer be cleansed except through judgment. Yet the mercy of the Lord was not exhausted; for though the people of Judah could not be saved, the remnant of Judah could yet be plucked as a burning brand out of the fire.

Therefore, even after the death of Manasseh, counseling prophets still went forth to the people; and after the ungodly reign of Amon there came yet a revival under King Josiah.

He was eight years old when he became king, and when he was sixteen years old he began to seek the Lord. He turned himself to God, and since his heart had been made steadfast in the fear of God, he began, at twenty years of age, a reformation in his kingdom—a cleansing from idolatry. He tore down all the images of idols, overthrew their altars, and forbade their sacrifices throughout the realm.

The following year Jeremiah was called to be a prophet, and his mighty revival preaching supported the king’s reformation. Six years passed, and then another remarkable event strengthened still further the cry, “Awake!” to the sleeping people: the Book of the Law was found by the priest Hilkiah. It was brought to the king and read before him, and it filled him with the deepest sorrow, because the people were so far from the goal of the Law. Then he assembled all the people and made a covenant with them, that from that time forth they should keep the Law of the Lord.

And Jeremiah went about in Jerusalem and in all the cities of Judah and proclaimed, by the command of the Lord: “Hear the words of this covenant, and do them” (Jer. 11:6). And thereafter Josiah kept the Passover with the whole people—a Passover such as had never been kept since the days of the Judges.

It was a wondrous season of stirring in the kingdom of Judah in the days of Josiah, and the Lord called to His people with a entreating voice: “Turn ye, turn ye from your evil way!” But the people would not. They said, “It is in vain; for we will walk after our own devices, and we will every one do the stubbornness of his evil heart.”

And whereas in the days of Josiah all things seemed to call the people to repentance and to lift them up unto the LORD, so after his death it was as though everything joined together to drive the people inexorably forward toward the goal of their hardening and their corruption. There arose ungodly kings, self-righteous priests, and false prophets. And while the kings turned themselves away from the LORD and sought help from the powers of the world, the priests cried out, “The temple of the LORD,” and the prophets cried, “Peace, peace, and no danger,” until the poor people, who had been sold over to destruction, slept and slumbered sweetly upon the heavy quilts of security and carelessness, so that they slept away their right and their calling. They shut their eyes to the danger, until suddenly destruction stood over them, as travail upon the woman with child.

Thus matters stood inwardly. Priests, prophets, and kings hollowed out the marrow of the people, until it was like dead flesh without heart, without courage, without faith—a helpless mass, whose secret allegiance most often inlined  toward its enemies. And those—yes, those—to whom Judah in haste and in helplessness had bound itself, it had fastened itself to them all. Judah had sought Egypt for help against Assyria, Assyria for help against Egypt, Babylon for support against them both. Now they came upon Judah, whom they all despised. Assyria had already, under Hezekiah and Manasseh, flooded the land of Judah, because it despised the softly flowing waters of Hezekiah (Isa. 8:6–8). Egypt became the instrument that brought death upon the last king of David’s faith, Josiah, in the crushing battle at Megiddo; and Babylon brought the judgment of destruction and of exile to its deepest depth: the holy city became a heap of rubble, while the temple went up in flames.

Meanwhile, in the far East, the foundation was being laid for the kingdom that was to break Israel’s bonds and cast down proud Babylon. The sword-bearing, warlike Persians prepared themselves for conquest against the southern builders of Babylon, enfeebled by debauchery and luxury.

\subsubsection{The Prophet's Life}

Jeremiah was born in the small village of Anathoth, which lies about four English miles northeast of Jerusalem, in the tribe of Benjamin. Being of priestly birth, he had from his youth onward abundant opportunity to hear the Word of God and, at least according to the letter, to learn to know God and His deeds in Israel, and to become familiar with the service of the Temple. He likewise had the best opportunity to observe both how the Temple of the LORD and His priests were despised and mocked under Manasseh, and how the priests, when Josiah began his reformation, sought to draw to themselves power and advantage from Josiah’s work, as a kind of compensation for what the idolatry of Manasseh had taken from them. Otherwise we know nothing of his life before he was called to be a prophet. This altogether decisive event in his life, which made him a man of affliction and suffering from the time when his countrymen in Anathoth sought to slay him (Jer. 11:21) until he sat weeping among the ruins of Jerusalem—this heavy calling of the LORD came to him in the thirteenth year of Josiah. The reformation which had begun in the twelfth year of Josiah, when the king himself had been awakened, had indeed also been a preparation for Jeremiah for his calling. From this time forth he was to become the true awakening prophet in Judah. For hymns and laws might well be able to foster outward conformity with the Law; but the Word of God alone could pierce the hearts.

And while the king had the comparatively easy task of breaking down idols and restoring the Temple service in its purity, Jeremiah received the arduous and painful task of bearing witness before the people, the priests, and the princes that, unless their hearts were broken in pieces by the Word of God, the Temple was to them nothing but an idol, and self-righteousness nothing but a worse apostasy than any other. And all the heavier did the prophet’s labor become, because he spoke to deaf ears and daily saw a beloved people draw near to their destruction by trusting, in the midst of their ungodliness, in the great outward change which had taken place through Josiah’s reformation.

There are two things in the calling of the prophet Jeremiah to which we draw attention. The first is that he is appointed as a prophet to the nations (Jer. 1:5). Nothing is more surprising than this. For to the Jews the Word of God had indeed been given; to them, according to God’s promise, the prophets were to be sent; and yet here Jeremiah is set as a prophet to the nations, a designation which includes both Jews and Gentiles without distinction.

There is only one other Prophet in the Old Testament who was not sent to the Jews, and his sending also is exceedingly wondrous; yet it serves to cast light upon the calling of Jeremiah. This is the Prophet Jonah, who is sent to Nineveh, and who, however unwillingly he goes, must nevertheless at last come thither.

Jonah was sent to the great city, which was the Lord’s instrument of punishment over the kingdom of Israel, with this brief preaching: “Yet forty days, and Nineveh shall be overthrown” (Jonah 3:4). But Nineveh repented in sackcloth and ashes, and God had compassion upon the world-city and allowed it to endure yet a little time. This was the first great sign which the Lord set before His people, whereby He bore witness to them that there is no difference between Jew and Gentile; for all have sinned. Therefore the one stands just as much under God’s judgment as the other. But when the Ninevites repented, then God set His sign of mercy and pointed toward the mystery which is revealed in Christ: that God hath shut up all under sin, that He might have mercy upon all. And there is no difference between Jew and Greek; but apart from the Law the righteousness of God is revealed through faith in Jesus Christ, unto all and upon all that believe.

But the more the Jewish people, by their sin and stubbornness, placed themselves on a level with the Gentiles, the more decisively did the Lord testify to them that they who were equal in sin should become equal in punishment. And all the more clearly did He also make it known among men that they who had become equal in brokenness under God’s judgment, they should become equal in restoration by God’s grace, if they called upon the Lord in their day of distress. Therefore Jeremiah, who in this manner must walk sorrowing behind the funeral-bier of his people, is already in his very calling appointed as prophet over the nations and over the kingdoms (Jer. 1:10).

In the closest connection with the fact that Jeremiah is thus made the prophet of the nations and of the kingdoms stands the other remarkable feature of Jeremiah’s calling. For he is set over the nations and over the kingdoms, to root out and to pull down and to destroy and to overthrow, to build and to plant. It is first and foremost four strong and mighty words of demolition that designate the prophet’s work, and thereafter two words directed toward healing, which express his labor of edification. For the reason that the nations and the kingdoms (among which the people of Israel are also included) are placed under Jeremiah’s prophetic hand lies herein, that they have now filled up the measure of their sins, and therefore must all drink the cup of wrath which the prophet hands them from God. But then it is indeed destruction that stands nearest at the door. God cannot come with His kingdom and His people except where the other kingdoms and peoples are crushed and brought low; and God cannot come with healing and life where there is not sickness and death. He kills in order to make alive. Yet since God wills not the death of any sinner, but that he should turn and live, therefore no prophet is appointed solely to tear down; but even for Jeremiah, who received the heaviest task of all the prophets, even for him it was added that he should build and plant. And as he had to proclaim judgment and ruin both upon Jerusalem and upon Babylon, so he was also granted to bear witness to grace and restoration both for Israel and for its enemies, when the Lord had broken their hearts.

Hereby the prophet Jeremiah was consecrated to be a preacher of repentance, a prophet of peace. But he was also consecrated to be a man of sufferings, a man of contradictions. For he who was to strike down all that upon which Israel relied, and to deprive it of all repose, so that it became poor as a heathen people, he had to endure all the hatred and persecution of those who imagined themselves to be something. Yet in all his suffering and in all persecution he was to have a firm refuge, a citadel of deliverance in distress: the Lord, the God of Israel. Therefore, on the day of his calling, he was given with him that rock‑firm word which could not fail: “And I, behold, I make thee this day a fortified city, and an iron pillar, and brazen walls against the whole land, against the kings of Judah, against its princes, against its priests, and against the people of the land. And they shall fight against thee, but they shall not prevail against thee; for I am with thee, saith the Lord, to deliver thee.” (Jer. 1:18–19).

Thus the Prophet went forth in his calling, frail, gentle, and fearful according to the flesh, yet strong in the power of the Lord. He was appointed to tear down peoples and kingdoms by his word. He was to proclaim unto them the crushing judgments of God and to be a witness to their fulfillment; but above all he was to overthrow the mighty fortresses which his own people had set between themselves and their God through the stubbornness of their own hearts and their trust in false and insidious words.

We have already said that Israel, under the reign of Josiah, passed through a remarkable season of agitation, and the Prophet labored with burning zeal in that time. The first false ground of consolation against which he lifted the hammer of the Word was Judah’s confidence that it had survived the kingdom of Israel, which had fallen before the power of Assyria. Judah believed itself more secure than her sister Israel; but the Prophet tore away this deceitful comfort and declared that the kingdom of Judah had doubled her sin, for Judah beheld her sister’s fate, and yet did not repent. This was to wound Judah’s pride at its most tender point; for Judah had imagined that Israel had fallen because she had attacked Judah, and it was commonly believed in Judah that her own sin was still less than that of her sister. Therefore the people were embittered against the Prophet, who spoke so freely the word of God, and they would not listen to the man who called them to repentance.

The priests had gained much outwardly through Josiah’s reformation. Their position was strengthened and honored, and the service of the temple was restored; but their hearts were little bowed. For thus Jeremiah bears witness concerning them: “The prophets prophesy falsely, and the priests rule at their direction; and my people love to have it so” (Jer. 5:31). This brief saying opens for us a deep insight into the true condition of things. The prophets proclaimed, Peace! Peace! and healed the hurt of the daughters of Israel lightly; for there was no peace, but the judgment of God’s wrath. And the priests took advantage of the prophets’ reckless doctrine of consolation. They confirmed the false prophecy with the lying word: “The temple of the LORD, the temple of the LORD, the temple of the LORD are these” (Jer. 7:4). As though they would say to the people: So long as the temple is in our midst, we cannot perish; make your gifts great and your sacrifices many, and it shall go well with you.

Then the Prophet arose in burning zeal, and, according to the word of the LORD, stood in the gate of His house and cried out to all Judah: “Trust ye not in lying words. This house have ye made a den of robbers; and behold, as Shiloh, where I dwelt, which is now desolate and forsaken, so shall my house become.” From that day the priests, whose heart’s pride was the temple, became Jeremiah’s enemies; for he had torn down their stronghold. And they would not hear the call of the LORD: Repent ye.

Nor did it go any better with the false prophets. The tangled snare of their pride was their reckless cry: Peace! Peace! They gave themselves out to proclaim the truth of God; then came the Prophet of truth, sent by God, and without mercy he tore the mask from their hypocritical face and bore witness to them with the word of the LORD: “Hearken not unto the prophets that prophesy unto you, and make you vain; they speak a vision of their own heart, and not out of the mouth of the LORD” (Jer. 23:16). “Behold, I am against them that prophesy false dreams, saith the LORD, and do tell them, and cause my people to err by their lies, and by their light-mindedness; yet I sent them not, nor commanded them: therefore they shall not profit this people at all, saith the LORD” (Jer. 23:32).

Then there arose a bitter hatred against Jeremiah in the hearts of the prophets, because their pride had been broken. And they would not hear the grace of God: Repent ye!

So long, however, as King Josiah lived, Jeremiah was relatively secured against the malice of his enemies. But Josiah fell in the battle at Megiddo; and after Joahaz had reigned three months and, according to Jeremiah’s prophecy, had been carried away to Egypt to die there, then came the wretched reign of Jehoiakim, in whose time the king himself also became the prophet’s enemy. For in his fourth year the prophet delivered that mighty discourse which is written in the twenty-fifth chapter of his book. In that same year Nebuchadnezzar became king in Babylon, and the prophet declared plainly that because Israel for three-and-twenty years had heard the counsel of the LORD through him, early and unceasingly, yet would not turn again, therefore Nebuchadnezzar should come, and Judah and all the surrounding nations should serve the king of Babylon for seventy years.

Then should the voice of mirth and the voice of gladness, the voice of the bridegroom and the voice of the bride, the sound of the millstones and the light of the lamp vanish from Judah, and the land become a desolation and its cities ruins. Then neither should the temple of the LORD avail them who trusted that they were the chosen people of the LORD; nor should alliances with foreign princes avail them who held flesh for their arm; but all supports should break, for the hour of the LORD was come. His cup filled with the wine of wrath was full, and the prophet must now take it and bring it to all the nations upon whom the judgment came. But Jerusalem must drink first; for with His own city would the LORD begin.

Thus Judah was made equal with the heathen. Thirty years of hardening had filled the measure. And then also the embitterment against Jeremiah arose. The priests and the prophets and the people laid hold on him and said: This man is worthy to die, for he hath prophesied against this city (Jer. 26:11). But this time the princes delivered him. And yet in that same year Jeremiah wrote all his prophecies in a book, and it was brought to King Jehoiakim; and he read it, and when three or four leaves had been read, the king took the scribe's knife and cut them asunder and cast them into the fire and burned them. And he commanded that Jeremiah and his faithful scribe Baruch should be seized; but the LORD hid them.

And Jeremiah, who had overthrown both the king’s fleshly policy and his fleshly reliance upon the promise of David’s eternal kingdom, Jeremiah could now number the kings also among his enemies.

What had not his faithful heart suffered under this! Day by day he saw his people draw nearer to their destruction; he saw the love of his God, which without ceasing called and allured them unto repentance and salvation, mocked and despised; he himself was persecuted, and saw his life threatened daily by people and priests, by prophets and princes. — And his soul was beaten down in the trial, so that he cursed the day on which he was born. Yet still he was constrained to go with the message of the Lord’s wrath; for if Israel would not be broken by the word of the Lord, then it must be broken by the judgment of the Lord. That judgment was near; Jeremiah was to live to behold it.

The last king before the captivity to Babylon was the weak and faithless Zedekiah. It was Nebuchadnezzar, king of Babylon, who, after having carried away Jehoiakim and the best of the people with him, set Zedekiah as king over those who remained in Jerusalem. Zedekiah had sworn loyalty and faithfulness to the king of Babylon, and the kingdom of Judah was tributary to the world-empire. Jeremiah bore witness to the people to the same truth which the Savior later set before Israel, when it was subject to the Roman emperor: “Render unto Caesar the things that are Caesar’s, and unto God the things that are God’s.” But to no avail. Zedekiah trusted in political cunning and made flesh his arm. He entered into alliance with several other petty princes, who like himself were subject to Babylon; and in reliance upon the help of Egypt, he and his confederates broke their oath of loyalty to Nebuchadnezzar, refused to pay their tribute, and set themselves in a posture of defense against the advancing army of Babylon.

Then Jeremiah lifted up his voice in rebuke. As a prophet of the Lord he had borne witness to the people before the uprising and had said: Serve the king of Babylon, and ye shall live (Jer. 27:17). And after the uprising, when the king of Babylon and all his army came to fight against Jerusalem, then his word sounded thus: “Thus saith the Lord: Behold, I give this city into the hand of the king of Babylon, and he shall burn it with fire” (Jer. 34:2). But Israel would not hear. The people were seized with fear for a little while when the mighty army appeared before the walls of Jerusalem, and in their terror they resolved to release their Hebrew menservants and maidservants, in order thereby to appease the Lord’s wrath and to increase the number of their fighting men. But this was only a work of fear, not of faith; and therefore, when Nebuchadnezzar for a little while withdrew from Jerusalem to meet the Egyptian army, which had indeed gone out against him, then the Jewish masters took back their word and once again deprived their servants of their freedom.

Then the wrath of the Lord was kindled against them, and he sent Jeremiah to say: Ye have proclaimed liberty to your servants, and have taken back that liberty; therefore will I proclaim liberty for the sword, and for the famine, and for the pestilence, to rage among you. And I will bring the army of the Babylonians back unto this city, and they shall fight against it, and they shall take it and burn it with fire; and the cities of Judah will I make a desolation, without inhabitant (Jer. 34).

And the army of Nebuchadnezzar came, never again to depart from Jerusalem until it had been made a heap of ruins. But if the bitterness against Jeremiah had been great while he proclaimed the sentence of punishment, it became doubly violent when the judgment came. It is the dreadful mystery of hardening, that the more clearly the truth shines before the eyes, the more firmly the eye closes itself against it; the heavier the hand of the Lord lies upon it, the more stiffly it raises its neck against the hand that strikes it. And when Jeremiah spoke to the besieged people of the salvation that lay in the humiliating way of surrendering themselves to the king of Babylon, then he could no longer be endured; he had to be bound and cast into prison. And as though it were not enough that he was shut up, the princes would yet take his life by casting him into an empty cistern, where, sinking down into the mire, he was to find death by hunger.

But the Lord had promised to deliver him out of the hand of his enemies; and Ebed-melech obtained from the king permission to rescue him from the pit. And in prison he was forced to remain until the day when Jerusalem was taken.

Wondrous are the ways of the Lord. There in the prison, where Jeremiah seemed to human eyes truly laid away in darkness, there the Lord caused His light to arise for him and showed him Israel’s glorious future. From prisons there commonly go forth the wildest threats against the people, curses against God and men; but from this prison there go forth the most life‑blessed promises in lovely words, which still gladden the city of God.

Jeremiah, who in the prison waited for the fulfillment of the destruction he had proclaimed, received precisely there also the charge to build and to plant. In the prison Jeremiah was commanded to buy a field which lay trampled beneath the feet of the Chaldeans, a piece of land near Jerusalem, as a sign that when the Chaldeans had destroyed everything, then the Lord would restore everything; that out of death itself the Lord would bring forth life. After this bitter winter there was to break forth a glorious spring.

In the prison Jeremiah received the word of the Lord, that the Lord would gather His people from all the lands to which He had scattered them, and bring them back to this place and cause them to dwell securely. And they should be the Lord’s people, and He their God. And they should be given one heart and one way.

In the prison Jeremiah received that glorious message: “Behold, the days come, saith the Lord, that I will perform that good word which I have spoken concerning the house of Israel and concerning the house of Judah. In those days, and at that time, will I cause the righteous Branch to spring forth unto David; and He shall execute judgment and righteousness in the land. In those days shall Judah be saved, and Jerusalem shall dwell safely; and this is the name whereby it shall be called: The Lord our Righteousness.”

Thus Jeremiah built and planted a people and a kingdom which should endure forever, in the midst of the ruins of fallen kingdoms and shattered peoples. And in the prison, where the narrow walls shut him in on every side, the prophet’s eye was opened to see unto the last days and unto the farthest ends of the earth, unto the kingdom which by the Gospel comes to all nations.

Thus the prison had, in a spiritual sense, been a season of rest for the Prophet from his struggle with the hard and rebellious people. Yet he was not finished. His imprisonment was indeed broken open by the Chaldeans, and life and freedom were offered him with those who were carried away to Babylon. But Jeremiah chose to remain behind in the devastated Judah and to labor among the wretched remnants of the people, if perchance a soul might be saved. The Lord had sent to the captives another mighty Prophet for awakening and consolation, the Prophet Ezekiel. Jeremiah remained with the scattered people in Judah, over whom the king of Babylon set Gedaliah as governor. But even now he was not permitted to live in peace. Certain robbers, led by Ishmael, slew Gedaliah, and the people fled in terror to Egypt and forced Jeremiah to go with them. There also he continued to bear witness to them that the Lord’s punishment would come upon them for their unfaithfulness and idolatry. His death is unknown, but an ancient tradition says that he was stoned by his own countrymen.

Such is the life of the man who received the heaviest prophetic calling, and who saw Jerusalem’s last days before the exile. No prophet is so much as he a pattern of the suffering servant, who was required to proclaim to the Israel hardened a second time that Jerusalem and the Temple should be laid waste, until there was not one stone left upon another, because it knew not the time of its visitation. But from the life of this man of God there goes forth a crying voice to all who are set as watchmen upon Zion’s walls: Be faithful in the Lord’s work and stand manfully upon your watch; for the Lord will require the blood of His people at the hand of the unfaithful watchmen. And there is a cry of awakening to all dead Christians: Trust not in lying words! Hearken not to priests who cry, “The temple of the Lord, the temple of the Lord are these,” nor to prophets who cry, “Peace! Peace! and no danger.” Turn ye, turn ye! Stand in the Ways and ask for the old paths, where is the good way, and walk therein, and ye shall find rest for your souls. Behold, the Way, Jesus Christ, is revealed unto us; God grant that none may harden themselves and say, We will not walk therein. For the Lord’s judgment is near to be revealed, and from the house of God it shall begin.

\subsubsection{Jeremiah’s Book}

In the Old Testament there is told of many Prophets who lived and labored in Israel, yet who have left behind no written record of the spoken word which the Lord sent them forth to bear unto Israel. There is the great prophet of repentance, Elijah—he who contended against the ungodly king, the idolatrous queen, against the priests and prophets of Baal; he who prayed that it might not rain, and it rained not for three years and six months, and he prayed again, and heaven gave rain and the earth yielded her fruit; he who together with Moses met the Savior upon the Mount of Transfiguration—how gladly would we not have had his penitential sermons written down as a testimony for us! There is Elisha, the man of faith, with his unwearied labor for his people; there is John the Baptist, the greatest of those born of women, his voice crying in the wilderness: Prepare ye the way of the Lord! What joy would it not have been to possess a book from their own hand, wherein they themselves had set down what the Spirit had given them to testify unto Israel! But the Lord has willed it otherwise.

There are other Prophets, to whom He granted both to speak and to contend—men to whom He Himself revealed what should come to pass in the latter times, and whom He commanded to write down what was revealed to them, as a hope for the future for God’s children under the old Covenant, as admonition and consolation in the faith for us who live in the last days, in the blessed times of fulfillment.

The writing Prophets are all found within a definite period of the history of the people of Israel. And this period is especially that in which Israel more and more hardened its heart against the Word of God, and thus the proclamation of the Prophets did not immediately bear the blessed fruit of repentance in the stiffened people. \textbf{Therefore the Prophets were given command to write down the sermon, that the Word which in the days of the Prophets seemed only to bear the fruit of hardening might, in the Lord’s good pleasure and appointed hour, also bear a rich fruit of awakening and contrition when it was read anew.} At the same time God also caused, in the Gentile world, the mighty kingdoms of the world to arise as instruments of punishment over the obdurate Israel; and the Prophets foretold both their victories and their fall, as a testimony for the Gentiles in the fullness of time, that it was the Lord who gave His people into their hand, and that it was not their great strength and proud power which overcame Israel and Israel’s God. The omnipotence and mercy of God were to shine forth toward the Gentiles in the prophetic Scriptures, when the Lord, through the Gospel of His Son, caused the rays of grace to shine into the darkness of the Gentile world.

Thus it is that precisely in Israel’s darkest hour and in the proudest days of the world-kingdoms the Lord caused His good Word to be richly written down by the holy Prophets, that His faithfulness and truth might appear all the more gloriously—unto the Jew first and also unto the Greek—when the Sun of Righteousness arose for both upon the morning of Christ’s resurrection. Then were the Scriptures to burn within the disciples, when Christ and His Holy Spirit opened the Scriptures unto them.

Also the prophet Jeremiah was to write in a book what the LORD caused him to see. Why he was to do this, the LORD himself has explained in Jeremiah 30:1–3; 30:24; and 36:2–3. It is because the LORD still wills to gather all his judgments and his promises to his people into a book, if so be that they might be willing to hearken unto them. It is because in the latter days he will turn again the captivity of his people Israel and bring them back to the land which he gave unto their fathers; then shall they acknowledge his faithfulness. It is because Israel is now hardened and understands not the good counsel of the LORD and his thoughts of peace toward them; but in the last times, when the LORD has poured out his burning wrath upon his people and upon the proud heathen, when he has fulfilled the thoughts of his heart, then shall they understand it.

We have already seen that the two great matters in Jeremiah’s calling are these: that he is set over nations and kingdoms, and that he is appointed first as a prophet of destruction and thereafter as a prophet of restoration. We must therefore expect in his book to find prophecies against Israel and prophecies against the heathen, and that these prophecies must contain, first and foremost, proclamation of judgment and thereafter promise of salvation.

And so it is indeed the case that, just as the prophet’s life was governed by the divine calling, so also his book is determined thereby and ordered according to its content. Over Jews and heathen he is set, and his book of prophecy sets forth truer words for them both. Crushing and raising up were to mark his labor, and his book is a mighty juxtaposition of smiting and healing oil and wine poured into deep wounds.

The prophet’s book is divided into two principal parts, according to the order: the Jew first and then the heathen. The first principal part consists of the first forty-five chapters and is directed chiefly to Israel. The second principal part is chapters 46–51, where the prophet turns to the heathen; to this chapter 52 is added as an appendix, which once more recounts the destruction of Jerusalem, a matter so dreadful and so grave in Jeremiah’s entire prophecy and life.

The portion of the Book that concerns Israel, chapters 1–45, is, however, so extensive that in this Guide for believing and zealous readers of the Bible we shall yet append a few additional remarks concerning it, which we hope may assist one or another in his labour with the precious Word of God. Alas, only far too few are truly inwardly familiar with their Bible; and even those who by the grace of God have found life and salvation in the Word are only too often accustomed to read the Scriptures as though they were nothing more than a collection of detached verses, having little true connection one with another. All the more important, therefore, is it—by the help of the Spirit of God—to attempt to open up the Scriptures, if perhaps someone might be led ever further in upon the green pastures, and find refreshment by the waters of rest, even from the streams of the river that make glad the city of God.

There are three sections in the portion of the Book that concerns Israel: chapters 1–29, 30–33, and 34–45. The prophet’s first calling was to pluck up and to tear down, to destroy and to overthrow by his mighty prophetic word. He was to strike down all the false supports upon which Israel relied; he was to tear away from them everything upon which they built their fleshly security—both their confident trust in election and their priests’ vain cry of the temple of the Lord; both the prophets’ shouting, “Peace! Peace! and no danger!” and the kings’ fleshly confidence in the promise made to David. By the Word he was to seek to bend the heart of the people into brokenness before the Lord, if only they might repent. But if this should not succeed, then he was in his word to declare to them that the people would be uprooted from their land and carried away to their enemies in bondage; that the land would be laid waste until it became a wilderness; that the temple would be torn down and burned with fire; that the king would be cast down from David’s throne and have no son to inherit the power. This prophetic demolition by the Word Jeremiah sets before us in the first section of the Book concerning Israel, chapters 1–29, which we may call the Book of Demolition.

In these first twenty-nine chapters there is first related the calling of the Prophet in chapter 1. Thereupon begins the great series of crushing penitential sermons, in which here and there flashes a glimpse of hardening forth between the heavy, dark storm-clouds. Judah is punished in chapter 2, because it has exchanged the LORD for idols, because it has forsaken the fountain of living waters and hewn out for itself broken cisterns that can hold no water. Chapters 3–6 are a mighty sermon against the pride of Judah, which exalted itself above its sister, the kingdom of Israel. Judah has seen Israel’s fate and punishment, yet has not repented; therefore its punishment shall be the heavier. Priests and prophets and princes, who have strengthened the people in its sin upon the brink of the abyss, shall be utterly helpless in the day of visitation, and the whole people shall be called rejected silver; for the LORD has rejected them (6:30).

Then follows in chapters 7–10 the great Temple Sermon, so called from 7:2 and 4. The self-righteous reliance upon the Temple is altogether in vain; and because it only hardens the heart of the people, therefore such judgment shall come upon them that death shall be preferred to life by all the remnant that remains of this evil generation (8:3). Thereupon follows discourse upon discourse, in which Israel is exhorted to repentance and threatened with destruction, now by words, now by symbolic actions. Israel is set on a level with the heathen in punishment, and the Prophet is not permitted to pray for them; and ever harder sounds the judgment, until the Prophet in chapter 19 receives command to go, buy a potter’s earthen flask, take it with him down into the Valley of the Son of Hinnom, defiled with idolatry, and there break it in the sight of the people as a sign that thus shall Israel’s people also be broken and become utterly useless, like a warped vessel of the potter, because it has hardened its neck.

In chapters 20–24 the overthrow turns itself more toward individual prominent persons. In chapter 20 judgment is pronounced upon Pashhur, who mistreated the Prophet; in chapters 21 and 22 upon the wicked kings; in chapter 23 upon the evil shepherds, and especially the false prophets; yet here also a glorious light breaks forth in the midst of the darkness in the promise of the good Shepherd, David’s righteous Branch. In chapter 24 the judgment is directed against those left behind in Jerusalem after the removal of Jehoiakim, the bad figs that cannot be eaten. In chapter 25 the measure of hardening is full; there the seventy-year exile is proclaimed. And in chapter 26 the Prophet shows that the LORD’s Temple will not shield Israel from judgment; for it too shall be broken down. Nor can Zedekiah’s alliance with the heathen princes stay the judgment (chapters 27 and 28). Finally, those already carried away shall not expect any speedy return from Babylon; therefore the Prophet sends them a letter, that they should with patience be content in the foreign land until the LORD’s time comes to bring them back (chapter 29).

This Book of divine judgment is, in chapters 1–24, chiefly a book of penitential preaching, whereas chapters 25–29 are a definite proclamation of a judgment that is unavoidable, and whose time and hour are already revealed, in that at the same time the last grounds of comfort, to which the hardened people clung, are torn away without any sparing.

The prophet’s calling had also set him to build and to plant, and in his book there follows upon the heavy discourses of demolition and crushing a most beautiful Book of Consolation. This is the second section of Israel’s portion, chapters 30–33. The prophet proclaims a great time of tribulation over Jacob; yet he shall be saved out of it. It is so painful, that little while, for it is an hour of birth, and the Lord brings forth for himself a people (chap. 30). The people that are born are the tribes of Israel and Judah, whom the Lord gathers; and when he has gathered them, he establishes with them a new covenant, a covenant of forgiveness of sins and of regeneration, so that the old people has passed away, and behold, there is a new people, born of God (chap. 31). A sure sign of this superabounding grace of the Lord the prophet receives in the prison. In the midst of this siege, which ends with the fall of Jerusalem and the devastation of the land, he is commanded to buy a field, as though there were the deepest peace in the land; thereby the Lord will point forward to the great peace that shall be in the land when he gathers the scattered into one people and gives them one heart and one way (chap. 32). The people that are gathered and born shall be a people in joy and blessedness, an innumerable host of singers and priests unto God (chap. 33).

Finally, chapters 34–45 form the third section of Israel’s portion. Its purpose and content are to show how the Lord fulfills the words of his servant Jeremiah. Jerusalem fell, the temple was burned, the people were carried away, the land became a wilderness. This section is therefore chiefly narrative, with brief prophetic utterances interwoven into the account. Yet not only is this section the Book of Judgment over the hardened people; it is also the Book of Jeremiah’s sufferings. Here we are told in great detail how the people, upon whom the Lord’s punishment fell, in their bitterness turned against the Lord’s messenger and mistreated him in many ways, so that he truly became a foreshadowing of the suffering Savior, who suffered, the righteous for the unrighteous. Thus these chapters are placed here to show that God is not mocked. His threatenings he fulfills, and even over those who are his prophets and messengers he brings the purifying fire of tribulation, that they may come forth as gold tried in the fire and be unto praise and glory and honor at the coming of Jesus Christ. The faithful prophet had to suffer with his people; but God gathered his tears and his pain and caused them to become a harvest of joy, when his time came to comfort his people, to give it double from his hand for all its sins, to gather all its scattered ones and to make the children of the desolate more than hers who had a husband. Jeremiah was the one who more than any other prophet in the old covenant went forth with weeping; now the time has come when another crop of the gray-haired fathers is gathered into the barn. May the Lord grant of his grace that we all might be of this new people of God, with one heart and one way! For us all this is written, for us all this struggle is waged; let us therefore diligently consider also this man of sorrows and his day’s work in the Lord’s vineyard, that we might thereby be strengthened to remain faithful in the suffering that is entrusted to us.











