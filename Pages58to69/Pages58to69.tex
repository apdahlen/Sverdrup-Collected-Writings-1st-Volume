
\begin{center}
\includegraphics[width=0.9\textwidth]{OpenImage1.png}
\end{center}

\bigskip

\section{Second Section}

The pieces and essays included in this section may be regarded as segments of one facet of Professor Ederdrup’s great work within our Church, in that they provide an insight into his clear and strongly pronounced historical mind. He had learned from the past to judge the present, and from them both to read the signs of the future.

In the first piece, “Christendom in History,” he points out that it is far from being a matter of consolation that we here in America have “Christendom and not a State Church”; rather, history teaches that neither can Christendom escape the judgment of God nor avoid being corrupted at its root, unless it is willing to allow itself to be governed and led by the Spirit of God.

The second piece, “The Church and the Ecclesiastical Bodies,” is a masterly treatment of its subject. It points out both what “the Church” is, why a plurality of “ecclesiastical bodies” must necessarily arise, and wherein their right to exist consists. The strong tendency of the Norwegian Synod to regard itself as “the Church,” and the other Norwegian Lutheran church bodies in this country as sects or parties, forms a background against which this essay will naturally have to be viewed.

The third piece, “The Congregation within State Structures,” presents a beautiful picture of the Christian congregation and its standing under persecutions. It bears witness to the invincible power which the congregation possessed so long as it preserved the Spirit and fidelity to its calling and to its Savior. The piece further contains many particularly interesting and valuable details concerning the legal position of the Christian congregation within the pagan Roman state.

The fourth piece, “Reminiscences from Norway,” has not previously been published. It is a fragment of a book and was written by the author during a stay in Norway in the summer of 1886. A table of contents found together with the manuscript shows that it had been the author’s intention also to discuss the Bible translation, the hymnbook question, the deaconess cause, Inner Mission, and the Seamen’s Mission, in addition to the subjects which he has treated. It is known to me that at least on one later occasion (1889) he also gathered material and made preparations to complete this work; but other matters always made such heavy demands on his time and strength that this final undertaking had to be deferred.

There have naturally occurred several changes in ecclesiastical matters in Norway since 1886, both with respect to the people as a whole and to individual men. This must necessarily be borne in mind when reading this otherwise exceedingly interesting sketch. Thus the author’s characterization of Bishop Heuch is entirely correct and very apt for the period with which it is here concerned. But in his later years Heuch became quite another man. The extremely high-church parish pastor of Kristiania and editor of the “Lutheran Weekly” later became one of the most low-church bishops Norway has ever had. Yet viewed from the point in time at which these “Reminiscences from Norway” were written, they present a particularly apt picture of the ecclesiastical situation in our old country in the middle of the eighties, as well as an understanding portrayal of the development during the period under consideration.

“Reminiscences from Norway” also form an important contribution to the understanding of Professor Sverdrup’s own spiritual development. When we have read through these pages, we gain some sense of the significance of the struggles through which he himself must have passed in his youth, and of how he also came to establish with spiritual clarity his resolve to choose the way of the Congregation and faithfully to follow it unto his dying day.

As a — though certainly very condensed — parallel to “Reminiscences from Norway,” the fifth piece is then added, “The Norwegian Lutheran Church in America.” In the briefest possible form, and with particular regard to the work of the Seamen’s Mission, the author has here given the French Lutherans a portrait of our people’s ecclesiastical position in this country. The piece was written for the French Lutheran church paper “Le Témoignage” and was thereafter published in translation in the Norwegian paper “Vestlandsposten,” which accompanied it with the following editorial note: “This interesting portrayal has been written by Professor Sverdrup at the request of the French Lutherans’ paper ‘Témoignage,’ from which the article has been translated for ‘Vestlandsposten.’ As readers will observe, it is the history of the Norwegian Lutheran Church in America which he here unfolds for us in brief outline, and it is not a contribution to the controversy being waged on that subject; this is merely mentioned, and that in a manner which will hopefully give offense to no side. — Ed.”

The final piece in this section, “The Lutheran Church,” was written by the author shortly before his death. Concerning this, the then chairman of “Kristen,” Pastor C. C. Smith, has given me permission to relate the following: Early in the spring of 1907, Pastor Smith was requested by the Census Bureau in Washington, D.C., in his capacity as chairman, to submit a report concerning “The Lutheran Church.” At that time he was suffering from a prolonged illness which for months rendered him almost entirely unfit for work, and he therefore requested Professor Sverdrup to write the portion of the said report which was to deal with the principial aspect of the “Free Church” and its work. This request he met with his customary readiness, and it is this report which is here presented in Norwegian — not, however, in translation, but from the author’s own hand, since he had written it both in English and in Norwegian. It gains increased interest from the fact that it is Professor Sverdrup’s final word on what, in his view, the “Free Church” is and ought to be. In this respect, this little sketch may be said to constitute his testament. — Publisher.



\subsection{The Free Church in History}

A manuscript which, judging from the handwriting, belongs to the very earliest years after the author’s arrival in America. It is presumably the manuscript of a speech; at any rate, I have found no trace that it was ever published before. See moreover the note on page 58. — Ed.

This section appears on pages 60–69 of the original volume. — Present Ed.

\bigskip

Throughout Christendom, in these days when the development of the world’s freedom has shown the way to self-government, there is a living question concerning the free Church. When there is clamor on all sides, without distinction—by conservatives as well as by liberals, by monarchists as well as by republicans, by those who fervently confess the Christian faith as well as by those who fervently deny Christianity—this leads of necessity to the thought that this name, “the free Church,” carries with it many different conceptions. At once we must suppose that some who desire a free Church would therein have a substitute for absolutism, which is a lost cause in the political sphere, while others would therein seek the same freedom as in the state, and others again perhaps seek the Church’s true good, regardless of whither this might lead them.

If, then, the free Church has as yet no definite meaning, but merely designates the Church which is independent of the state, and which otherwise each person imagines for himself as he will, it is evident that here in America—where no other kind of Church exists than the free Church—it will be most timely to attempt to make clear to ourselves what this is to mean.

Meanwhile, the Church is not of yesterday, nor is America the only place where there has been a free Church; and it is therefore fitting for us first to cast a very brief glance backward, to see how the free Church has been before our own days, and to what it has developed, in order that we may, if possible, learn to take heed of the dangers and to make use of the advantages in that condition of affairs in which we are set.

It is an incontestable fact that if by a Free Church one understands a church that is independent of the State, then there has been and still is a great and powerful Christian Church that deserves this name; it is the Catholic. That this Church at various times has deeply entangled itself in affairs of state is here of no consequence, since this did not rest upon dependence upon the State, but upon a supremacy over the State, which it precisely possessed as a Free Church. In later times, however, this supremacy over the State has in many ways been compelled to be relinquished.

It can therefore not be subject to any doubt that a serious investigation of the historical development of the Free Church, of its dangers and its advantages, must begin with the Catholic Church, because there we possess the longest experience and the most legible records. There it is written in strokes of stone how a Free Church labors under the influence of the world, and what we are liable to inherit from it if it begins in the same manner. For however much one in our own day, in fanatical Protestantism, may wish to exclude it from the number of Free Churches, we must never forget that it has been Church in a sense such as no other has been. And we add that the strong zeal to deny the Catholic Church its right is to a great extent grounded in a fear lest any instruction or warning should be drawn from its sorrowful history. [Sverdrup does not enjoy condemning Rome; he fears becoming her. — Present Ed.] The Pope is called Antichrist, and all Catholic endeavors unchristian, in order that no one may be led to examine whether the spirit of the Antichrist and hierarchical tendencies might also be found among ourselves. We stand in America in a manner similar to that in which the Catholic Church once stood in the Roman Empire, and no one ought to marvel if there is danger of going the same way now as then, since the way is always broad toward Catholicism, but exceedingly narrow toward a truly free Christian Church.

The Catholic Church has not always been what it is today. It has suffered and struggled; it has labored and prayed; it has had blood-witnesses and fire-witnesses, before it came so far that it itself gave the “heretics” their blood-witnesses and their belts upon the stake. From the days of the Apostles down to our own time it has one single continuous history, which never seems broken or interrupted, so that we ask ourselves with dread whether it is indeed the case that the Free Church has no other path to tread; and we are driven to the most careful examination of whether this in full earnestness is the way of the Church, or whether it might perhaps be a byway.

It is impossible here, within a few words, to pass through this entire long development. We have here only briefly to point out the dangers under which the catholic Free Church succumbed, so that everyone may judge how near these same dangers lie to every Free Church.

There is a goal set before every Free Church that has even a spark of Christianity within it: it is the Kingdom of God. The Kingdom of Heaven upon earth, with its eternal consummation in heaven, this is the aim of Christ’s work; and this Kingdom He Himself is—just as He is the second Adam, or the firstborn of the new race of mankind—the living foundation of the eternal Zion. Never has this goal shone more clearly for any communion than for the first poor Christians in the midst of the heathen. Nowhere has this goal been drawn in such pure and radiant colors as for the Christians in the catacombs of Rome. Here, in these burial vaults beneath the great city of the world, where the splendor and corruption of a world-empire roared above them in all its greatness and abomination, the darkness became a light round about them, when their hearts were lifted up in faith toward the eternal glory in which their sufferings and tribulations were to be forgotten for evermore. There they felt the powers of the new world stirring within them, and in hope they saw the empire crumble into dust before the Lord, who is at hand. It was the free Church which, clothed in the full armor of the Lord, cast itself into the battle, while the bodies of martyrs were laid to rest with songs of victory, and their graves were marked with the crown of honor and the palm of triumph by those who hoped to follow after both their faith and their death. That Church which is not permeated by this unshakable conviction, that the Kingdom of God in eternal glory is its goal, has never deserved the name of the Bride of Christ.

It is this strong life of the Spirit and this world-overcoming faith in the Saviour and His work which constitute the Christian foundation upon which the Roman Church rests—and no Free Church can rest upon anything else. It must have firm ground upon the Rock, the eternal Rock, if it is to dare to believe in its eternal calling and its eternal life. But if the beginning was so good, how then has the outcome become what we see before our eyes?

The imperial realm of Rome was heathen. It was so thoroughly interwoven with heathendom that when the Emperor Constantine, in the fourth century after Christ, deemed it necessary to impart to his empire a new spiritual vitality by taking Christianity under his protection and seeking to make it the religion of the state, he at the same time found it politically prudent to provide his realm with a new capital in Constantinople. The new spiritual central power which Christianity was to become required a new temporal center. From Constantinople the Roman Empire was to be reshaped in Christian form, and Christianity was to assume the same state-serving position which heathendom had formerly held.

It was not without joy that all Christians heard of the Emperor who had become a Christian. All alike saw in this a new victory for Christianity, and many Christians even looked with gladness upon the new power which Christianity now received, since it had the strong arm of despotism upon which to lean; and in faithfulness to the truth we must add that some Christians zealously gave their approval to Constantine’s efforts to transform the free Church into a state Church. The many and powerful heresies which were just then breaking forth were to be warded off by the state Church. Thus arose that peculiar form of church polity which, after the ancient name of Constantinople, is called Byzantinism. This is the name for that system of church government in which the Emperor is the possessor of all power in the Church because he is so in the State, and in which the faith and doctrine of the Church are determined according to the Emperor’s or the court’s whim, or, at best, according to political prudence.

This system had its attraction, in that the Church thereby at once became a kingdom, which indeed was the same as the imperial realm, yet nevertheless bore the name of the Kingdom of God. It had its attraction for the flesh, in that the Church at once obtained power over her enemies and could persecute the persecutors and punish the heretics. But it had this corruption within itself, that the kingdom was no longer in truth the Kingdom of God; that the Church had acquired worldly power and worldly interests; that the Church was no longer herself nor her own ruler. And it brought with it this corruption for the people, that consciences were no longer free, but that the Emperor’s faith was law and rule for the people. From this followed the fullness and the hypocrisy which increased within the Greek Empire and made it what it now is: a home of barbarity and bloodshed, of bondage and brutality.

And these dangers and this devastation followed whenever the Emperor was orthodox, right-believing; and it shall be the curse of Byzantinism everywhere, that it destroys both Church and people, both land and realm, be it as orthodox as it will. But for the Church there came in Byzantinism yet another danger besides. For if the Emperor was heretical—what then? Then the whole Church, or the whole people (for Byzantinism makes no distinction between the two), must change its faith and be converted to the Emperor. This was a danger which one may perhaps have forgotten in the first enthusiasm, but which the Byzantine emperors were soon to make sufficiently evident to all who had earnest concern for the Church.

It was beyond doubt Constantine’s intention that Byzantinism, or the most fully developed form of the state church, should now replace ecclesiastical freedom throughout his entire empire. He had not the remotest idea that this could be contrary to Christianity; on the contrary, it was his conviction that it was a divine calling laid upon him to act thus. Yet he did not succeed, and the favor shown by his sons to the heretics called forth a lively opposition. Nor can it surprise us that Rome and its bishop stood with zeal upon the side of this opposition. The division of Constantine’s empire led to divergent developments. Rome continued to possess a free Church, interrupted only for a brief period.

Rome felt itself involuntarily set aside from the day when Constantine transferred the capital to Constantinople, and the Roman Christians for many reasons shared the dissatisfaction that ran through the world-city. But as it gradually became clear to them that it was not the Emperor who would serve the Church, but that the Church was to serve him, new thoughts seem also to have awakened—at least in the Roman bishop. For it was evident that the Church had become a kingdom, and yet at the same time a political ruler had become lord within that kingdom. The free Church understood that this was bondage. It envied the despotic power of the Greek Church, yet counted itself happy that it had no despot over it. The Christian Byzantine empire became the seduction of the free Church, in that it began to shape its conceptions of the Kingdom of God after this pattern, with only this difference, that not an emperor, but a bishop, ought to be the head of God’s kingdom.

If, then, the free Church was to assume such a position—that, like the Greek Church, it should become an ecclesiastical despotism, an absolute ecclesiastical sovereignty, in which the decisive aim was to hold the world and the nations beneath itself—then it was at the same time a settled matter that the Roman bishop must be the ecclesiastical emperor in the realm. And there was no lack of compelling reasons for the free Church to strive after power over the people, and for the Roman bishop to assert his supremacy over the Church.

It was the Age of Heresies, and it was evident that the imperial power was wholly unfit to preserve the pure doctrine; for the Emperor himself might become a heretic. Amid the many upheavals of the world, the Church was compelled to lay the greatest weight upon the purity of doctrine; but how was it to be preserved? The pagan ideas of the ancient world had followed the pagan population into the Church, and these pagan ideas exercised a great spiritual power over men’s minds, so long as the study of the pagan writings remained open to all. The free Church was manifestly threatened not only with being defiled by false doctrine, but with being dissolved outright into a new paganism. When the Greek Church was so fortunate as to possess an orthodox Emperor, his power could maintain an purely external right-belief. But in the West, where there soon was no Emperor, and where one had learned that it was dangerous to have an Emperor who might become a heretic, where one consciously set oneself against the imperial power within the Church—how was the threatening danger there to be averted? How was the people to be preserved from the peril? How was the Kingdom of God, or the Church, to preserve itself in purity, and how was the Church to be preserved in unity?

Unhappily, one had fallen into such disdain for men and such trust in power that even in the West one could not conceive that there existed any other means of preserving the purity and unity of doctrine than power. Yet it was far from the case that this was clearly recognized or that one thought primarily of outward violence. On the contrary, the free Church felt that its danger lay in the spiritual anarchy of paganism, and it sought to set spiritual power against it. It is a spiritual Kingdom of God that is the goal of Catholicism, wherein all the peoples of the world are compelled to bow beneath the Church’s spiritual authority. The establishment of a doctrinal aristocracy was the first step upon the slippery path.

Directly over against the heathen world of culture and its natural quest for truth through the strained labor of the understanding, the Church set, with the might of authority, God’s revealed truth. Every free church must do this; for it knoweth whom it hath believed. Yet it did not stop there. In the course of time it fashioned the pure doctrine, which was not God’s revealed truth itself, but the Church’s recognition of the revealed truth. It was this pure doctrine of the Church which the free Church had the most natural temptation to endow, against the heretics, with the authority of God’s Word. It succumbed to the temptation. And when the Church surrounded its pure doctrine with sufficient condemnations, and when, with its bishops, it gathered itself together in unanimous agreement thereon, and when it did not refrain from taking to its aid the whole of antiquity’s human learning and subtlety in order to defend the pure doctrine, which still contained enough of the divine truth to be contrary to reason, then thereby the power of doctrine over the old cultural world was established. But the doctrine itself was thereby exposed to the same danger which is orthodoxy’s abiding fundamental defect. Human reason is taken as a helper in order to defend the doctrine, in order to render the doctrine clear and explicable, at last even in order to prove the doctrine. A foreign aid is taken, and it works itself little by little into the doctrine itself, which must undergo small and slow changes in order to fit the concepts of reason. Thus it was that the Catholic free Church for a time maintained itself, in order to preserve God’s truth, in order to hinder the influence of heathendom, but diverged from the simple way of truth to the way which leads — — — —\footnote{Here unfortunately a leaf is missing from the manuscript. For that reason, however, I have not thought it right to leave the piece unused. The readers will with some ease be able to infer the content of the missing lines, which do not in any essential degree disturb the course of thought in what follows. — Ed.}

And this prepared the way for that which one necessarily had to possess in the free church that desired a great multitude of people and great power over them, namely, a head of the Church who in every respect was ruler and leader.

From ancient times the Bishop of Rome had enjoyed a good reputation for pure doctrine. It was said of the Roman congregation and its bishop that it possessed an apostolic tradition or transmission, so pure and uncorrupted as that of no other. Thither one could go and learn the truth; from thence one could hear the apostolic testimony; from thence the illuminating word was spoken into the many intricate questions of the age. Little wonder, then, if the dignity which the Bishop of Rome possessed, the unshakable firmness with which he held forth the pure doctrine, was compared with the unworthy courtly snobbery of the Greek patriarch and his vacillation in doctrine, as the emperor set the tone. The Bishop of Rome gained by the comparison. He was regarded as the rock against which the restless waters of the world dashed themselves in vain, and as the spiritual successor of the rock-apostle Peter. To him men looked up, and the voice of Rome received decisive weight in the development of doctrine.

The free church had taken a great step forward toward its corruption when pure doctrine had become power, and Rome had become the dogmatic center. It was at once the tradition of the Roman Empire and the struggle against the paganism of the Roman Empire that had brought about this position. But the free church, which had lost its pure ecclesiastical character by becoming a mass church within the Roman Empire, was soon to be enticed still farther from the narrow way by admitting in great numbers the new barbarian peoples who at this time were streaming in over the Empire.

“Where the carcass is, there will the eagles be gathered together,” rang over the old Empire. New, fresh, and vigorous peoples hewed for themselves bloody paths toward the lands where the old world had unfolded its strength and squandered its life. They seized the strong man’s fortress and divided his spoil. In them the Church received a new mission field. The free church wills the conversion of the world; once it has become half Catholic, it wills the subjugation of the world. All peoples shall bow the knee to Christ—this is the free church’s involuntary lofty goal; all peoples shall bow to the power of the Church—this is Catholicism’s never-abandoned striving, after which it pursues with the restless craving of desire. When the migration of the barbarians began, Rome was still far from having become Catholic; before the new peoples had passed over to Christianity, it had become fully conscious of itself in its pursuit of universality (catholicity).

It was evident that just as the whole old world had been Christianized, so too the elements of the new world must be Christianized. The Church had learned that this was most quickly accomplished by imperial power. And however many glorious and uplifting images of true Christian mission the Church’s activity among the barbarians may present, it is nevertheless essentially by power and coercion of a new emperor, Charlemagne, that these peoples’ transition to Christianity was effected. Their political subjugation and their baptism en masse were one and the same thing. And the Church once again stood, as at Constantine’s transition to Christianity, face to face with great pagan masses who bore a Christian name. They belonged to the Church outwardly; they did not belong to it with the heart.

But the situation here was nevertheless very different. The heathen masses who followed Constantine bore within themselves the whole of the old pagan culture, and they were mighty to wage a struggle within the Church, which had as its consequence a powerful unfolding of ecclesiastical energies in defense of the pure doctrine. By contrast, the heathen peoples who, with the sword over their heads, had been compelled to receive baptism, who had lost their freedom and their religion at one and the same time, lay far more will-less in the hand of the Church. They offered, indeed, a natural resistance, such as the flesh always offers; yet it was essentially a resistance of lust and ungodliness, which, because it wounded the conscience, made the souls all the more easily predisposed, in other respects, to suffer under the Church’s power.

Here there arose a new danger that threatened the free Church. What was it to do with all these new members? They themselves could do nothing; they were just as helpless as they were ignorant. The Church could either labor to let the light of the Word shine upon them, so that they might be awakened to become independent and self-acting members of the Church; or it could let them lie helpless, and make use of their lack of independence in order to lay them under itself, always pretending that it was mighty to save them. The Church did indeed strive, as befitted the free Church, in its best moments and through its best powers, to raise these peoples to the maturity of manhood; but alas, it grew weary before the time, and the evil beginning—conversion by the sword—received an even more corrupt continuation.

For the free Church is in a peculiar manner tempted, precisely because it is Church, to forget that the children of the kingdom shall be cast out into the outer darkness. And the delusion of the saving Church allured with double force there, where a multitude of ignorant heathen—whom one took pleasure in comparing to children—lay subdued and bowed at the foot of the Church. To take them all into its bosom, to lift them into heaven, to pray them in unto God, and from God to dispense grace to them—that was the Church’s joyful thought of victory. There was no other way: the doctrine of grace-dispensing, saving Church had almost by force to press itself upon the Roman Church.

\textbf{And if we will truly go to the bottom of the entire Roman free-church system, wherein the kingdom of God was betrayed for power, we find that there are two gigantic seeds of injustice which were laid down in it at the very beginning, and which thereafter—constantly kept alive by the evil spirit of heartlessness and world-covetousness—have grown throughout the whole of its development and given it the character we know so well. The one is the authority of the pure doctrine,\footnote{Paper pope. — Present Ed} over against which no criticism is permitted—not even the judgment of God the Holy Spirit in God’s own Word—once the bishop of Rome has given the doctrine his sanction. The other is the Church’s power to grant salvation, so that all who receive grace only from the Church and in obedience to the Church are saved, whether they live in God or not.}

In the whole course of the Catholic Church it is these fundamental ideas that have asserted themselves. And are they far removed from any Free Church? Or is it not so, that the Free Church strives for the Kingdom of God, and does it not follow therefrom that it must strive to embrace all men? Is it not so, that the Free Church possesses the truth, and must it not then with authority maintain the truth? Is it not so, that men cannot be saved except through the means of grace which the Church has, and must not the Church therefore seek to direct the means of grace wherever it can, and thereby serve the salvation of men? No Christian dares to answer these questions with anything other than an affirmative Yes.

\textbf{But let there now enter into this one of the world’s foremost, enkindled loves for power and for dominion over the world, and all these things will, by the force of that seductive spirit, be transformed into their very opposite, while they yet retain their fair appearance.}

We know this one Free Church which has sought to develop itself as independently as is at all possible in the world. In reality we know no other. But this example is little cheering; for it seems to show with certainty that the Free Church must end in a complete annihilation of Christianity, just as is the case with Catholicism. For no one knows the frailty of his own heart, and if the desire to become great can lead to such results, where then is the Free Church which, according to all that we can see, does not have and must not have this desire? But the truth of the matter is that there lies an abyssal chasm between desire and love, and only the latter may guide the Church, not the former, if it is to walk on the way toward the Christian community.








