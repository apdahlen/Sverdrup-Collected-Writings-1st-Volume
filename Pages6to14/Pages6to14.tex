
\begin{center}
\includegraphics[width=0.9\textwidth]{OpenImage2.png}
\end{center}

\bigskip

\subsection{The Prophet Samuel}

[Source: Quarterly Journal. 7th Year. 1881. Pages 142–174. Separate offprint. “Vinggaardsmanden” Publishing House. 1903. — Ed.]

This section appears on pages 6–35 of the original volume. — Present Ed.

\bigskip


God’s people are, in one respect, distinguished above all other peoples upon the earth. They are an elect people, a people for God’s own possession. In other respects there is no difference between Israel and the other nations. There is the same human nature, the same lust toward sin, the same needs, the same struggle for existence and for life among all peoples, Israel not excepted. “The LORD did not set his love upon you, nor choose you, because ye were more in number than any people; for ye were the fewest of all people” (Deut. 7:7), says Moses, through whom the LORD had also said: “Understand therefore, that the LORD thy God giveth thee not this good land to possess it for thy righteousness; for thou art a stiffnecked people” (Deut. 9:6).

By nature Israel is like the heathen; but by election it is lifted up out of the multitude of the nations, and it became true what Balaam said when he saw the camp of Israel spread out before him: “Lo, the people shall dwell alone, and shall not be reckoned among the nations” (Num. 23:9).

Election made Israel unique among the peoples, and its life and history are unique in the history of the world. For election gave Israel a particular calling in the world and a particular trust from God. Israel alone received the special calling to become a blessing to all nations, and Israel alone received the special calling that the Word of God should be entrusted to it.

While all the other peoples desired the world and its goods, and sought to tear them to themselves and to rob one another, Israel, by the election, was called to love the world; and so far from robbing anything from the world or snatching anything from the poor race of mankind, the people of Israel were rather, in love and compassion and sorrow, to bear within themselves a promised treasure, which they should always be willing to share with all: the Word of God. With a rich gift for the world Israel went its heavy way and suffered for the peoples, while the peoples tumbled themselves in the lust of the world and reaped perdition from their sin.

But in the light of the Word of God Israel was, under its heavy pilgrimage, to see the goal clearly and shining before it; and while the heathen groped in the darkness, ever fearing the uncertain future, while they sought soothsayers and diviners and casters of lots and interpreters of dreams and oracles, in order to gain some clarity concerning the things to come, Israel was to have the light of prophecy burning and shining in a dark place. As Balaam had said: “— in thy time it shall be said of Jacob and of Israel, what God doeth.” (Num. 23:23).

It is the glory of the election that Israel may walk in the light of the Word. And herein the history of Israel, and of all God’s people, becomes different from all other history: that it not only has its goal—for all history has that—but that the goal is clearly and distinctly set before the people, so that the ancient people may lift up its gaze with certainty and courage toward the eternal kingdom which the Lord will prepare, and that by the light of prophecy the way to the goal is clearly marked out, so that it need not go astray.

But against the light of the Word rises the doubt and cleverness of the understanding, and against the heavy duty of the calling rises the lust of the flesh. And the whole history of Israel becomes filled with an unceasing struggle between flesh and Spirit, between prescription and faith, so that scarcely is there given on earth a more torn and storm-filled history than that of God’s people. God has set the calling and the destiny, and he continually raises up men and women in Israel who lift its glory and its responsibility high before a corrupt and perverse people; and again and again the great multitude in Israel spares itself and says: We will not walk in the way of the Lord! And thus there arise violent upheavals and heart-rending struggles between God’s glorious calling and the fleshly nature of the people.

There are great men in Israel, as in every other people; there are chieftains and standard‑bearers; there are leaders of the people and speakers for the people. But in truth great in God’s people is only the man or the woman who follows the Spirit’s calling and gives himself wholly over to the goal which God has appointed for His people. “A man in whom is the Spirit” (Num. 27:18) is the only one who is fitted to be leader and shepherd in Israel; he alone is the man “who may go out before their face and go in before their face, and who may lead them out and bring them in” (Num. 27:17). It is the Lord’s man whose calling it is to behold the Lord’s goal and follow its summons; and only he who himself has walked in the way of the Spirit can go before the people on their way toward the goal set by God.

Great abilities and glorious gifts do indeed make a man eminent in Israel, as among all peoples, even when the gifts are taken into the service of the flesh and of the world; but great in truth is only he who laid the great and glorious gifts down before the Lord and said, “Here am I; send me,” and who thereafter went forth in the Spirit’s power and the Spirit’s light before God’s people, and showed them where the Lord’s way was.

Such a Spirit‑borne leader in God’s people is Samuel. He is one of the few men in history who have been an instrument for the renewal of an entire people, and who, without bloodshed and deeds of violence, were able to gather what was scattered and unite what was divided. At his death Israel stood moved and established; for his own time he saw the people united and strong as never since the death of Joshua. Yea, though such great things had been accomplished and so much had been changed, Samuel could nonetheless step forward and receive from his people a full and complete testimony to righteousness and innocence in his entire public course.

It would be more than worthwhile to follow the struggle and labor of his life for his precious people. The Spirit of God has preserved the chief features of the great man’s life as a testimony for the latest generations.

\subsubsection{Israel in the Time of the Judges}

In a people that grows in a sound and vigorous manner, there are two powers which often seem to draw each its own way, yet which only when they are united become a true blessing for the people; these are freedom and unity. Israel had been driven together in need and oppression under the Egyptians’ harsh bondage. Moses had led the wretched people out into freedom, and his strong hand and superior might had preserved the people, corrupted by slavery and oppression, from again being split and scattered and thus perishing. Joshua had taken the governance over Israel with dauntless courage and had led it through struggle and distress, through victory and triumph, into the promised land.

But the two mighty leaders whom God had raised up for Israel were gone. There was none to take their place. Israel had to try to walk without the guidance which in its childhood had been so sorely needed, and which the Lord had given it in such abundant measure. Israel, which had been held together by the superior spirit of strong rulers and by the distress and peril amid which it was born, now entered a season of freedom and cohesion. It had been as a child under its mighty leaders, freely protected, yet in many ways still immature. After the death of the great men, Israel was to try to stand upon its own feet; and with the Law and the worship of God as constraining bonds, the people were to preserve in freedom what they had received as an unmerited gift.

Israel had strength enough to preserve its independence, if only it had faith enough. Israel’s strength lay in its election, and that election could become a living power only through faith. If faith failed, then election became merely a threatening responsibility instead of a saving force.

But Israel did not preserve faith; when distress was gone, when the pressure was over, when good days followed upon the heavy times of struggle, then Israel laid itself to rest; it grew fat and wanton, and in its soul, out of its freedom and advantage, it turned away from the Lord who had saved it, and from the God who had borne it upon eagles’ wings.

Israel sought freedom, not in God, but in the world. Toward the more vile gods of the Canaanites, and toward their immoral and cruel idolatry, they were drawn with irresistible desire. And as Israel sank down into the vices and idolatry of the Canaanites, it lost the power by which it had been upheld and knit together. The result was bondage and dissolution. The strong, youthfully vigorous people who under Joshua had marched into Canaan were captured and ensnared by the Canaanites’ licentiousness and worldliness. It renounced its election; it let its high goal slip out of sight; and thus it lost both its freedom and its unity. The foreign peoples found easy prey, and the individual tribes were severed from one another. The bonds of brotherhood, which were the one spirit, were gone, because the manifoldness of the world scattered the minds which could be united only in God.

But the Lord had not forsaken His people. Though it plunged itself like a prodigal son into the pleasures of the world, the Lord yet also let it taste the bitterness of the world’s bondage; and in need and misery Israel was brought unto the Lord and gained new experience of His faithfulness and grace, when the Lord raised up deliverers for His people who, in the power of the Spirit, led the people to victory and freedom. The judges and their work bear witness that the Lord remained faithful where Israel was unfaithful. The election and the covenant with the fathers stood firm where Israel failed.

But despite the work of the judges and the working of the Spirit of the Lord, Israel declined ever further. And the last judge, Samson, stood entirely alone against the enemies of God’s people. Yes, not only that. Seduced and swept away by Philistine heathendom, inflamed by Samson’s own intemperance, the men of Judah would even bind their champion themselves and deliver him into the power of the enemies. So low had the people sunk through the lust of the world and its bondage. And Samson himself, so great and richly gifted, fell so deeply that sorrow and shame cleave to his name and to his history forever. Samson, ensnared, sleeping with his head in Delilah’s lap while the hair of his head—the sign of his election—was shorn away, so that powerless and helpless he was given into the hands of the Philistines—this is an image of the people of Israel in that time. As Samson renounced his election for the lust of the world and sank strengthless into the hand of his enemies.

But Samson had strength to die for his people; alone, cast off and despised, trampled underfoot and abused, he had not forgotten his people; in the hour of his death his thought and his soul were with the precious, chosen Israel. But there is hope for that people whose sons go into death for it. Thus Israel still had a future before it. Yet it could not attain it except by the hand of the Lord; therefore the Lord sent them the voice of awakening through the mouth of Samuel.

The life of the people in the time of the Judges was in many respects wild and crude. Clearly and solemnly had the Lord admonished His people through the mouth of Moses and of Joshua, that they should not bind themselves to the Canaanite nations. Scarcely is there any people in the world who hastened more swiftly toward destruction through the corruption of morals than precisely the Canaanite peoples. Sodom and Gomorrah fell as the first terrifying example of the Lord’s judgment upon moral corruption and unnatural lust. Yet the punitive judgment did not put an end to the corruption of morals. The cruel and cunning Canaanites united the lust of the world with the wisdom of the world, and their cities, which distinguished themselves by wealth and prosperity, became also the homes of vice and excess, unto which ungodliness ascended unto heaven crying for vengeance. The divine mercy was therefore compelled to cry words of warning to young Israel, which entered into so corrupt a land, where even the temples of the idols had become dwellings of indecency.

But the solemn admonitions did not bear the desired fruit. It was not long before we find the sins of Sodom within the cities of Israel. The Israelite and the Canaanite youth took pleasure in one another’s company. Although there had stood a separation between the older generations of the two peoples, keeping them apart from one another, this separation no longer had the same meaning or the same effect upon the rising generation. The young men of Israel took Canaanite wives, and Israel’s daughters were given in marriage to the Canaanites. The wild, luxuriant life of the world seemed intent on swallowing all and everything.

Yet God had preserved His own. Though the corruption was great and widespread, the spark of faith was not extinguished, nor was the voice of testimony silenced, and the God-fearing life of Israel had not wholly ceased. There were still not only great and chosen instruments of the Lord, but also here and there small and inconspicuous souls who faithfully preserved Israel’s faith and lived by the blessed hope of the promises. There was Naomi, whose faith even drew the heathen woman into God’s people and the truth of His promises. Ruth left Moab, her land and her people, her kindred and her gods; and there was also a man such as Elkanah and his household, who observed the Lord’s commandments and ordinances and year by year went up to the sanctuary to appear before God at Shiloh, where the Tabernacle had stood since the days of Joshua.

Elkanah had two wives; and this unhappy violation of God’s order, which was and is so common among the peoples of the East, brought also in Elkanah’s house the usual restless consequences. Quarreling and bitterness, sorrow and tears have followed polygamy everywhere, and do so unto this very day. Women often have hard conditions and a burdensome life even among us, who boast that we have lifted them up out of subjection and bondage. When once the voice of the oppressed and tormented shall be heard in righteousness and truth, it will be made manifest that many a sorrowful and broken woman has borne more heart, even among us, than many have been able to imagine. Yet both the one and the other woman suffers among us wherever polygamy rules, and this becomes common and daily. Hannah, the mother of Samuel, was also made to prove this in full measure. She had no children, while the other wife rejoiced in sons and daughters.

Hannah’s lot was heavy and painful; yet the Lord had made it thus for Hannah, because He would reveal His glory and highly exalt the humbled and bitterly afflicted woman. “The barren woman shall dwell in the house as a joyful mother of children” (Psalm 113:9). And there was to be more blessing in the one son who was given her in her sorrow and prayers than in many sons and daughters over whom there had been jubilation and delight. The heart learns first in distress and sorrow the great mystery of yielding itself wholly to God; and only that life which has been consecrated through sorrow is able to become a full fountain of joy both for itself and for the people of God.

Hannah had learned to find consolation in sorrow and help in distress. To God she went with her affliction, and before Him she poured out her heart in His sanctuary. Israel’s high priest, Eli, was witness to her prayer in that hour of endurance. He was a respectable and agreeable man. Calm and without clamor was the stamp of his inner life. Appointed from childhood to be the Lord’s high priest and the spiritual leader of God’s people, he had doubtless also from childhood been accustomed to the Word of God and the temple of God. It was to Eli so familiar, so everyday, so commonplace. It seemed to follow of itself that he should be God-fearing, at least outwardly. And though he could not entirely avoid many serious thoughts and moving moments, yet earnestness and the struggle of life seemed not to have laid hold of his whole soul.

Thus it goes with many who have become priests and servants of God from outward considerations: for the sake of the family, for the sake of honor and reputation, or even for the sake of livelihood. Their hearts do indeed feel a seriousness and gravity; yet it often passes away with a sigh and a small prick of conscience, and of any share in full surrender to God in repentance and faith there is often nothing at all. Such priests indeed know the Word of God, but not the Spirit of God; and the inward spiritual movements and the struggle, the death of the soul and the fear of sin, the joy of the Spirit and the gladness of the life of God are things unknown to them. They account it enthusiasm or madness, or at least a peculiarity.


So it went also with Eli. He saw Hanna sunk down in prayer; her bitter affliction pressed her low, and her pain was too great for her to find words for it. But Eli did not understand what was stirring within this neglected heart, and his blunted spiritual sense did not so much as divine the mystery of distress and prayer. Therefore it seemed to him incomprehensible what it was that truly moved the woman, whose bitterly sorrowing and sorely oppressed mind was mirrored in her countenance, and whose lips moved according to the inner longings of her heart, without any audible word passing over them. Eli supposed her to be drunken, much as those who were witnesses of the outpouring of the Spirit upon the Apostles supposed that they were “full of new wine.” It is that spiritual sluggishness, that heartless indifference to the need or the joy of others, that materialistic incapacity to comprehend the nature of the Spirit, which here makes itself known. Eli’s official piety was unacquainted with such soul-rending workings of the Spirit as those Hanna experienced in her soul.

But Hanna had prayed as one who struggles for life, and she had vowed to the Lord the son for whom she had pleaded. There was full earnestness in her soul; and when she prayed for a son, it was not with the selfish desire to have him for herself, but with the true love’s willingness to offer him up to God and to His people. She had won peace in the struggle; she had conquered herself and the turmoil of her agitated mind; she had pressed through to the heart of the Lord and found mercy, and she was able calmly to answer Eli’s harsh and almost mocking words.

She told Eli what had been hidden from him. It was not wine or strong drink that made her so strange and incomprehensible to men; but anguish of heart and soul, which had opened her whole being before God, had made her manner a mystery to men. She had prayed, not as one who prays in order to display an ability before God, but as one who cries out of the depths unto the Lord.

It struck Eli in an instant. For the old man knew well that such was the prayer which breaks through all hindrances and presses its way up to the living God, Israel’s mighty Father. Thus had Jacob wrestled in the night by the Jabbok. Thus had Moses striven upon the mountain for the faithless people and for those chosen from among them. Eli understood that he stood face to face with one of the Lord’s elect, and his words were heart‑comforting for Hannah: “Go in peace: and the God of Israel grant thee thy petition that thou hast asked of him” (1 Sam. 1:17).

Comforted and relieved, Hannah went to her home. The encounter with the Lord in the sanctuary had given her new courage to take hold of life’s daily round of quiet struggles and the toil, burden, and weariness of household labor. In the great conflict and the bitter sorrow, her heart had been strengthened for the lesser sorrows and cares through which she yet must pass.

The Lord fulfilled Hannah’s prayer. She received a son and called his name Samuel; it means, “Heard of God”; for, said she, of the Lord have I asked him.

This time the son’s name is full truth. By prayer he was given; and in prayer the strength of his life was to abide. This is the secret of the life of God’s people, that God hears prayers. It is the expression of God’s fatherly heart toward His children. And it is faith’s proper victory over all unbelief and materialism, that it prays to the living, personal God and is heard. Around this stood the great conflict in the days of Samuel, as it repeats itself also now, after three thousand years have passed away.

God’s living personality and the conquering power of faith lie most simply and most clearly expressed in the word: God hears prayers. And around this all in Samuel’s life was to turn—to bear witness for Israel to the living, almighty God, the merciful and gracious Father, whom Israel was in danger of exchanging for dead idols, and to that distinctive power of faith which was to be Israel’s mark and the world‑overcoming strength.

Samuel, therefore, is the name that designates the champion of God over against the seductive powers of idolatry and materialism. Samuel is the name that designates the giant of faith over against the coldness of unbelief and the darkness of superstition.

