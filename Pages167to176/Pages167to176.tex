
\begin{center}
\includegraphics[width=0.9\textwidth]{OpenImage1.png}
\end{center}



\section{Memories From Norway}

Source: Fragment of a manuscript written during a stay in Norway in the summer of 1886. See the note on pp. 58–59. — Ed.

This section appears on pages 167–218 of the original volume. — Present Ed.

\bigskip

The year 1848, which throughout the whole of Europe was so remarkable a moment in time, may with some justice also be designated as a turning point in the development of the Norwegian people. True it is that Norway at that time was to no small degree a country lying apart unto itself, and that the surging waves from the great European currents of culture only slowly forced their way up into the poor land of cliffs and rock. Two circumstances contributed powerfully to hinder the penetration of foreign influences into the country: first, the land’s isolated situation, which had not yet been overcome by modern means of communication; and secondly, the general lack of popular education. As yet no popular newspapers were published, simply for the reason that there were too few who could read.

And yet there exists a mysterious interconnection in the development of nations; and even in Norway there were many among the country’s student youth who with ardor followed the struggle for freedom fermenting in Europe, and who with their whole heart drank in those thoughts of the people’s freedom and right to self‑government which at that time were striving to find their way into practical realization.

If one reflects more closely upon the matter, one will readily perceive that it was altogether natural that around the year 1850 there must needs arise a different age in Norway. For precisely at the midpoint of the century those men came to maturity and to participation in the public life of the land who had been born in the time when Norway won her political freedom. These young men were wholly and entirely reared and developed under the conditions which emerged after 1814. They had — to use an expression which has now been much mocked — with their mother’s milk absorbed a love of freedom and a popular consciousness. And if there should have been lacking something in the enthusiasm for Norway’s popular freedom in one or another narrow and impoverished home in the land, this lack had been powerfully remedied for these young men by the poets who in that time sang so beautifully of Norway’s mountains and fjords, of Norway’s freedom and people. Foremost among them was Henrik Wergeland, who kindled in the hearts of many young men an enthusiasm that was never again extinguished.

Every one will understand that a new age must needs begin when this new generation first gained influence in state, church, school, commerce, and all the various branches of national life.

We cannot therefore marvel if we find that a distinct division may be set in the history of Norway around the year 1850. And it is altogether reasonable that the ecclesiastical and spiritual development of Norway likewise took a clearly discernible turn about this time.

\subsubsection{The Revival through Hauge}

God, who governs the nations according to His wise counsel, granted the Norwegian people a singular providence. At the same time as the afflictions which preceded the birth of national liberty in Norway, there fell the blessed labor of Hans Nielsen Hauge. In the heavy years through which Norway passed at the beginning of this century, this chosen instrument wandered about the land and, amid much opposition and persecution, scattered the seed of life in the hearts of the people. And, praised be the Lord, manifoldly did the seed fall into good soil, and it sprang up and bore blessed fruit.

It is indeed easy to build the prophets’ tombs, while it is exceedingly hard to bow oneself wholly and sincerely beneath their earnest words. And therefore it is not always easy to understand the praise which now from many quarters is poured out also upon Hauge and the friends of Hauge; yet the fear of falling under this judgment must not deter us from praising the Lord, who looked upon our people in their distress and made Himself known to them, and who raised up in their midst a man of whom it may in truth be said, as of Peter and John: “Now when they saw the boldness of Peter and John, and perceived that they were unlearned and common men, they marvelled; and they took knowledge of them, that they had been with Jesus” (Acts 4:13).

Hauge took with full earnestness the Word of God, that one must obey God rather than men; and in defiance of human laws, and in defiance of the civil and ecclesiastical officials who sought to use these laws to hinder the testimony of Christ among the people, Hauge went his afflicted way through our nation, to blessing for thousands—yes, to living joy for many generations.

The cause which was thus expressed at the beginning of the century awakened ever a strong spiritual movement among the people. This is not the place to recount how Hauge’s practical application of the freedom of conscience over against all human commandments, and his patient sufferings for the right of conscience and of religion, became a seed of freedom in the whole course of our national development. Yet it ought nevertheless to be mentioned, especially now when in our own days one would deny Christianity all significance for popular freedom and assert it to be a yoke of bondage that ought to be shaken off as soon as possible. History bears witness just as surely that the Gospel of Christ, which gave the conscience its freedom over against all human commandments, is in truth the primal fountain from which the entire development of freedom in the human race has taken its origin. Thus it is also the case in the Norwegian people. Hauge’s awakening became, like every genuine ecclesial awakening, a mighty power to “put down the mighty from their thrones and exalt them of low degree,” to “raise up the poor out of the dust and lift the needy from the dunghill.” Not only did Hauge’s personal sufferings awaken a powerful sense of the injustice of persecution; but many of Hauge’s friends have in truth been brought forward by the power of the awakening and by the might of the Spirit to stand among the foremost of the land in the most diverse callings.

The movement which had been awakened through Hauge shared this with all true Christianity, that it was not so bound to his person that it ceased—or even declined in strength—at his death. It had life-power in itself through the Word and Spirit of God, so that it propagated itself and grew after this witness of the Lord himself had long since been laid in the grave.

Nor were witnesses lacking who were awakened through Hauge’s activity; they took up the work where Hauge left off, and there may be named a whole series of men who labored as revival preachers and as leaders among the awakened after Hauge. Most widely known are Jens Tønsgård from Tøier, Anders Haabe from Sognfjord, John Haugvaldstad, and several others. And that it might also this time be fulfilled what the Lord says: “Ye shall be my witnesses unto the ends of the earth,” He so ordered it that from this circle of witnesses there arose the thought of forming a Norwegian heathen mission society, so that the seed which was sown in the valleys of Norway now bears fruit far away among the blinded heathen peoples of Africa. Wondrous is the power of the Gospel of Christ—the power of God unto salvation for every one that believeth.

This strong spiritual movement within the Norwegian people was thus, through and through, a lay movement. Only few priests granted it their approval or encouragement; the great majority of the Norwegian clergy saw in the movement a more or less dangerous religious enthusiasm.

Even if these awakened ones were not, in the strictest sense, persecuted as Hauge himself had been, they were nevertheless all too often mocked and despised and deeply wounded by those who ought to have been their spiritual guides. Such treatment always gives pain, and especially among the Norwegian people. Our people have a warm heart and a strong sense of honor, and there was often laid deep bitterness in the depths of the heart when spiritual distress and the soul’s most sacred stirrings were met with cold contempt and condescending superiority.

Many examples are told of such heedless and unchristian conduct. A man in one of our deepest fjords, where nature itself already made the people somber and cautious, had been awakened to anxiety for his soul’s salvation. In his distress he gathered courage and went to the pastor. He found him standing and polishing his rifle after a hunting trip. The man explained his condition to the pastor, who listened from the door to the sitting room and called out to his wife: “Mother, give this man some medicine for his stomach; he is not well today.” Such a word is enough to plant hurt in a Norwegian man for his entire lifetime, and many a time enough to sow evil suspicion in a whole congregation against everything that bears the name of priest. If many such sufferings were cast into the hearts of the Norwegian common people, it cannot surprise us that in the period from 1814 to 1850 no strong trust grew up between the Norwegian congregation and the Norwegian clergy.

Therefore, however, no one should suppose that no change at all took place within the Norwegian clergy during this time. The point is only this: that there remained, on the side of Rationalism, a good measure of condescending superiority seated in the offices, and that this wrought immeasurable harm through the manner in which it confronted the awakened people. There prevailed a mutual suspicion and a mutual estrangement that could not be overcome in a short time. Shy and fearful, the awakened laity drew together by themselves; they feared nothing more than the deathly chill which so often struck them from the men of the Church.

Therefore it lasted a long time before any understanding arose between the awakened people and the more serious, Spirit-driven pastors who now began to go forth from Norway’s new University. The sins of the fathers were visited upon the children, and the ugly memories from the past still stood, like sharp thorns, in many places between pastor and congregation. And if it is therefore the case, in general, that in this period one finds a strong suspicion and great distance between the common people and the holders of office, then this is not least the case within the sphere of the Church.

But, as already indicated, there came forth from the new Norwegian University a new kind of pastors. And in order to understand the development after 1850, it is necessary to say a few words about this change within the Norwegian clergy.

\subsubsection{Hersleb and Stenersen}

This change stands in connection with the establishment of Norway’s University in 1811. Hersleb and Stenersen, the two first theological professors at the young institution of higher learning, were men who had completely broken with the lifelessness of Rationalism, and who sought—though often groping their way—to return to a biblical and Lutheran theology. Both of these men had received their theological education at the University of Copenhagen. Both had also received strong impressions from N. F. S. Grundtvig, who was approximately of the same age as they.

Those who have only heard Grundtvig’s name in connection with the childish notion of “the living Word,” which later came to be called “Grundtvigianism,” will have difficulty forming any true conception of the significance of Grundtvig’s first appearance for the Danish Church. After Rationalism had for a long time been practically and almost exclusively dominant in school and in church, Grundtvig stepped forward with a glowing enthusiasm for ancient Christianity and the Lutheran Reformation. Warm and fresh, he set upon the hollow and dry conception of Christianity and of human life held by the Rationalists; and like an wrathful prophet he ascended the pulpit and asked with the earnestness of a land in distress: “Wherefore has the Word of the Lord vanished from the house of the Lord?”

That question fell like a thunderclap through the whole Danish clergy when it rang out across the land; and in many young hearts this boldness and power awakened a strange enthusiasm and a mighty sympathy.

Among Grundtvig’s friends were also the two newly appointed Norwegian professors, C. B. Hersleb and S. S. Stenersen. Grundtvig held Hersleb in deep affection and even lingered with his next‑to‑last vision upon this youth. When Hersleb departed for Norway in order to become a teacher at the University, Grundtvig sent with him this wish to Norway’s University: “May God fill it with His glory, and cause it to become a house unto His Name!”

Hersleb’s friendship with Grundtvig brought him a measure of unpleasantness in Norway. For the provost Nikolai Wergeland, father of the poet Henrik Wergeland, wrote a book concerning “Denmark’s political crimes against Norway from 995 to 1814.” In this writing the provost declared that in Denmark one “hatched scornful barbs over humbled Norway,” and cited as an example “the doggerel which that dreamer Grundtvig permitted himself in his lay to the Dannering in 1812.” When, therefore, Hersleb came forward with a defense of Grundtvig, Wergeland returned the charge upon him, saying that “Hersleb and Stenersen are friends and defenders, and perhaps disciples, of Grundtvig and others whom ye seem to love more than your fatherland.”

To this Hersleb replied that he had no need to defend Grundtvig; “but it is a dear duty to me, upon this occasion, publicly to say that for several years, and in the most intimate manner, I have known Grundtvig, and that I love him deeply and sincerely esteem him for his eminent fear of God, his warm and loving heart, his strict righteousness, and his glorious spirit; that I owe him infinitely much, and that, so long as my heart beats, I shall hold him dear as a brother.”

From this one may see that Hersleb stood in a close relation to Grundtvig; yet this must not be understood in such a way as that Hersleb in any sense was what one now understands by a “Grundtvigian.” The distinctive views concerning “the living Word,” which Grundtvig later advanced, and in which a number of immature theologians and schoolteachers followed him, never won Hersleb’s approval. But in the struggle against Rationalism, and in the enthusiasm for the divine Word and for the history of the Church—above all for the Lutheran Reformation—Hersleb stood with his whole heart upon Grundtvig’s side.

Stenersen also, who was somewhat younger than Hersleb and Grundtvig, was influenced by Grundtvig and was particularly drawn to his view of history. He likewise declared himself publicly to be his friend and follower, and Bishop Pavels complained of his “excessive Grundtvigianism, intolerance, one-sidedness, and peculiar opinions.”

Hersleb was, when in 1813 he became a teacher of theology at the University in Christiania, not yet thirty years of age (born in 1784 in Alstahaug parish in Helgeland). Stenersen was, when in 1814 he became a university lecturer, about twenty-five years old (born 1789 at Mo in Søndre). For approximately twenty-one years these men labored together as teachers of pastors, until death called them both away. The younger, Stenersen, died first, namely on the 17th of April 1835, at an age of a little over forty-five years. Hersleb died on the 12th of September 1836.

In order to be in some measure just toward these men, one must remember that they received their entire theological education from Denmark, and that their upbringing falls within the darkest period which the Lutheran Church has experienced since the Reformation. When this is taken into consideration, one must rather marvel at the light and clarity that are to be found in these men’s work, in speech and in writing, than complain of the shortcomings that cling to them.

One of their very earliest and most gifted disciples was the well-known pastor W. A. Wexels. He writes of them: “Their legacy will be kindly; with a truthful voice it will proclaim that in the first decades after the establishment of Norway’s university there stood, in the midst of a faithless and violently agitated age, two friends side by side as teachers of pastors from the chair, and bore faithful witness to the old, guileless Christianity, despised by the world, as the divine source of light and life, of wisdom and righteousness, of power and freedom; and the Lord was with them, and their labor was blessed by Him and powerfully contributed to forming for Him instruments for the proclamation of His name among Norway’s mountains.”

From this one will understand that with the activity of these two men there necessarily followed a considerable change within the clergy. It was no longer rationalists who came forth from the university among the people. But on the other hand, one can by no means be surprised that the disciples of Hersleb and Stenersen did not at once find full confidence among the awakened people.

For in the first place one must observe that a whole number of the earliest theological graduates from Norway’s University were far from having borne witness to any personal Christianity. Whatever one may think of Hersleb’s and Stenersen’s own personal life in God, the influence that proceeded from them upon the studying youth was not, in the proper sense, of such a kind as aimed at a personal conversion unto God. They awakened reverence for the Word of God and for the history and doctrine of the Lutheran Church; they awakened interest in a scolarly pursuit of theology; they awakened enthusiasm for the truth and holiness of Christianity. But the entire conception of the relationship between theological professors and students was at that time not such that one could expect any real personal influence in a Christian sense.

Student life in this period was, on the whole, rather frivolous, and in part even wild and noisy. And the theological students did not always form an exception to the rule. According to old German and Danish custom, the students in Christiania at this time enjoyed an “academic freedom” which often degenerated into a perilous levity. When, then, the reports from the student world spread out over the countryside, and the strict Haugian believers heard that the students were often complete masters of drink, dance, and gaming, one cannot be surprised that their suspicion toward the clergy rather increased than diminished; and it was by no means easy even for the earnest, Christian pastors of Hersleb’s and Stenersen’s school to overcome this mistrust and to win for themselves a full, heartfelt confidence among the awakened laity.

Finally, there was a considerable difference between the conception of Christianity that accompanied the Grundtvigian movement and that which accompanied Hauge’s revival. These two kinds of views of the Christian life are about as different as the plains of Denmark and the mountains of Norway. Yet just as no one denies that the same Creator’s hand brought forth such great contrasts in nature, so it would be unjust to judge that true Christianity of the heart was to be found only among the awakened laity and not also among men such as Hersleb, Stenersen, and Wexels. Indeed, so great was the likeness in what was essential between these two Christian tendencies that, from the old rationalistic side, Hersleb, Stenersen, Wexels, and their like‑minded companions were regarded as “mystical enthusiasts and Haugianists.”

Although, therefore, both tendencies must be described as serious and Christian, the difference was nevertheless so great that they had exceedingly great difficulty in attaining any genuine intimacy with one another. In particular, the two movements’ understanding of the so‑called “innocent amusements” was so different and so contradictory that offense was continually given and taken in daily life. And even if one must acknowledge Wexels as a devout and sincere Christian, everyone nevertheless knows that throughout his long service in the Norwegian Church from 1819 to 1866 he never attained any general trust among the awakened laity. At times it seemed as though a reconciliation might come about; but then there always occurred one event or another that caused new suspicion to arise, and the distance became rather greater than less.

If now, as stated above, Wexels must be regarded as one of the foremost representatives of godliness and piety as these were found in the Hersleb‑Stenersen school, and if he could not reach the confidence of the laity, then it is self‑evident that the less warm‑hearted disciples of the same tendency never won their way into the heart of the Norwegian church people.

To this there must be added yet another consideration which must not be lost from sight. The Norwegian Constitution of 1814 had promised the Norwegian peasantry so much; but in reality they received, at first, very little. This is not the place to develop this matter more fully; yet one hardly errs in saying that when the farmers reflected upon their situation, they felt themselves disappointed. The civil officials continued in many places to govern in the old absolutist manner, and fairly soon there arose a considerable tension between officials and farmers. From this disappointment and tension there was born, in many places, even great bitterness; and since the clergy were officials of the Crown, it often came to pass that the pastors had to share in the disfavor that prevailed toward the whole bureaucracy.

One may therefore say, as the general result of these considerations, that although through the labors of Hersleb and Stenersen a new direction was indeed called forth within the clergy, a great and joyful change for the better, yet no reconciliation between the clergy and the believing laity came into being. Such reconciliation was sorely needed soon after the injustices committed against Hans Nielsen Hauge and his friends; but it did not come so soon or so easily as might have been desired. The chasm was too deep, the mistrust too pervasive. It was the clergy who had committed the wrong, and it was not willing to make toward the believing laity such an approach as was both right and necessary.

When Hersleb and Stenersen died (1835 and 1836), there came in their place as theological teachers at the University men who indeed belonged to the same general tendency, but who did not attain to the same height as the two earlier professors. Professors Dietrichson, Keyser, and Kaurin were not able to exercise any particularly strong influence upon the theological students. The theologians, as a rule, read for their examinations much as jurists and physicians did. Of any inner calling from God to be witnesses of Jesus Christ and shepherds of souls for His dearly bought congregation there was little talk.

Moreover, the minds of the student world in this period were greatly distracted by other interests. It was the time which the poet Welhaven called “Norway’s Dusk,” but which perhaps might more rightly be called Norway’s Awakening. The struggle between the Danish and the Norwegian tendencies in literature, art, and politics was so stormy and clamorous, and the studying youth were so powerfully shaken and stirred by it, that the theological students did not obtain sufficient quiet for deeper theological study and for an inward devotion to their divine calling. Henrik Wergeland and Johan Sebastian Welhaven were leaders of the two literary and aesthetic parties, and the influence which these two men exercised upon the youth far outweighed the influence exerted by the quiet and modest theological professors upon their theological students. If the studying youth possessed any higher interest than that of obtaining an examination and a comfortable home on a Norwegian parsonage, it was rather popular, national, and literary interests that animated them than Christian and theological ones.

The theologians of this period who possessed some deeper Christian view of life were still influenced by Grundtvig and Wexels; and among the people they were therefore still regarded with suspicious eyes. In America this opposition therefore at once comes to light in the struggle between Elling Eielsen and his friends on the one side and the “state pastors” on the other.

How deeply popular interests laid hold of the theologians of this time may be seen when one recalls how Jørgen Moe labored to collect folk tales and M. B. Landstad to collect Norwegian folk ballads. However meritorious these labors were in their own kind, they were nevertheless far from suited to inspire the awakened laity with any confidence in the seriousness of the Christianity in question.



