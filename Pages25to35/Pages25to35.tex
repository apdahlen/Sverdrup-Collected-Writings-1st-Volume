\subsubsection{The Deliverance from the Philistines}

The LORD judges His people; yet He does not bring shame upon His Name. The victory-proud Philistines returned home with the Ark of the Covenant: they believed that they had conquered both the people of Israel and its God. Meanwhile Israel sat in mourning and dejection, with tears in the eye and a heavy yoke upon its neck. It appeared as though all were lost—freedom, the land, the sanctuary, the worship of God, the Ark of the Covenant, the election itself. The great memories of the past were put to shame by the wretched present. Hope for the future had vanished. The glorious deliverance from the bondage of Egypt lay too far behind, and a humiliating diminution of the mighty past was Israel’s allotted portion. The holy tables of the Law, which were the pledge of God’s covenant at Sinai and which so surely guaranteed Israel’s future, were in the hands of the enemy. Had the word of the LORD become a lie, and the promises mere shadows, an empty sound upon the lips?

Such could not be. The truthfulness and faithfulness of the LORD, the honor of His Name and the earnestness of His love, could not perish; nor could the Philistines triumph over the Name of the LORD. While Israel, through Samuel’s intercession, learned to serve the LORD in spirit and in truth, and gathered that inward strength which would soon cast off the outward yoke, the Philistines were to learn that it is perilous to lay violent hands upon the people of the LORD and upon that which is His.

Judgment had begun with Israel; it would soon strike down upon the Philistines. The Ark of the Covenant, which they had borne home in pride and triumph, quickly became their terror and dread. They set it within the temple of Dagon, and the idol fell down; they moved it from city to city, and sickness and misery followed in its wake. As the LORD in former times, through signs and wonders, made known to the Babylonian world-empire that He was not impotent, though His people had been carried away into exile, so also did the LORD work wonders in the cities of the Philistines, lest the heathen should exalt themselves and imagine that their gods were greater than the God of Israel. For the LORD willed not Israel’s deliverance alone; He willed that Israel should be His servant unto the blessing of all peoples. Therefore it was as necessary that the nations should know His power as that Israel should know His mercy.

The Philistines knew of no other counsel than to send the Ark of the Covenant back again. Fear and terror had seized them, and they were compelled to seek to avert the wrath of Israel’s God from their land. The conquered Israel was, by the power of its God, mightier than the victorious Philistines. Thus the Ark of the Covenant was sent home again in a wondrous manner; yet to the childless tabernacle at Shiloh it never returned. The Ark found no dwelling until David brought it into the new tabernacle which he raised on Zion, and from there it was at last carried into the temple which Solomon built.

Meanwhile Samuel labored in the backslidden and oppressed Israel. With hope against hope he must surely have struggled through the long twenty years that passed before he could once more gather Israel for battle against the oppressors. Yet the hand of the LORD was with him, and his prayers and cries did not meet deaf ears. Wretched Israel, the poor and misguided people, was nevertheless still the LORD’s chosen, and the light of the promises was not extinguished.

Scripture is silent concerning this activity of Samuel; yet from its results we dare judge it, for a people does not rise from bonds and chains of slavery unless strong spiritual powers renew its heart and mind. And Scripture bears witness that after the course of twenty years all Israel went sighing after the LORD. Truly, Samuel cannot have been inactive, nor can his labor have been unfruitful. And what had been accomplished in secret and in stillness was soon to be gloriously revealed.

There was a mighty contrast between that day when Israel, defiant and proud, took the Ark of the Covenant with them into battle and treated the LORD as an idol, and this day, twenty years later, when Israel, broken and penitent, cast away its idols at Samuel’s word. And surely Samuel must have rejoiced in the Spirit on this great day of his life, when he summoned Israel to Mizpah, there to meet its God. With repentance in their hearts and confession of sin upon their lips, the diminished people gathered around their intercessor Samuel. Such a day a people does not experience many times. The people of God experienced once more a manly awakening, and unity and freedom were the fruits of that man for the people.

They experienced a salvation that recalled that glorious season of youth, when the Lord brought them forth out of the house of bondage. The Philistines, alarmed and embittered that the oppressed Israel dared once more to gather itself together, went up to smite the reconciled people and to crush its newly budding hope. They fought indeed against powers, though they perceived not the deep and hidden spiritual forces which now bound Israel to the Lord and united them into an unconquerable brotherhood.

Israel saw its ancient enemies draw near. Pride and haughtiness were gone; for they knew now better than ever how unworthy they were of the Lord’s salvation. But Israel knew its true weapons, and from a thousand mouths it sounded forth to Samuel: “Cease not for us, and do not refrain from crying unto the Lord our God, that He may save us from the hand of the Philistines!” A people in distress, a people in repentance and faith, Samuel beheld around him; and his heart was lifted up in fervent prayer for the beloved Israel, and the Lord heard him.

Victory and triumph were the Lord’s answer. The Lord fought for His people, and Israel shook the foreign yoke from its neck. Samuel was permitted to see how the spiritually awakened people, which had cast away its idols, now received freedom as a glorious gift upon the day of battle. Samuel’s prayer had been Israel’s weapon; and once again we perceive that above all else it was of consequence that the Lord should be known as He who hears prayers. The dead idols and the dead trust in the ark of the covenant were gone; the living, personal God, who suffers Himself to be found by penitent and broken hearts, revealed Himself gloriously upon the day of salvation. Samuel had been a witness to the awakening of the living knowledge of God and the living fear of God, in opposition to idolatry and dead superstition; he had seen living faith bear the fair fruit of freedom for the people; well might he set up the memorial stone Eben-Ezer and say: “Hitherto hath the Lord helped!” A happier day no prophet has beheld in his people.

\subsubsection{The Kingdom in Israel}

Samuel had gathered Israel into a living unity through his prophetic word, by the awakening voice that recalled their calling and election. Samuel’s prayer had given Israel freedom and victory. Samuel administered justice for Israel and judged with incorruptible integrity. All went well, and Israel experienced a season—indeed a continuing season—such as it had perhaps never known since the day they entered the promised land.

But Samuel grew old. He appointed his sons as judges over Israel; yet his sons did not walk in his ways. When the sons came to stand on their own, they succumbed to the customary temptation of judges: they bent justice for gain and gift. It must have been hard for the aged Samuel thus to behold his sons turning aside from the path of righteousness and integrity.

The people looked toward the future with anxiety. So near had they come out of the rent and troubled time of the judges, so fresh was the memory of their distress and misery, that they shuddered at the thought that, at Samuel’s death, it should all begin anew. This could not be allowed; something had to be done to avert a repetition of the calamities and sorrows of the time of the judges. And Samuel’s labor had borne so much and such fair fruit that it could not be permitted to be lost. The solidarity and concord which Samuel’s life had called forth had to be preserved. A popular ripening had begun, the goal of which Samuel himself had already seen, and it could not be arrested.

The people came to Samuel and asked for a king. They regarded him as a father; they would not and could not pass him by. Yet their desire was not according to Samuel’s spirit. Their words, “Give us a king to judge us, like all the heathen nations,” did not please God’s prophet. There lay within them a worldly spirit and a misunderstanding of the glory of God’s people. The old prophet would so gladly have spared God’s people the bitter experiences which the monarchy would bring upon it; he had so earnestly wished that God’s people might answer to its calling and, in the freedom of the land,\footnote{In this context, ``freedom of the land'' refers to the theological ideal of the Covenant: a state where Israel is free from human tyranny because it is governed directly by God and His Law. — Present Ed.} preserve its unity without being bound together by the coercion of kingship. The Lord was to be its King, the sanctuary its center, the word of the Lord its guide and rule, and the land itself its inner, binding power. He understood that the people desired a visible and tangible head to which they could look, and he understood that this was a step down from the height upon which the Law of Moses had sought to place the people. It lay heavy upon him that his labor should bear such fruit—that the people should long for a sensory, earthly kingship, when he had done all that stood in his power to teach them that the Lord was their King.

But Samuel did not act hastily or unreflectively in this matter; for he saw that a turning point had been reached in Israel’s history, and that the decision on this occasion would become of far-reaching significance for the whole future. “And Samuel prayed unto the LORD.” It was the LORD who must here give the answer, upon whose will everything depended. If He would allow that the people’s fleshly craving for an earthly king and a visible head should be satisfied, then Samuel must needs submit himself thereto. And if the LORD had time to tarry yet a thousand years before the kingdom of the land with the infinite King should be established upon the earth, then Samuel must also be content to behold the hope of the promise lying far off, and greet it with trustful expectation.

And so it came to pass. The LORD had greater compassion for Israel’s frailty than Samuel. All human impatience to behold at once the kingdoms of the Earthly Realm established was far from Him who knew so well that the time had not yet come; yet surely and calmly the fulness of time would dawn, when the kingdom of heaven with its divine Lord should be set up upon the earth. While the LORD waited, the people should have their will, and learn that though the earthly kingship might indeed accomplish much that was good, it was nevertheless not the true form of life for the people of God. And this new disappointment should prepare Israel for the kingdom of the Man, which the LORD in His own time would establish.

Therefore Samuel received the answer that the will of the people should be done. For although their desire was a misunderstanding of their spiritual calling and a rejection of the LORD, who was their rightful King, yet the people had not advanced in spiritual understanding, and they must learn by painful experience, since they would not suffer themselves to be instructed by the Word. And the LORD would take even this folly into His service, and cause rich spiritual instruction to flow to the people both through the glorious beginning of the kingship under David and Solomon, and through its lofty fall and humiliation under their successors.

But the people were not to remain merely in ignorance of what calamity they were bringing upon themselves. The people thought only of the firmness and unity, the peace and order, which the kingdom was to give. They knew well the firm and, as it seemed, immovable calm of which Egypt boasted in its famed constitution; they believed that through their kingship they would attain the same order. Samuel was to enlighten the people concerning the shadow-sides of the kingdom, concerning what the outward unity would entail in the loss of popular freedom. Israel had hitherto possessed a half-patriarchal constitution with the most unchecked freedom; they would come to feel something very different when royal power should bind and constrain them together.

Samuel portrayed for the people “the manner of the king.” With sharp and striking strokes he showed how the people’s right and freedom and exalted independence would be violated and distorted by the kingdom. It was a voice that sought to guard the people’s freedom against the people themselves. But it availed nothing. The people had no ear for the dark shadow-sides of royal power; they looked only back upon the turmoil they had experienced in the time of the Judges, and they looked toward this new plan as toward a deliverance from all evil. They would and must have an earthly king, and they received him. The Lord let the people’s will come to pass; for only thus could they enter the heavy school which they needed, in order to understand that the Kingdom of God is not of this world, and that its King is not the ruler of power and bondage, but the mediator of peace and freedom.

Thus the kingdom in Israel arises by the will of the people; but the people’s earthly desire carries with it its own punishment in the diminution of freedom and the bondage of the world over God’s freeborn people. Therefore the Lord lets the people’s will come to pass; for what the people desire as their true end, that the Lord would make into a means of chastisement and preparation for the Kingdom of Man.

\subsubsection{Saul and David}

Valiantly had Samuel sought to stay the people’s desire to obtain a king. For he had seen that it was an attempt to render God’s people carnal, which must rob it of the freedom to which its high spiritual calling entitled it. And it seemed to him grievous that the people who had but now been led out from the bonds of superstition and the tyranny of a depraved priesthood should now be led into a new bondage through their own folly. He had hoped that the day of spiritual freedom would dawn through the prophetic word. But the people were not yet ready to carry through this spiritual conception of society, and the Lord’s hour had not yet come.

And even as manfully as Samuel had set himself against the desire of the people when he first heard it, so firmly and earnestly did he now set about the carrying out of the people’s resolve, when he understood that in this manner the LORD would lead His people into a new period of their development, in order thereby to prepare them for that fulness of time which Samuel believed already to be at hand. The aged prophet did not withdraw himself dissatisfied and murmuring; but freely and vigorously he laid hold of the events which he would rather had never come to pass.

Saul was anointed king. From the lowly tribe of Benjamin, from the smallest tribe in Benjamin, himself an unknown and insignificant man, he was by Samuel known and acknowledged as the one whom the LORD had chosen to be prince over His people. It was in accordance with the LORD’s wondrous rule, that what is lowly and of no account in the eyes of men, the LORD has chosen. Samuel spoke long and earnestly with the young man, who until then seems not to have harbored any serious thought in his soul. The solemn hour of the anointing became a decisive turning point in Saul’s life; God changed his heart, and the Spirit of the LORD came upon him at the meeting with the prophets. Saul seemed destined truly to become a king after God’s own heart.

Samuel assembled Israel; he once again set before them that they had rejected the invisible King, who had so mightily saved and helped them before. Yet this brought about no change in the people’s decision, and the casting of lots decided who was to be king over God’s people. Saul was taken.

As yet there was no one who truly knew what dwelt within him. He was soon to have opportunity to show it. And Samuel waited, until Saul was thus revealed and made known to the people, before relinquishing his office as judge. The Ammonite king Nahash mocked one of Israel’s cities, and Saul gathered with haste, wisdom, and courage of Israel's men of valor, and within seven days he had an army in the field of three hundred and thirty thousand men, more than enough to bring to naught Nahash’s barbarous threat.

Saul had manifested his royal gifts and his noble nature; all Israel paid him homage. Samuel could therefore with confidence lay down his office as judge and take leave of the people in that capacity. Such are they who, like Samuel, have held power and yet voluntarily made way for later times and younger men. Rare indeed is such moderation among men; for the usual course is that even capable and broad‑minded men, in their old age, lose the ability to keep pace with the times and, with a withered spirit, remain fixed upon what they learned in a long‑vanished youth, having nothing left but reproaches for the generation that grows away from them. Not so with Samuel. Though it pained him to see the kingship established in Israel, yet justice and truth commanded him to withdraw from a position in which he could only become an obstacle to the authority and influence of the new government.

Samuel desired to speak to the people one final time. Two things he sought to accomplish by his speech on this occasion. He would settle accounts with his people, take leave of them as a faithful servant from his lord; no misunderstanding should remain between them, and no hidden injustice be left in any heart to embitter the old leader’s declining years. In righteousness he would part in peace from the place he had filled with such great honor.

The second aim he set before himself was to bring the people to the recognition that it was sin that they had demanded a king, that it was a fall from the Lord’s free grace, and that death and contrition before the Lord were the condition upon which the kingship could become a blessing to them. He would bow the heart and mind of his people, so that the kingship, which had been desired in sinful worldliness, might become a blessing to them by their receiving it in a godly spirit and with broken and penitent hearts. For it was not the kingship itself that was sinful; it was Israel that sinned by demanding an earthly kingdom when it had the Lord Himself as King and Savior. And if Israel could see its fall and begin anew to serve its way up again along the steep paths of Scripture toward spiritual communion with God, then all was gained that, for the present time, could be gained.

This was Samuel’s greatest struggle and his greatest victory. His mighty words and his dreadful sign smote the people with fear and terror, with weeping and anguish. Samuel humbled their proud minds by showing that Israel’s strife hitherto had been their idolatry toward the living God. In Egypt’s bondage, in the again and again recurring apostasies and subjugations of the time of the judges, it had been distress and prayer that taught Israel to find salvation in the Lord, and so it would continue to be. The kingship would not alter the matter. And that ruinous thought of pride—that now they should be able to help themselves without God—that they had to relinquish, else it would become a new callousness upon them that they had a king in their midst.

Then as thunder upon thunder and rain in the time of wheat harvest unexpectedly accompanied Samuel’s profound address, it became too much for the people. Had they truly forfeited their whole future? That thought struck their soul like lightning. And in distress and supplication they turned to Samuel: “Pray for thy servants unto the Lord thy God, that we die not; for we have added unto all our sins this evil, to ask us a king.”

And Samuel could then speak words of consolation—“Fear not!”—to the anxious people. Despite all disobedience, there should yet be a way of salvation for Israel; for the Lord forsaketh not His people.

Thus Samuel’s immediate aim was attained. The people went forward into the new age, which with the kingdom was to dawn upon them, with a God-fearing mind and a bowed heart.

But Samuel’s labor was not ended. Israel had received a king, and Israel had bowed itself under the word of the Lord and accepted the king in sincere fear of God; yet the king himself did not long remain obedient to the Lord. Saul’s exaltation had come too swiftly; there was no corresponding humility in his soul. His first victories emboldened him, and instead of simple humility beneath the word of the Lord, he began to act willfully and in defiance. This could not be. That vigorous natural disposition, which through the Spirit of grace might have become so great a blessing to the people, became ungovernable and reckless in its frenzy. Samuel was compelled to make the grievous journey to Saul and proclaim to him that the Lord had rejected him:

\begin{quote}
Hath the Lord as great delight in burnt offerings and sacrifices, as in obeying the voice of the Lord? Behold, to obey is better than sacrifice, and to hearken than the fat of rams. For rebellion is as the sin of witchcraft, and stubbornness is as iniquity and idolatry; because thou hast rejected the word of the Lord, He hath also rejected thee from being king.
\end{quote}

It was the final and unshakable decision, and despite all Saul’s pleadings, the matter stood by this word. Samuel went home, heavy of heart and sorrowful, and he did not see Saul again until the day of his death. It had been a hard blow for the prophet, who so gladly would have seen the people spared the new trials and afflictions which were now unavoidable. But the Lord would not have it go so easily. There was a great and glorious future awaiting Israel; yet no great age is ever born save through great tribulations. The first attempt at kingship failed, and however grievous it is to behold a people’s joyful expectations so swiftly shattered, there lay nonetheless in this chastisement a spiritual lesson for the nation: it was to gain a deeper insight into the mystery that the way of God’s people to glory is a way of the cross and of thorns. Both Israel and Saul had striven too lightly and too hastily up to the heights, and they had lost a serious apprehension of the Lord’s cause, in that they thought only of their own honour.

The Lord therefore had to choose a new man to be king, a man who in the hard school of suffering could learn to know his own heart and his God better than Saul; a man who, bowed beneath the cross, could grasp Israel’s high calling without being exalted and puffed up with pride; a man who by the Spirit of the Lord could go before Israel’s people and lead it forward on its way toward the promised goal of becoming a blessing to all nations. Samuel was to anoint David to be king after Saul. And once more it was the lowly who was exalted, and the despised who was taken into honour. And never has there been any earthly king who, like David, became a comforter of the wretched, a refuge time and again for the persecuted, a blessing to peoples in the most distant lands. Truly, a king for the people of the promise and in the spirit of the promise was the little David of the house of Jesse from Bethlehem. And had Samuel done only this one thing— in the clarity of the Spirit to go to Bethlehem and anoint the great Psalmist-king, whose songs were to resound in the hearts of hundreds and millions—Samuel would be reckoned among the greatest in the history of God’s people. And it has surely been, despite all the sorrow Samuel endured in his latter days over the fate of the poor, demented, and hardened Saul, an unspeakable consolation to him that David was the Lord’s chosen, and that great promises followed him of brighter and better days for Israel.

It was Samuel’s final act. He was to leave this great promise of the future to the people and to point forward toward happier times. He himself was not to experience them. He died before the struggle and the strife in Israel had yet come to an end. He was not to behold with his earthly eye the fair fruits of his labor and his sowing, as they so gloriously unfolded under David and Solomon. Yet so much will history with fairness say at his grave: What Samuel sowed in tears, Israel reaped with shouts of joy under the two great and glorious kings. The spiritually powerful element in Samuel’s work was the foundation upon which David’s kingdom was built, and the wellspring from which his psalmody drew its true life. Spiritual life, in opposition to dead worship of God and superstitious reliance upon the sanctuary and a crippled priesthood, was the standpoint of Samuel’s life; it was the standpoint of all prophets, for it is the work of the Spirit of the Lord at every time. And Samuel is, after Moses, the first and greatest prophet in Israel.

Once more Samuel’s voice is heard in night and darkness from the realm of death and the land of the grave, speaking to the hardened Saul. Or was it not Samuel’s voice that sounded on that dreadful night at Endor, when Saul’s embitterment turned into despair and hopeless terror? No one knows; for the secret of death no mortal has fathomed. Better to listen to the living prophetic voices than to brood over the sinister enigmas which the realm of death presents to us. Better to follow the call that awakens us to living faith and vigorous labor for the Lord’s cause and His people. Better to follow Samuel’s example in struggle and in prayer for the life of the land and the freedom of the land.

