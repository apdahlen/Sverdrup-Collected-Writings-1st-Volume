\subsubsection{The Awakening Missionary Spirit}

There was thus as yet no significant alignment in national direction between the clergy and the laity. Things continued to move along in their quiet course. Yet in silence, during this period, the conditions for great changes for the better were nevertheless being prepared. Among the theologians, even the interests of the common people which had been awakened became a bridge toward greater familiarity with the modes of thought of the laity in the religious and ecclesiastical sphere. And the Lord prepared for Himself instruments from among the theologians who, in a wholly different manner than the earlier theologians, were laid hold of by the seriousness and simplicity of Christianity, and whose labors were therefore to come into far greater harmony with the old revival. At the same time, the old revival advanced in the cause of missions in such a way that among many of the earnest theologians and pastors there was awakened an attention to, and respect for, its power and truth such as had never existed before.

It is around the year 1840 that these movements, which until then had unconsciously been striving toward a measure and a union, begin to be discerned. We reckon among these chiefly the founding of the Missionary Society in 1842, Schreuder’s commissioning in 1843, the efforts toward the repeal of the Conventicle Act, the Dissenter Law, and several other lesser matters. We must therefore here briefly touch upon these things, because they are necessary for understanding the ecclesiastical and Christian movement in Norway after 1850, which will properly be the subject of the present little treatise.

In the region around Stavanger, the Haugean revival, chiefly led by John Haugvaldstad, had maintained itself in a sounder and fresher condition than in several other places. It was this movement which, in part spurred on by other forces (Pastor Kielland and Mrs. Kielland), brought forth the Norwegian Missionary Society. At first it was a missionary association in Stavanger which, after having existed for a longer period, invited to a common meeting of friends of missions; and from this followed the establishment of “The Norwegian Missionary Society,” whose founding day is the 8th of August, 1842.

At the same time that the Norwegian Awakening thus began to recognize its world-embracing calling, the Spirit of the Lord drove Hans Schreuder (born 1817 in Sogndal in Sogn\footnote{Theological Candidate 1841. Authorized for ordination as a missionary pastor in 1842. Consecrated as Bishop of the Norwegian Mission in Zululand in 1866. Died 1882. — Ed.}) to offer himself for mission work. Long and quietly was Schreuder prepared for this work; and when, in 1841, he became a theological candidate, he was also ready to say: Here am I; send me.

From the beginning there was no connection between Schreuder and the friends of mission in Stavanger, nor with the newly founded Missionary Society. In order to support Schreuder, a committee was first formed in Christiania, consisting of Pastor Wærle and the theological professors; and this committee collected more than 2,450 specie-dollars for Schreuder’s mission. But after prolonged hesitations and negotiations it was arranged that Schreuder should be the missionary of the Norwegian Missionary Society. Yet this first step toward cooperation between the awakened laity and the earnest theologians was still so unclear, and the relationship so little marked by mutual confidence, that Schreuder in reality never understood the matter as though he were to stand in the service of the Norwegian Missionary Society, but rather held the conception that the Norwegian Missionary Society was essentially to serve him by gathering the necessary means for the furtherance of his work. It was therefore not so strange if there later came about a breach between these two, who in so little an organic manner had been brought into cooperation.\footnote{This refers to the breach between Schreuder and "The Norwegian Missionary Society" which occurred in 1873, and which led to the missionary work in Zululand being divided between this Missionary Society and "The Church of Norway Mission by Schreuder." — Ed.}

One must therefore take great care not to imagine that already at this event, as significant as it is, there had occurred a full reconciliation and union between the awakened laity and the Christian theologians in Norway. There remained still a deep gulf, and soon there appear traces of the opposition.

It was on the 1st of July 1843 that Schreuder, together with a helper, Thomassen, went aboard the ship which was to carry him away from his fatherland to England, from whence he intended to proceed to Cape Town.

\subsubsection{The Struggle for Religious Freedom}

Meanwhile, another struggle was also being fought, one which stirred no small agitation in the more serious minds in Norway.

This was the struggle concerning religious freedom in the general sense, and the particular freedom, within the Lutheran State Church, to hold what were called “godly assemblies.”

The one question was resolved by the Dissenter Act of the 16th of July 1845, and by the Act concerning the admission of Jews to the Realm of the 24th of September 1848. The other question was resolved by the Act of the 27th of July 1842, which repealed the so‑called Conventicle Ordinance of the 13th of January 1741.

Both of these matters were in part treated and considered simultaneously, as there is indeed a close kinship between them. For in reality the question concerns the freedom to labor for the cause of God’s Kingdom according to one’s best conviction, whether within or outside the State Church. This freedom, namely, was not present in Norway, despite its liberal Constitution.

In this connection, however, an item of information must be communicated which is far from being as generally known as it ought to be. Paragraph 2 of the Constitution, which deals with “the State’s public religion,” did not originally read as it is now to be found in the Constitution of the Kingdom of Norway. The original form of Paragraph 2, as it was originally adopted by the Constituent Assembly at Eidsvold, read as follows:

“The Evangelical‑Lutheran Religion shall remain the State’s public Religion. All Christian religious sects shall be granted free exercise of religion; yet Jews and Jesuits are still excluded from the Realm. Monastic orders shall not be tolerated. The inhabitants of the country who confess themselves to the State’s public Religion are obliged to raise their children in the same.”

This paragraph, which was drafted by the County Governor B. F. R. Christie, was adopted by 94 votes out of 111. But when the Constitution as a whole was read aloud and finally adopted, Paragraph 2 had in an incomprehensible manner been altered, so that the entire sentence concerning free exercise of religion for all Christian sects had been completely omitted. And none of the men of Eidsvold took notice of this, for it was too late. And Paragraph 2 of the Constitution read in this new form as follows:

“The Evangelical‑Lutheran Religion shall remain the State’s public Religion. The inhabitants who confess themselves to it are obliged to raise their children in the same. Jesuits and monastic orders shall not be tolerated. Jews are still excluded from admission to the Realm.”

Since, then, in this paragraph of the Constitution there stood nothing concerning the right of the sects to the free exercise of religion, and no new law had been given concerning them, the old Danish laws on these matters therefore remained in force, and according to these laws proceedings were carried out against them. Someone had to suffer for the cause of religious liberty in free Norway. And, strangely enough, it was the quiet, peaceable Quakers who were required to bear the sufferings that were deemed necessary in order that the cause of liberty might be advanced.

In Christiania and Stavanger, Quakers were fined for the benefit of the poor fund because they had buried their dead in the open field. And when the church authorities were considering granting royal permission to the Quakers in Stavanger to order themselves according to their religious conviction, Bishop Sørensen in Christiansand advised against granting such permission, on the ground that “irreligion would thereby receive nourishment, since many would presumably, by going over to the Quakers, seek to evade confirmation.” And yet these dreadful Quakers, against whom legal proceedings were raised, were but a handful; in Stavanger, six men and five women.

As already intimated, this stain upon Norway’s freedom was wiped away by the Dissenter Law of 1845, whereby Christian sects were, in all essentials, granted free exercise of religion in Norway.

This Dissenter Law was scarcely well received by the awakened laity. It was rather liberal-minded theologians and jurists who labored for it and carried it through. Chiefly there belongs much honor in this matter to Parish Priest Arup, later Bishop in Christiania, for the law’s passage. As chairman of the Storting’s Church Committee, he labored valiantly for the adoption of the law against several of the committee’s own members.

Different was the case with the repeal of the Conventicle Ordinance. It was the lay people who forced this matter through.

This Conventicle Ordinance of the 13th of January 1741 has become so significant in the history of the Norwegian Church, because it was by virtue of it that Hans Nielsen Hauge was arrested, held in prison for ten years, and finally sentenced to pay 1,000 riksdaler in fines and legal costs, after the lower court had first sentenced him to two years of penal labor (slavery) together with the costs of the case.

This same law, under which Hauge had been convicted in 1813, thus continued to threaten all lay preachers in Norway after his time. Yet so much had Hauge’s sufferings for God’s cause accomplished, that after his time there was great reluctance to bring the law into application against his successors. The law was also somewhat ambiguous; for it did not outright forbid “godly assemblies,” but restricted them to consist of “a few” only, and made them entirely dependent upon the clergy’s arbitrary consent.

Against this ungodly law the Norwegian church people rose with power and demanded its unconditional repeal. But the government, and a considerable number of the clergy, indeed wished the old law repealed, yet only on condition that a new law concerning godly assemblies should be put in its place.

Twice, in 1836 and in 1839, the Norwegian Storting resolved that the Conventicle Ordinance should be unconditionally repealed. Both times the King interposed his veto. In order then to set the matter upon a new track and avert a total defeat, the government resorted to its customary expedient of appointing a royal commission to draft a new law concerning “godly assemblies.” This commission consisted of Professor of Theology J. K. Dietrichson, District Judge Sørensen (who later died as a Councillor of State in 1853), and Pastor W. A. Wærles. And this commission did in fact prepare a new law concerning godly assemblies and the activity of lay preachers, which was presented to the Storting on the 7th of March 1842.

But it was too late. Five days earlier the well-known Ole G. Ueland, who was undoubtedly strongly seized and brought forth by the Haugean revival, had already introduced a proposal for the unconditional repeal of the Conventicle Ordinance. This proposal, which was also recommended by the chairman of the Church Committee, Pastor J. L. Arup, passed. It was the third time; and with that the notorious ordinance of the 13th of January 1741 was thus dead and buried. Imperishable honor follows Ueland’s memory for this.

As has already been noted, the Dissenter Act was thus added in 1845, and there had in this way been made a small beginning toward freedom for Christian movement and activity, both within and outside the Established Church. But so little was the principle of religious freedom at that time acknowledged by the clerical estate, that even earnest and God-fearing pastors disapproved of both of these actions of the Storting. Thus the Danish pastor Mau communicates a letter from the well-known Bishop R. Gislesen in Tromsø (born 1801, died 1861), in which it is said:

“In regard to the safeguarding of religious freedom I cannot even follow you, dear brother! It is our Dissenter Act that I chiefly have in view. I can never find it Christian to invite all manner of foreign religious confessors and sectarians to confuse our uninstructed people, and to encourage our own unlearded countrymen, to set themselves up as founders of new sects and church bodies. I have seen that it is quite other forces in human beings that have been loosed by the Dissenter Act than conscience, whose freedom can surely be had without the religious freedom which we now have in Norway. The only thing on which I rely is that God so governs matters that the foolish and untimely law does as little harm as possible, and that the Revival, which is from God, is not altogether halted and loses its blessing through the excesses to which the law invites.”\footnote{* Mau, Wærles’ Life and Work, p. 209.}

It is almost incredible that a Norwegian bishop could have written thus in 1860; yet so thought perhaps more than a few, even among the most earnest of the clergy. And one notices all too clearly that when there is talk of “our own native countrymen lacking theological training,” there are meant not merely the Dissenters and sectarians; rather, the expression is aimed at Lutheran lay preachers who have not passed through the regulated theological course at the University of Christiania.

We thus observe that although at this time, in the first place, a far more popular current runs through our entire official and clerical estate, and although, in the second place, within certain ecclesiastical fields a rapprochement is taking place between the theologians and the laity, yet there still remains much that divides. For on the one hand, the national enthusiasm of the university educated is still a stumbling block to the laity, and on the other hand there continues to be a distrust of the laity’s free activity. One senses that a portion of the clergy is well inclined, by new means, to restrain the free activity which the law no longer forbids.

But even if some had not been attentive to this, the opposition between the two tendencies within the Norwegian Church at this time came to manifestation in the sharpest possible manner in the textbook controversy.

\subsubsection{The Textbook Controversy}

In this matter the situation is such that already in the days of Rationalism, and after its time, an opposition arose against P. Saxtorph’s abridgment of Pontoppidan’s Explanation. Saxtorph, who died in 1803 as parish pastor at St. Nikolai Church in Copenhagen, had already in 1771 published this work, excellent in many respects. Little by little it was introduced into the great majority of schools in Norway. But new theological viewpoints gained ground among the educated classes; Saxtorph’s language began to grow antiquated, and from certain quarters there arose clamor for a change. Already in 1828 the Norwegian Church Department issued an invitation to those Norwegian pastors who might feel both ability and calling thereto, either to revise the abridgment or to compose a new textbook for use in the common schools and in confirmation instruction. On that occasion Wærles wrote in 1831: “That the Saxtorphian abridgment might stand in need of revision, this at least I will not deny; but the book has become dear to the people, and no one in our day provides a more heartfelt, more spirited, and on the whole more simple Christian exposition of Luther’s Catechism (which indeed ought surely to be laid as the foundation for every textbook intended for the religious instruction of children).”

Meanwhile, following the above-mentioned invitation from the Department, nine drafts for a new textbook in religion were submitted. They were examined, but none of them was approved. The government therefore appointed a commission, consisting of Professors Keyser and Kaurin and Pastor Wærles, to prepare a new edition of Pontoppidan’s Explanation. It was chiefly Kaurin and Wærles who carried out this work, and the result was that in 1842 there appeared a revised Explanation and a reviewed edition of Saxtorph’s abridgment. And in 1843 the government determined that after five years, that is, from 1848 onward, it should be permitted to use only one of these two books in religious instruction.

And thus the storm broke loose, so that one could plainly see what strong contradictions the Norwegian Church bore in her bosom. The revised Explanation bore a strong mark of the Grundtvigian leaven, which was so repugnant to the awakened people. And when this view now came forward with a claim to be the sole authorized one in the religious instruction of children, there arose against it a resistance as natural as it was justified. Indeed, exaggerated expressions were employed when it was said that the authors of the revised Explanation were “false prophets who deceive souls with a hope of salvation, until they awaken in hell”; yet in the matter itself the struggle against the revised Explanation was in reality a struggle for serious Christianity and for the necessity of conversion. For what the awakened people had against the book was that it: 1) in the doctrine of regeneration and of the Church made room for the mode of viewing all the baptized as regenerated and therefore as good and proper members of the Church, who no longer had need of regeneration; 2) taught a preaching of the Gospel to the dead, whereby these could be converted; and 3) struck out Pontoppidan’s judgment upon “dancing, games, comedies, tavern-going, and such things as are all sin in themselves”; finally, that 4) in the third article of faith it had placed only “a holy, universal Church” instead of Pontoppidan’s “to be a holy, universal, Christian Church,” or Luther’s: “a holy Christian Church.”

It is of no use to deny that when all these four points are taken together, one receives the impression that the simple Christian truth had been accommodated in such a way that it should not stand in too sharp a contrast to the whole state-church system. Forced baptism brought people into the Church, and the most frivolous worldly amusements could not be a testimony that one had fallen out of the grace of baptism and the state of regeneration.

This hard struggle bore first of all the praiseworthy fruit that nothing came of the foolish coercive regulation which the government had established in 1843, that after five years only the “revised” and the “reviewed” Explanation should be found in use in Norway. Secondly, it brought fully and clearly to light that it was not the intention of the Norwegian church people to allow the children’s catechism to be corrupted by the theologians. The standpoint of the laity had come forward in a manner that plainly made it understood that there would be neither peace nor cooperation in the Norwegian Church unless the Grundtvigianizing tendency among the theologians were pressed back.

With this controversy, then, we draw nearer across the way toward 1850. The forces which were to contribute to better understanding between the laity and the clergy had already been matured in the Lord’s school, and they stepped forth with a determination and a youthful freshness that promised great things. At the same time the people came forward with demands which found support among several liberal-minded theologians, and a vigorous and powerful ecclesiastical development seemed to be at hand.







