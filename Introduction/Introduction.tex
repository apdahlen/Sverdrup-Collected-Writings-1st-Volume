\section{Introduction}

By Wilhelm Petersen.

—

My first attempt to portray Professor Georg Sverdrup’s life\footnote{Professor Georg Sverdrup. A biographical sketch} was written under the strong influence of the recent loss caused by his sudden death, and without any preparation other than the state of mind into which this naturally placed me, and without any other material to work with than the information I either myself possessed or could easily procure. It was a fluent and imperfect, yet nevertheless sincere and well-intentioned attempt to honor the man whom Augsburg and Kristiania without hesitation place in the foremost rank among the greater spirits and fine, well-formed personalities of the Norwegian Lutheran Church.

A worker of the Church—indeed, a man of the country who is deeply loved—whatever position he may occupy, provided he accomplishes something of real worth, has in his lifetime two difficulties to contend with. The first is that his person, his character, his gifts, his spiritual orientation, and his view of life are, and by the nature of the matter must be, misunderstood and misinterpreted, distorted and unjustly judged by those who happen to be his opponents. Certain traits in an opponent can of course always be acknowledged; but it is almost always the purely formal or phenomenal aspects, that which appears on the surface, such as courage and bravery, aggressiveness and fearlessness in the struggle itself, skill and perseverance in conducting it, strong will and energy, and whatever else belongs to all that which outwardly gives the struggle its purely phenomenal character. To the same category of one-sided and inadequate judgment belongs, naturally, also the overestimation by admirers, supporters, and devotees of the person and his distinctive gifts. There are always fundamental features in a struggle, and it often brings forth certain powerful traits in those who fight. There is always more or less spiritual agitation, which causes the conflict to be viewed from a distance and the judgment to become unjust.

The second difficulty is that what in reality is the significant, the essential, and the characteristic element in a person is concealed or pushed into the background by that which the person in question happens, at the moment, to need most. Professor Sverdrup stood, to a high degree, like many of the Lutheran Church’s contending men, under both of these difficulties—despite a sober historical sense and a sound and just judgment of his person built upon it.

The remarkably brief introduction to Professor Sverdrup’s Selected Collected Writings is an answer to a request that I write such an introduction, made to me by the publisher of the work, Professor Andreas Høiland, who himself, in his introductory remarks to the individual sections, has touched upon several points that properly belong to a biography, and whose clear and perceptive view of Professor Sverdrup’s person and work will, to a considerable degree, help the untrained reader toward a more distant and comprehensive understanding of his life and the work to which he devoted it.

The publication of the work will moreover serve a twofold purpose: first, to refresh the memory of the man to whom the Lutheran Church in America owes so much; and second, to preserve for the congregation, in a more accessible and familiar form, the writings in which he set down the fundamental views upon which his entire ecclesiastical and popular work rested.

For the reasons mentioned in my first biographical attempt, it is still too early to write a biography of Professor Sverdrup that can place his life and his work in their proper historical context and apply the correct historical measure.

The purpose of this brief introduction is therefore only to emphasize what is essential in Professor Sverdrup’s work, especially as it emerges over time in his posthumous writings, which in themselves form the most beautiful and enduring memorial of him.


A historical personality may be viewed from different sides. It may be seen in the light of the conditions and surroundings that outwardly condition its appearance in history. Or it may be seen in the light of the aptitude for a particular life calling, recognized through abilities and education. Finally, it may also be seen in the light of the efforts undertaken to attain a certain goal, to solve a particular task.

Taken together, these three elements constitute the complete and full presentation of the personality from a historical point of view, while at the same time casting light upon the complex of spiritual currents and labors upon which the life in question has exerted both transient and lasting influence.

If we then look—if only at the titles—at the written works that are here before us from Professor Sverdrup’s hand and mind, we immediately gain the impression that we are standing face to face with a man who occupies himself exclusively with solid questions and tasks, both principled and practical. His spirit pressed downward and inward to that which acts as a driving force in the life of the Church and the congregation; or else to that which, as a historical result, revealed itself to his spiritual clarity of vision as a determining factor in the development of life, and from the recognition of which he never shrank back; or again to that which stood before his spirit as an ideal—radiant and full of promise, yet bearing within it the full content of prudence.

In this way I believe I have indicated what constituted the essential distinguishing mark of his rich intellect. He could never rest in the mere outward fact; rather, from that—which served as the given material for thought—he sought to return to the very essence, to the principle. And in this tireless search to find not only the true interconnection of things, but also their logical development from a principle firmly grounded in life, his entire scholarly orientation was revealed—one that was in all essential respects dialectical, that is, the method of Socrates as presented in Plato’s writings. How he admired the sharp, incisive, ironic quality of Socratic thinking—the truth-content recognized by its own keen observation, built upon an uncompromising and prejudice-free gaze!

Richly endowed by nature and developed through the study of Greek philosophy—which he had assimilated in an entirely original way—he was thereby fitted, as few are, to untie historical knots and to cast light upon historical life, which his spirit sympathetically lived through and absorbed in the fullest measure. Thus his presentation gained life and color, in that he himself lived along with everything he depicted. It would be difficult to find a more intimate balance between subjective and objective understanding and presentation.

But although he was scientific in his pursuit of truth, and although he possessed a rare power of exposition, he was nevertheless far too much of a fighting nature to settle into a purely formal recognition of truth or into a barren, dogmatic comprehension of results already attained. He did not merely work with the truth; the truth worked within him. And thus, in accordance with his entire temperament, he ranged beyond forms and formulas until he stood freely above the whole spiritual battlefield, where, like a field commander, he brought his heaviest artillery forward at the point where he saw the position most urgently required it. People easily misunderstand such a man. And if he is placed within what are commonly called “small” circumstances—where one must struggle against ingrained prejudices, where people’s narrower influences and self-interests render both friends and enemies, both supporters and opponents, cowardly, malicious, and wavering—then a rare strength of character is demanded, an almost unceasing vigilance, in order not to turn aside from the path once perceived as the one true way. Professor Sverdrup suffered under the difficulty of such a position to the same degree as many others who are endowed by nature with rare gifts and are called by God to wage the struggle of personality for truth and justice against complacency and class hatred, against indifference and self-interest.

And should one seek a clear, unmistakable testimony to the intensity with which he perceived this position of struggle, and at the same time to the clarity and sharpness of thought with which he was able to present his view, one could scarcely find anything better than the following brief piece, discovered among his papers and which was either a draft for a speech or an outline for a longer article. Any commentary is entirely superfluous. It is, however, fairly likely that he wrote this at a relatively young age. Both style and handwriting bear witness to that.

\begin{quote}
\bfseries
All hands on deck; draw the sword from the scabbard; let the cannon flash in the sun; strike upon the souls, that the sound of arms may be heard far across the land. The time is come; the air is heavy with thunder; let the thunder roll in the cloud and the lightnings strike with flash upon flash in the murk of night. The powers of the world are stirred; let the powers of heaven be stirred within us. God’s Church is in peril; let men girded with weapons guard her. Against the Lie we use sharp swords; against the Lie we bear shining helmets; against the Lie we take up bright shields. But the Lie is not outside God’s Church, but within her walls. Therefore awake, thou Christian Church, wherever thou art found; therefore awake, thou watchman of Zion, and sound the trumpet, that it may give a clear sound.

The Lie is this, that Christ is dead; for behold, He lives. The Lie is this, that Christianity is pure doctrine and the understanding of intellects and the clarity of thought; for behold, it is the life of love and the foolishness of love. The Lie is this, that the Church is an institution to rule over men; for behold, it is the home of freedom and the fellowship of the freed. The Lie is this, that the Church must employ the world’s falsehood and deceit in order to be cunning as the serpent; for behold, she is the holy witness of truth and simplicity. The Lie is this, that the world is the Church’s most dangerous enemy; for behold, it is the object of God’s love. The Lie is this, that the Church is pure; for behold, she is full of rottenness and stench. The Lie is this, that Christianity is the world’s hatred; for behold, it is the world’s love. The Lie is this, that the Church must fear the world; for behold, the world is overcome. The Lie is this, that the Church must go outward to find her life; for behold, her enemy is in her midst. Let us find him where his work is among us; let us know him for what he is. Let us meet him as Christ once did: I am He whom ye seek.

\end{quote}

“The truth in love” was perhaps as much as anything an expression of the whole direction of Professor Sverdrup’s spirit.\footnote{In this introduction, the editor’s narrative voice has been rendered in contemporary English for clarity, while quoted material originating with Professor Sverdrup has been translated in a more elevated register. This reflects the contrast of tone present in the Norwegian original between historical exposition and exhortative prose.}

For such a man there was scarcely room in Norway in the seventies. But with the elasticity of spirit that was his own, he soon made himself at home with the thought of working among his people outside Norway. It is not the intention here to portray the profoundly world-embracing character of Sverdrup’s understanding of Christianity and church work in contrast to that which prevailed in Norway at the time. He himself has portrayed this in masterly fashion in Memories from Norway, a work that must always form the point of departure for a fuller historical account of the reasons that brought Oftedal and Sverdrup to America. The justification of this understanding, and the correctness of these reasons, are testified to clearly enough by the subsequent ecclesiastical development in Norway.

All that remains is, in brief outline, to characterize the position into which Professor Sverdrup necessarily came when he thus tore himself away from the mother soil and “chose rather to suffer affliction with the people of God” than to hold a comfortable position in a state church in which, according to his ecclesiastical convictions, he could not feel at home.

This position was essentially the same for the two men whose names in Norwegian-American church history are so closely bound together that it is possible to separate them only in a purely formal sense. Confronted with the emerging church life in the new land, these two men stood searching for a way to move beyond the old-fashioned, purely institutional conceptions. And both of them, with the burning zeal and driving force of youth, threw themselves into the work of building—indeed, we may just as well say founding—the Church upon the congregation.

This word, which together with freedom and spirit has been so terribly abused, casts light upon all the work that constitutes the key to understanding all the views that bear the name Augsburg Seminary. It was a struggle on all sides that these two men undertook and carried on. That they should be regarded as disturbers of the peace, as dangerous men, was something anyone familiar with similar historical phenomena must have expected. Nor did they in any way attempt to seek shelter from the attacks that were directed against them from so many and varied quarters. That, moreover, only served to focus attention upon them all the more sharply. But it is neither too early nor too hasty—nor yet too late—to say here that these two men, whatever else may be said of them, represented an ecclesiastical vision and an ecclesiastical work that history has sanctioned.

Here it suffices to point to the writings of Professor Sverdrup that lie before us, which as clearly as any human presentation can express it contain two things: dissatisfaction with, and the inadequacy of, the old modes of thought and inherited forms; and the advocacy of a free and living congregation as the only foundation—both biblically and historically sound—for the building of a true kingdom of God on earth. For this Professor Sverdrup labored and fought; for this he suffered and died; and no one shall deprive him of the honor of having kindled a light in the Church upon which, even to this very day, many still look with joy and gratitude.
